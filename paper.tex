%\documentclass[9pt,preprint,nocopyrightspace]{sigplanconf}
%\documentclass[9pt,nocopyrightspace]{sig-alternate}
%\documentclass[pageno]{Submission-Info/jpaper}

%replace XXX with the submission number you are given from the ASPLOS submission site

\documentclass[pageno]{jpaper}
\newcommand{\asplossubmissionnumber}{XXX}

% The following \documentclass options may be useful:

% preprint      Remove this option only once the paper is in final form.
% 10pt          To set in 10-point type instead of 9-point.
% 11pt          To set in 11-point type instead of 9-point.
% authoryear    To obtain author/year citation style instead of numeric.

% \usepackage{epsfig, minted}

\usepackage{algorithm}
\usepackage{amsmath, amssymb}
\usepackage[noend]{algpseudocode}
\usepackage{enumitem}      % adjust spacing in enums
%\usepackage{subfigure}
\usepackage{multirow}
\usepackage{rotating}
\usepackage{wrapfig}
\let\bibhang\relax
\let\citename\relax
\let\bibfont\relax
\let\Citeauthor\relax
\let\citet\relax
\let\citep\relax
\let\citealt\relax
\let\citealp\relax
\let\Citet\relax
\expandafter\let\csname ver@natbib.sty\endcsname\relax
\usepackage[natbib=true,backend=bibtex,firstinits=true,style=numeric-comp,sorting=nyt,defernumbers,maxnames=99,maxcitenames=99]{biblatex}
\usepackage{balance}

\usepackage{pgfplots}
\usepackage{graphicx}


%\documentclass[letterpaper,twocolumn,10pt]{article}
%\usepackage{usenix}
%\documentclass[conference]{IEEEtran}
% \documentclass[pageno]{jpaper}
% %\usepackage[normalem]{ulem}
% %\usepackage{float}
% %\usepackage{url}
% %\usepackage{makeidx}
% \usepackage{graphicx}
% \usepackage{graphics}
% \usepackage{multicol}
% \usepackage{xspace}
% \usepackage{chngpage}
% \usepackage{wrapfig}
% %\usepackage{narrow}
% %\usepackage{fullpage}
% %\usepackage{subfigure}
% %\usepackage{subfig}
% \usepackage{multirow}
% \usepackage{rotating}
% \usepackage{color}
% %\usepackage{algorithmic2e}
% \usepackage{algpseudocode}
% \usepackage{algorithm}
% \usepackage{savesym}
% \usepackage{amsmath}
% \usepackage{amssymb,latexsym}
% \everymath{\displaystyle}
% \savesymbol{iint} 
% \usepackage{txfonts} 
% \restoresymbol{TXF}{iint}
% %\usepackage{mathptmx} % amsmath makes the paper longer
% \usepackage{mathrsfs}
% %\usepackage{leading}
% %\usepackage{algorithmic}
% %\usepackage{algorithm}
% %\usepackage{fullpage}
% %\usepackage[font={small,bf}]{caption}
% %\usepackage{hyphenat}
% %\usepackage{leading}
% %\let\labelindent\relax
% %\usepackage{enumitem}
% \usepackage[natbib=true,backend=bibtex,firstinits=true,style=numeric-comp,sorting=nyt,defernumbers,maxnames=10,maxcitenames=10,doi=false,isbn=false,url=false]{biblatex}
% \usepackage{balance}

% \newcommand{\asplossubmissionnumber}{XXX}


\definecolor{mygreen}{rgb}{0,0.6,0}
\definecolor{mygray}{rgb}{0.5,0.5,0.5}
\definecolor{mymauve}{rgb}{0.58,0,0.82}
\usepackage{listings}
\lstset{ %
  backgroundcolor=\color{white},   % choose the background color; you must add \usepackage{color} or \usepackage{xcolor}
  basicstyle=\scriptsize\ttfamily, % the size of the fonts that are used for the code
  breakatwhitespace=false,         % sets if automatic breaks should only happen at whitespace
  breaklines=true,                 % sets automatic line breaking
  captionpos=b,                    % sets the caption-position to bottom
  commentstyle=\color{mygreen},    % comment style
  deletekeywords={...},            % if you want to delete keywords from the given language
  escapeinside={\%*}{*)},          % if you want to add LaTeX within your code
  extendedchars=true,              % lets you use non-ASCII characters; for 8-bits encodings only, does not work with UTF-8
  frame=leftline,                  % adds a frame around the code
  keepspaces=true,                 % keeps spaces in text, useful for keeping indentation of code (possibly needs columns=flexible)
  keywordstyle=\color{blue},       % keyword style
  morekeywords={*,...},            % if you want to add more keywords to the set
  numbers=left,                    % where to put the line-numbers; possible values are (none, left, right)
  numbersep=5pt,                   % how far the line-numbers are from the code
  numberstyle=\tiny\color{mygray}, % the style that is used for the line-numbers
  rulecolor=\color{black},         % if not set, the frame-color may be changed on line-breaks within not-black text (e.g. comments (green here))
  showspaces=false,                % show spaces everywhere adding particular underscores; it overrides 'showstringspaces'
  showstringspaces=false,          % underline spaces within strings only
  showtabs=false,                  % show tabs within strings adding particular underscores
  stepnumber=1,                    % the step between two line-numbers. If it's 1, each line will be numbered
  stringstyle=\color{black},     % string literal style
  tabsize=1,                       % sets default tabsize to 2 spaces
  title=\lstname                   % show the filename of files included with \lstinputlisting; also try caption instead of title
}

\usepackage{pgfplots}
% options for pgfplots
\pgfplotsset{compat=1.8,compat/show suggested version=false}
\usetikzlibrary{calc,trees,arrows,patterns,plotmarks,shapes,snakes,er,3d,automata,backgrounds,topaths,decorations.pathmorphing,decorations.markings}
%\pgfplotsset{compat=newest}
\pgfplotsset{
   /pgfplots/bar  cycle  list/.style={/pgfplots/cycle  list={%
        {black,fill=black!30!white,mark=none},%
        {black,fill=red!30!white,mark=none},%
        {black,fill=green!30!white,mark=none},%
        {black,fill=yellow!30!white,mark=none},%
        {black,fill=brown!30!white,mark=none},%
     }
   },
}
% begin of externalization
\usetikzlibrary{external}
\tikzexternalize[prefix=out/]
\tikzexternalize
% don't externalize todonotes
%\makeatletter
%\renewcommand{\todo}[2][]{\tikzexternaldisable\@todo[#1]{#2}\tikzexternalenable}
%\makeatother
% end of externalization
\usetikzlibrary{patterns}
\usepgfplotslibrary{groupplots}
\pgfplotsset{
every axis label/.append style={font=\footnotesize},
tick label style={font=\footnotesize},
}
%\usepackage[normalem]{ulem}
%\setlength{\abovecaptionskip}{2pt plus 2pt minus 2pt}
%\setlist{noitemsep,topsep=0pt}

%\setlength{\itemsep}{0pt}
%\setlength{\topsep}{0pt}
%\setlength{\partopsep}{0pt}
%\setlength{\parsep}{1pt}
%\setlength{\parskip}{2pt}
%\setlength{\abovecaptionskip}{2pt plus 4pt minus 0pt}
%\setlength{\textfloatsep}{1pt}

\bibliography{seec}
%\renewcommand{\bibfont}{\scriptsize}
%\setlength\bibitemsep{0pt}

%reduce space around equations
\makeatletter
\g@addto@macro\normalsize{%
  \setlength\abovedisplayskip{4pt plus 2pt minus 1pt}
  \setlength\belowdisplayskip{4pt plus 2pt minus 1pt}
  \setlength\abovedisplayshortskip{4pt plus 2pt minus 1pt}
  \setlength\belowdisplayshortskip{4pt plus 2pt minus 1pt}
}
\makeatother

\newif{\ifanonymous}
\anonymoustrue

%\newcommand{\comment}[1]{}
\newcommand{\cutout}[1]{}
\newcommand{\smallcaption}[1]{\caption[#1]{{\protect\small \protect\bf #1}}}
\newcommand{\dids}{{\sc dids}}

%\graphicspath{{figs/}}
\newcommand{\SYSTEM}{CoPPer}


%\author{Connor Imes ~~~ Huazhe Zhang ~~~ Henry Hoffmann}{Department of Computer Science}{\{ckimes, huazhe, hankhoffmann\}@cs.uchicago.edu}

% \author{\IEEEauthorblockN{Connor Imes}
%   \IEEEauthorblockA{ 
%     University of Chicago\\
%     ckimes@cs.uchicago.edu}
%   \and
%   \IEEEauthorblockN{Huazhe Zhang}
%   \IEEEauthorblockA{ 
%     University of Chicago\\
%     huazhe@cs.uchicago.edu}
%   \and
%   \IEEEauthorblockN{Henry Hoffmann}
%   \IEEEauthorblockA{ 
%     University of Chicago\\
%     hankhoffmann@cs.uchicago.edu}
% }
% some useful shortcuts
\newcommand{\ie}{\textit{i.e., }}
\newcommand{\eg}{\textit{e.g., }}
\newcommand{\CC}{C\nolinebreak\hspace{-.05em}\raisebox{.5ex}{\tiny\bf +}\nolinebreak\hspace{-.10em}\raisebox{.5ex}{\tiny\bf +}}

% units for results
\newcommand{\us}{\,$\mu$s}
\newcommand{\ms}{\,ms}
\newcommand{\KB}{\,KB}
\newcommand{\MB}{\,MB}
\newcommand{\GB}{\,GB}
\newcommand{\MHz}{\,MHz}
\newcommand{\GHz}{\,GHz}

%\newcommand{\SYSTEM}{MEANTIME}
\newcommand{\system}{poet}
% new latex commands:
%   Remove long section
\newcommand{\PUNT}[1]{}
\newcommand{\TABLETWO}[1]{}
%   Label work to be done
\definecolor{gray}{gray}{0.75}
\newcommand{\TODO}[1]{\textcolor{gray}{\textbf{\ [TODO:\ #1]\ }}}
\newcommand{\TR}[1]{#1}
%\newcommand{\TR}[1]{}
%\newcommand{\TODO}[1]{}
\newcommand{\FIX}[1] {\textcolor{red}{\textbf{\ [FIX:\ #1]\ }}}
%   Referencing various pieces of the document:
\newcommand{\figref}[1]{Figure~\ref{fig:#1}}
\newcommand{\figsref}[2]{Figures~\ref{fig:#1} and~\ref{fig:#2}}
\newcommand{\figrref}[2]{Figures~\ref{fig:#1}--\ref{fig:#2}}
\newcommand{\secref}[1]{Section~\ref{sec:#1}}
\newcommand{\secsref}[2]{Sections~\ref{sec:#1} and~\ref{sec:#2}}
\newcommand{\eqnref}[1]{Eqn.~\ref{eqn:#1}}
\newcommand{\eqnsref}[2]{Eqns.~\ref{eqn:#1} and~\ref{eqn:#2}}
\newcommand{\eqnrref}[2]{Eqns.~\ref{eqn:#1}--\ref{eqn:#2}}
\newcommand{\insref}[1]{Instruction~\ref{ins:#1}}
\newcommand{\tblref}[1]{Table~\ref{tbl:#1}}
\newcommand{\tblsref}[2]{Tables~\ref{tbl:#1} and~\ref{tbl:#2}}
\newcommand{\appref}[1]{Appendix~\ref{app:#1}}

\newcommand{\algoref}[1]{Algorithm~\ref{algo:#1}}

% Custom hyphenation rules
\hyphenation{Ang-strom}

%\DeclareMathOperator{\minimize}{minimize}
%\DeclareMathOperator{\st}{s.t.}
%\DeclareMathOperator*{\argmin}{argmin}
%\DeclareMathOperator*{\argmax}{argmax}
\newcommand{\argmin}{\arg\!\min}
\newcommand{\argmax}{\arg\!\max}
\newcommand{\minimize}{minimize}
\newcommand{\maximize}{maximize}
\newcommand{\optimize}{optimize}
\newcommand{\st}{s.t.}

\newcommand{\app}[1]{\mbox{\texttt{#1}}}
\newcommand{\function}[1]{\mbox{\texttt{#1}}}
\newcommand{\variable}[1]{\mbox{\texttt{#1}}}


%-------------------------------------------------------------------------
\begin{document}

\date{}

\title{\SYSTEM{}: Controlling Performance with Power Capping}



%\conferenceinfo{OOPSLA 2012,} {June 3-7,2012, San Francisco, California, USA.} 
%\CopyrightYear{2013} 
%\crdata{ ACM 978-1-4503-1199-1112/06}

\maketitle 
%\begin{abstract}
  Mobile systems must deliver performance to interactive applications
  while simultaneously conserving resources to extend battery life.  There are two
  central challenges to meeting these conflicting goals: (1) the
  complicated performance/power trade-off spaces arising from hardware heterogeneity
  and (2) dynamic changes in application behavior and resource
  availability.  Machine learning techniques handle complicated
  optimization spaces, but do not incorporate models of system
  dynamics; control theory provides formal guarantees of dynamic
  behavior, but struggles with non-linear system models.  In this
  paper, we propose \SYSTEM{}, a combination of learning and control
  techniques to meet performance requirements on heterogeneous devices
  in unpredictable environments.  \SYSTEM{} combines a hierarchical
  Bayesian model (HBM) with a lightweight control system (LCS).  The
  HBM runs remotely, learning customized performance/power models.
  The LCS runs on the mobile system and tunes resource usage.  The
  Performance Hash Table (PHT) is the interface between the two and
  allows the LCS to apply the learned models in constant time.  We
  test \SYSTEM{}'s ability to manage ARM big.LITTLE systems.  Compared
  to existing learning and control methods, \SYSTEM{} delivers more
  reliable performance -- only 2\% error compared to 4.5-5.4\% for
  learning and 4.7\% for control -- and lower energy -- within 7\% of
  optimal on average as compared to 25-52\% for learning and 26\% for
  control. Furthermore, we demonstrate \SYSTEM{}'s ability to meet performance and energy goals in dynamic systems with phase changes and multiple applications running on the same system.

  \PUNT{ When multiple applications compete for resources, these
  numbers improve: 7\% error compared to 11-15\% for learning
    and 9\% for control and improvements of 2-20\% and 3\%,
    respectively, in energy efficiency.}
\end{abstract}

%\category{C.1.3}{Other Architectural Styles}{Adaptable architectures}
%\terms{Performance, Design, Experimentation}
%\keywords{Adaptive Systems, Self-aware Computing}
\section{sda}

\begin{figure*}[t]
  \begin{tikzpicture}
\definecolor{s1}{RGB}{228, 26, 28}
\definecolor{s2}{RGB}{55, 126, 184}
\definecolor{s3}{RGB}{77, 175, 74}
\definecolor{s4}{RGB}{152, 78, 163}
\definecolor{s5}{RGB}{255, 127, 0}

\begin{groupplot}[
    group style={
        group name=plots,
        group size=1 by 5,
        xlabels at=edge top,
        xticklabels at=edge top,
        vertical sep=5pt
    },
axis x line* = top,
xlabel near ticks,
major x tick style = transparent,
height=2.5cm,
width=0.88\textwidth,
xmin=0,
xmax=9,
enlargelimits=false,
tick align = outside,
tick style={white},
ylabel style={align=center},
ytick=\empty,
xtick=\empty,
xticklabels={},
yticklabels={},
ymin=0,
ymax=1,
]
\nextgroupplot[ylabel={$\mathsf{10\%}$},
y label style={rotate=270},
ylabel shift={12mm},
]
\addplot[thick,solid, color=black] coordinates {(0,1) (22,1)};

\nextgroupplot[
ylabel shift={12mm},
y label style={rotate=270},
ylabel={$\mathsf{30\%}$},
]
\addplot[thick,solid, color=black] coordinates {(0,1) (22,1)};

\nextgroupplot[
ylabel shift={12mm},
y label style={rotate=270},
ylabel={$\mathsf{50\%}$},
]
\addplot[thick,solid, color=black] coordinates {(0,1) (22,1)};


\nextgroupplot[
ylabel shift={12mm},
ylabel={$\mathsf{70\%}$},
y label style={rotate=270},
]
\addplot[thick,solid, color=black] coordinates {(0,1) (22,1)};

\nextgroupplot[
ylabel shift={12mm},
ylabel={$\mathsf{90\%}$},
y label style={rotate=270},
]
\addplot[thick,solid, color=black] coordinates {(0,1) (22,1)};

\end{groupplot}

\begin{groupplot}[
    group style={
        group name=plots,
        group size=1 by 5,
        xlabels at=edge bottom,
        xticklabels at=edge bottom,
        vertical sep=5pt
    },
axis x line* = bottom,
xlabel near ticks,
major x tick style = transparent,
xlabel={},
height=2.5cm,
width=0.88\textwidth,
xmin=0,
xmax=22,
enlargelimits=false,
tick align = outside,
tick style={white},
ylabel style={align=center},
ytick=\empty,
ymin=0.25,
ymax=1.75,
ytick={0,0.5,1.0,1.5,2.0},
yticklabels={,0.5,1.0,1.5,2.0},
legend cell align=left, 
legend style={ column sep=1ex },
ymajorgrids,
grid style={dashed},
]

\nextgroupplot[ybar=\pgflinewidth,
bar width=2.0pt,
legend entries = {{$\mathsf{DVFS}$},{$\mathsf{0.0}$},{$\mathsf{0.10}$},{$\mathsf{0.20}$},{$\mathsf{0.50}$}},
legend style={draw=none,legend columns=5,at={(.5,1.7)},anchor=north},
ylabel shift={0mm},
]
\addplot table[x index=0,y index=2, col sep=space] {img/ee/ee-dvfs.txt};
\addplot table[x index=0,y index=2, col sep=space] {img/ee/ee-copper-0.0.txt};
\addplot table[x index=0,y index=2, col sep=space] {img/ee/ee-copper-0.10.txt};
\addplot table[x index=0,y index=2, col sep=space] {img/ee/ee-copper-0.20.txt};
\addplot table[x index=0,y index=2, col sep=space] {img/ee/ee-copper-0.50.txt};

\nextgroupplot[ybar=\pgflinewidth,
ylabel shift={0mm},
bar width=2.0pt,
%ymin=.9,
%ymax=9,
%ytick={1,2,3,4,5,6,7,8},
%yticklabels={1.0,,,,5.0,,,8.0},
]
\addplot table[x index=0,y index=3, col sep=space] {img/ee/ee-dvfs.txt};
\addplot table[x index=0,y index=3, col sep=space] {img/ee/ee-copper-0.0.txt};
\addplot table[x index=0,y index=3, col sep=space] {img/ee/ee-copper-0.10.txt};
\addplot table[x index=0,y index=3, col sep=space] {img/ee/ee-copper-0.20.txt};
\addplot table[x index=0,y index=3, col sep=space] {img/ee/ee-copper-0.50.txt};

\nextgroupplot[ybar=\pgflinewidth,
ylabel={\footnotesize Energy Efficiency (Normalized)},
ylabel shift={0mm},
bar width=2.0pt,
%ymin=.9,
%ymax=9,
%ytick={1,2,3,4,5,6,7,8},
%yticklabels={1.0,,,,5.0,,,8.0},
]
\addplot table[x index=0,y index=4, col sep=space] {img/ee/ee-dvfs.txt};
\addplot table[x index=0,y index=4, col sep=space] {img/ee/ee-copper-0.0.txt};
\addplot table[x index=0,y index=4, col sep=space] {img/ee/ee-copper-0.10.txt};
\addplot table[x index=0,y index=4, col sep=space] {img/ee/ee-copper-0.20.txt};
\addplot table[x index=0,y index=4, col sep=space] {img/ee/ee-copper-0.50.txt};

\nextgroupplot[ybar=\pgflinewidth,
ylabel shift={0mm},
bar width=2.0pt,
%ymin=.9,
%ymax=9,
%ytick={1,2,3,4,5,6,7,8},
%yticklabels={1.0,,,,5.0,,,8.0},
]
\addplot table[x index=0,y index=5, col sep=space] {img/ee/ee-dvfs.txt};
\addplot table[x index=0,y index=5, col sep=space] {img/ee/ee-copper-0.0.txt};
\addplot table[x index=0,y index=5, col sep=space] {img/ee/ee-copper-0.10.txt};
\addplot table[x index=0,y index=5, col sep=space] {img/ee/ee-copper-0.20.txt};
\addplot table[x index=0,y index=5, col sep=space] {img/ee/ee-copper-0.50.txt};


\nextgroupplot[ybar=\pgflinewidth,
bar width=2.0pt,
ylabel shift={0mm},
xticklabel shift={0pt},
x tick label style={rotate=35, anchor=east},
xtick={1,2,3,4,5,6,7,8,9,10,11,12,13,14,15,16,17,18,19,20,21},
xticklabels={
{\scriptsize $\mathsf{blackscholes}$},
{\scriptsize $\mathsf{bodytrack}$},
{\scriptsize $\mathsf{facesim}$},
{\scriptsize $\mathsf{ferret}$},
{\scriptsize $\mathsf{fluidanimate}$},
{\scriptsize $\mathsf{frequmine}$},
{\scriptsize $\mathsf{raytrace}$},
{\scriptsize $\mathsf{swaptions}$},
{\scriptsize $\mathsf{vips}$},
{\scriptsize $\mathsf{x264}$},
{\scriptsize $\mathsf{canneal}$},
{\scriptsize $\mathsf{dedup}$},
{\scriptsize $\mathsf{streamcluster}$},
{\scriptsize $\mathsf{STREAM}$},
{\scriptsize $\mathsf{SWISH++}$},
{\scriptsize $\mathsf{HOP}$},
{\scriptsize $\mathsf{KMeans}$},
{\scriptsize $\mathsf{KMeans-Fuzzy}$},
{\scriptsize $\mathsf{ScalParC}$},
{\scriptsize $\mathsf{SVM-RFE}$},
{\scriptsize $\mathsf{\mathbf{Average}}$}},
]
\addplot table[x index=0,y index=6, col sep=space] {img/ee/ee-dvfs.txt};
\addplot table[x index=0,y index=6, col sep=space] {img/ee/ee-copper-0.0.txt};
\addplot table[x index=0,y index=6, col sep=space] {img/ee/ee-copper-0.10.txt};
\addplot table[x index=0,y index=6, col sep=space] {img/ee/ee-copper-0.20.txt};
\addplot table[x index=0,y index=6, col sep=space] {img/ee/ee-copper-0.50.txt};

\end{groupplot}
\end{tikzpicture}
  % \vskip -1em
  \caption{Application energy efficiency across different controller configurations for a range of performance targets, normalized to an perfect \emph{ondemand} DVFS governor (higher is better).}
  \label{fig:ee}
\end{figure*}

\clearpage
\printbibliography


\end{document}


