\section{Related Work}
Energy has long been an important resource for mobile and embedded
computing.  Several OSs have supported energy as an allocatable
resource \cite{Koala,Cinder,PowerContainers}.  Others have specialized
OS constructs to monitor \cite{quanto} and reduce
\cite{JouleGuard,flinn99,grace,grace2,Drowsy} energy for mobile and
embedded applications. We examine related work applying learning and
control to energy management.


\noindent \textbf{Offline Learning} approaches build predictors before
deployment and then use those fixed predictors to allocate resources
\cite{Yi2003,LeeBrooks2006,CPR,ChenJohn2011,petabricksStatic}.  The
training requires both a large number of samples and substantial
computation.  Applying the predictor online, however, is low overhead.
The main drawback is that the predictions are not updated as the
system runs: a problem for adapting workloads.  Carat is an offline
learner that aggregates data across multiple devices to generate a
report for human users about how to reconfigure their device for
energy savings \cite{carat}.  While both Carat and \SYSTEM{} learn
across devices, they have very different goals.  Carat returns very
high-level information to human users; \eg{} update a driver to extend
battery life.  \SYSTEM{} automatically builds and applies low-level
predictions to save energy.

\noindent \textbf{Online Learning} techniques observe the current
application to tune system resource usage for that application
\cite{Li2006,Flicker,ParallelismDial,Ponamarev,petabricksDynamic,LeeBrooks}.
For example, Flicker is a configurable architecture and optimization
framework that uses online prediction to maximize performance under a
power limitation \cite{Flicker}.  Another example, ParallelismDial,
uses online adaptation to tailor parallelism to application workload
\cite{ParallelismDial}.



\noindent \textbf{Hybrid Approaches} combine offline predictions with
online adaptation
\cite{Zhang2012,packandcap,Winter2010,dubach2010,Koala,Cinder,
  wu2012inferred}.  For example, Dubach et al.  use a hybrid scheme to
optimize the microarchitecture of a single core \cite{dubach2010}.
Such predictors have also been employed at the operating system level
to manage system energy consumption
\cite{Koala,Cinder,wu2012inferred}.  Other approaches combine offline
prediction with online updates \cite{JouleGuard,Bitirgen2008,Ipek}.
For example, Bitirgen et al use an artificial neural network to
allocate resources to multiple applications in a multicore
\cite{Bitirgen2008}.  The neural network is trained offline and then
adapted online to maximizes performance but without considering
energy.


\noindent \textbf{Control} solutions can be thought of as a
combination of offline prediction with online adaptation.  Their
formal properties make them attractive for managing resources in
operating systems
\cite{Steere99,KaramanolisEtAl-2005a,Hellerstein2004a}. The offline
phase involves substantial empirical measurement that is used to
synthesize a control system
\cite{grace,Wu2004,Chen2011,POET,ControlWare,Agilos,Rajkumar,Sojka,Raghavendra2008}.
Control solutions work well over a narrow range of applications, as
the rigorous offline measurement captures the general behavior of a
class of application and require negligible online overhead.  This
focused approach is extremely effective for multimedia applications
\cite{grace,grace2,flinn99,flinn2004,xtune,TCST} and web-servers
\cite{Horvarth,LuEtAl-2006a,SunDaiPan-2008a} because the workloads can
be characterized ahead of time to produce sound control.

The need for good predictions is the central tension in developing
control for computing systems.  It is always possible to build a
controller for a specific application and system by specializing for
that pair.  Prior work addresses the need for accurate predictions in
various ways.  Some provides control libraries that require
user-specified models \cite{ControlWare,Sojka,Rajkumar,POET,SWiFT}.
Others automatically synthesize both a predictor and a controller for
either hardware \cite{josep-isca2016} or software
\cite{ICSE2014,FSE2015}.  JouleGuard combines learning for energy
efficiency with control for managing application parameters
\cite{JouleGuard}.  In JouleGuard, a learner adapts the controller's
coefficients to uncertainty, but JouleGuard does not produce a new set
of predictions.  JouleGuard's computationally efficient learner runs
on the same device as the controlled application, but it cannot
identify correlations across applications or even different resource
configurations.  \SYSTEM{} is unique in that a separate learner
generates an application-specific predictions automatically.  By
offloading the learning task, \SYSTEM{} (1) combines data from many
applications and systems and (2) applies computationally expensive,
but highly accurate learning techniques.


