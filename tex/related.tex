\section{Related Work}

We discuss related work in managing resources to meet performance
goals with and reduce energy.  We first discuss machine learning
approaches, then control theoretic ones.  

\subsection{Machine Learning}
There are a huge array of different learning techniques that are
applicable to different prolems.  As discussed in in \secref{learning}
we break learning for reasource management into three categories:
offline, online, and hybrid approaches.  

\subsubsection{Offline Learning}
The offline approaches build models before deployment and then use
those fixed models to allocate resources
\cite{Yi2003,LeeBrooks2006,CPR,ChenJohn2011,petabricksStatic}.  In
these approaches, the model building phase is generally very expensive
requiring both a large number of samples and substantial computation
to turn those samples into a model that accurately captures the
relationship between the observed features and the behavior to be
estimated.  Applying the model online, however, tends to be low
overhead.  The main drawback is that the models are not updated as the
system runs, so there is no chance to correct mistakes or adapt to
specific workloads.

\TODO{Vijay's webbrowsing learner?}

\subsubsection{Online Learning}
Online techniques use observations of the current application to tune
system resource usage for that application
\cite{Li2006,Flicker,ParallelismDial,Ponamarev,petabricksDynamic,LeeBrooks}.
For example, Flicker is a configurable architecture and optimization
framework that uses only online models to maximize performance under a
power limitation \cite{Flicker}.  Another example, ParallelismDial,
uses online adaptation to tailor parallelism to application workload
\cite{ParallelismDial}.



\subsubsection{Hybrid Approaches}
Some approaches have combined offline predictive models with online
adaptation
\cite{Zhang2012,packandcap,Winter2010,dubach2010,Koala,Cinder,
  wu2012inferred}.  For example, Dubach et al.  propose such a combo
for optimizing the microarchitecture of a single core
\cite{dubach2010}.  Such predictive models have also been employed at
the OS level to manage system energy consumption \cite{Koala,Cinder}.
\cite{wu2012inferred}.


Still other approaches combine offline modeling with online model
updates \cite{JouleGuard,Bitirgen2008,Ipek}.  For example, Bitirgen et
al use an artificial neural network to allocate resources to multiple
applications in a multicore \cite{Bitirgen2008}.  The neural network
is trained offline and then adapted online using measured feedback.
This approach optimizes performance but does not consider power or
energy minimization.  LEO, the system we extend in this paper, also
uses a combination of offline and online approaches.  LEO collects
data about a number of applications offline and combines that with a
small number of observations made online for the current application
\cite{LEO}.

\subsection{Control}
Almost all control solutions can be thought of as a combination of
offline model building with online adaptation.  Usually the model
building involves substantial empirical measurement and statistical
regression to build a model (analogous to determining the value of $k$
in \eqnref{clock}) that is then used to synthesize a control system
\cite{Wu2004,TCST,Chen2011,PTRADE,Heartbeats2,ControlWare,Agilos,Rajkumar,Sojka,Raghavendra2008}.
Over a narrow range of applications the combination of offline
learning and control works well, as the offline models capture the
general behavior of the entire class of application and require
negligible online overhead.  This focused approach is extremely
effective for multimedia applications
\cite{grace2,flinn99,flinn2004,xtune,TCST} and web-servers
\cite{Horvarth,LuEtAl-2006a,SunDaiPan-2008a} because the workloads can
be characterized ahead of time so that the models produce sound control.

Indeed, the need for good models is the central tension in developing
control for computing systems.  It is always possible to build a
control for a specific application and system by extensively modeling
that pair.  More general controllers which work with a range of
applications have addressed this issue with models in several ways.
Several approaches provide control libraries that encapsulate control
functionality and require users to input a model
\cite{ControlWare,Sojka,Rajkumar,POET}.  Other approaches
automatically synthesize both a model and a controller for either
hardware \cite{josep-isca2016} or software \cite{ICSE2014,FSE2015}.
% Still others start with a general model, and customize it at runtime
% by either classifying an application's behavior and using a model
% for that class \cite{MICRO2016} or using some computationally ef
\TODO{There are some approaches that update a model using ``online''
  learning. We should talk about those, but only if we have data that
  shows that online is worse than LEO.  I think we will have that in
  the paper, though.}  Our approach is unique in that a remote server
generates the model automatically.  By offloading the learning part to
a remote server, we are able to (1) combine data from many
applications and systems and (2) apply computionally expensive, but
highly accurate learning techniques.


Perhaps the most similar approach to ours is Carat \cite{}.  Carat
aggregates data across many mobile devices and sends a report to human
users about how to configure their device to increase battery life.
While both Carat and our approach learn across devices, they have very
different goals.  Carat's goal is to return very high-level
information to human users; \eg you should update a driver to extend
battery life.  Our approach returns lower-level models to another
automated system that will apply those models to save energy.

