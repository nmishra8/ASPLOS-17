\section{Related Work}

\paragraph{Machine Learning}

%We break learning for resource management into 3 categories: offline,
%online, and hybrid approaches.

\noindent \textbf{Offline Learning} These approaches build predictors
before deployment and then use those fixed predictors to allocate
resources
\cite{Yi2003,LeeBrooks2006,CPR,ChenJohn2011,petabricksStatic}.  The
training phase is expensive, requiring both a large number of
samples and substantial computation.  Applying the predictor online,
however, is low overhead.  The main drawback is that the predictions are
not updated as the system runs: a problem for adapting workloads.
\PUNT{A good example of an offline approach applies learning to render
  web pages on mobile systems with low energy \cite{reddiHPCA2013}. It
  builds an offline model mapping web page features into performance
  for different core types.  When a new page is downloaded, the system
  estimates the resources needed to render the web page and uses the
  lowest energy configuration that meets user satisfaction.  This
  approach handles the complexity of allocating resources to webpage
  rendering, but cannot address dynamics; \eg{} when other apps run
  concurrently with the web browser.}  Carat is a good example of an
offline learner that aggregates data across multiple devices to
generate a report for human users about how to configure their device
to increase battery life \cite{carat}.  While both Carat and \SYSTEM{}
learn across devices, they have very different goals.  Carat returns
very high-level information to human users; \eg{} update a driver to
extend battery life.  \SYSTEM{} automatically builds and applies
low-level predictions to save energy.

\noindent \textbf{Online Learning} Online techniques observe the
current application to tune system resource usage for that application
\cite{Li2006,Flicker,ParallelismDial,Ponamarev,petabricksDynamic,LeeBrooks}.
For example, Flicker is a configurable architecture and optimization
framework that uses online prediction to maximize performance under a
power limitation \cite{Flicker}.  Another example, ParallelismDial,
uses online adaptation to tailor parallelism to application workload
\cite{ParallelismDial}.



\noindent \textbf{Hybrid Approaches} Some approaches combine offline
predictions with online adaptation
\cite{Zhang2012,packandcap,Winter2010,dubach2010,Koala,Cinder,
  wu2012inferred}.  For example, Dubach et al.  use a hybrid scheme to
optimize the microarchitecture of a single core \cite{dubach2010}.
Such predictors have also been employed at the operating system level
to manage system energy consumption
\cite{Koala,Cinder,wu2012inferred}.  Other approaches combine offline
prediction with online updates \cite{JouleGuard,Bitirgen2008,Ipek}.
For example, Bitirgen et al use an artificial neural network to
allocate resources to multiple applications in a multicore
\cite{Bitirgen2008}.  The neural network is trained offline and then
adapted online using measured feedback to maximizes performance but
without consideration for energy minimization.

\paragraph{Control}

Almost all control solutions can be thought of as a combination of
offline prediction with online adaptation.  The offline phase involves
substantial empirical measurement that is used to synthesize a control
system
\cite{Wu2004,Chen2011,POET,ControlWare,Agilos,Rajkumar,Sojka,Raghavendra2008}.
The combination of offline learning and control works well over a
narrow range of applications, as the offline data captures the general
behavior of a class of application and require negligible online
overhead.  This focused approach is extremely effective for multimedia
applications \cite{grace2,flinn99,flinn2004,xtune,TCST} and
web-servers \cite{Horvarth,LuEtAl-2006a,SunDaiPan-2008a} because the
workloads can be characterized ahead of time to produce sound control.

Indeed, the need for good predictions is the central tension in
developing control for computing systems.  It is always possible to
build a controller for a specific application and system by
specializing for that pair.  More general controllers, which work with
a range of applications, have addressed the need for accurate
predictions in various ways.  Some provide libraries that encapsulate
control functionality and require user-specified models
\cite{ControlWare,Sojka,Rajkumar,POET,SWiFT}.  Others automatically
synthesize both a predictor and a controller for either hardware
\cite{josep-isca2016} or software \cite{ICSE2014,FSE2015}.  JouleGuard
combines learning for energy efficiency with control for managing
application parameters \cite{JouleGuard}.  In JouleGuard, a learner
adapts the controller's coefficients to model uncertainty, but
JouleGuard's learner does not produce a new set of predictions.
Because JouleGuard's learner runs on the same device as the controlled
application, it must be computationally efficient and thus it cannot
identify correlations across applications or even different resource
configurations.  \SYSTEM{} is unique in that a separate learner
generates an application-specific predictions automatically.  By
offloading the learning task, \SYSTEM{} (1) combines data from many
applications and systems and (2) applies computationally expensive,
but highly accurate learning techniques.


