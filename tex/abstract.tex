\begin{abstract}

  Intelligent resource management is essential for modern computing
  systems that must provide reliable performance with minimal energy.
  Two central challenges arise when allocating system resources to
  meet these conflicting goals: (1) \emph{complexity}---modern
  hardware exposes diverse resources with complicated
  interactions---and (2) \emph{dynamics}---performance must be
  maintained despite unpredictable changes in operating environment or
  input.  Machine learning accurately predicts the performance of
  complex, interacting resources, but does not address system
  dynamics; control theory adjusts resource usage dynamically, but
  struggles with complex resource interaction. We therefore propose
  \SYSTEM{}, a combination of learning and control that automatically
  adjusts resource usage to meet performance requirements with minimal
  energy in complex, dynamic environments.  \SYSTEM{} breaks resource
  allocation into two sub-tasks: learning speedup as a function of
  resource usage, and controlling speedup to meet performance
  requirements. \SYSTEM{} also defines a general interface allowing
  different learners to be combined with a controller while
  maintaining control's formal guarantees that performance will
  converge to the goal. We implement \SYSTEM{} and test its ability to
  deliver reliable performance on heterogeneous ARM big.LITTLE
  architectures in both single and multi-application scenarios.
  Compared to state-of-the-art learning and control solutions, we find
  that \SYSTEM{} reduces deadline misses by 2--6$\times$ while
  reducing energy consumption by 7--10$\%$.
  



  \PUNT{ 

    includes (1) a control abstraction where key parameters are
    automatically estimated by a noisy learning mechanism, (2) an
    efficient implementation that applies the learned models in
    constant time, and (3) formal guarantees that the system converges
    to the desired performance without prior knowledge of application
    behavior

Mobile systems must deliver performance to interactive
    applications while simultaneously conserving resources to extend
    battery life.  There are two central challenges to meeting these
    conflicting goals: (1) the complicated optimization spaces arising
    from hardware heterogeneity and (2) dynamic changes in application
    behavior and resource availability.  Machine learning techniques
    handle complicated optimization spaces, but do not incorporate
    models of system dynamics; control theory provides formal
    guarantees of dynamic behavior, but struggles with non-linear
    system models.  In this paper, we propose \SYSTEM{}, a combination
    of learning and control techniques to meet performance
    requirements on heterogeneous devices in unpredictable
    environments.  \SYSTEM{} combines a hierarchical Bayesian model
    (HBM) with a lightweight control system (LCS).  The HBM runs
    remotely, learning customized performance/power models.  The LCS
    runs on the mobile system and tunes resource usage to meet
    performance goals.  The Performance Hash Table (PHT) is the
    interface between the two and allows the LCS to apply the learned
    models in constant time.  We test \SYSTEM{}'s ability to manage
    ARM big.LITTLE systems.  Compared to existing learning and control
    methods, \SYSTEM{} delivers more reliable performance -- only 2\%
    error compared to 4.5-5.4\% for learning and 4.7\% for control --
    and lower energy -- within 7\% of optimal on average as compared
    to 25-52\% for learning and 26\% for control. Furthermore, we
    demonstrate \SYSTEM{}'s ability to meet performance and energy
    goals in dynamic systems with phase changes and multiple
    applications running on the same system.  }



  \PUNT{ When multiple applications compete for resources, these
    numbers improve: 7\% error compared to 11-15\% for learning and
    9\% for control and improvements of 2-20\% and 3\%, respectively,
    in energy efficiency.}
\end{abstract}
