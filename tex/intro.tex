\section{Introduction}
% Dennard Scaling making energy essential.  Architects address energy
% by making more complicated processors which expose resources to
% software management.  For a wide range of applications, need to meet
% performance goals with minimal energy.
Large classes of computing systems---from embedded to cloud---must
deliver reliable performance to users while minimizing energy to
increase battery life or decrease operating costs.  To address these
conflicting requirements, hardware architects expose diverse,
heterogeneous resources with a wide array of performance and energy
tradeoffs.  Software must allocate these resources to
guarantee performance requirements are met with minimal energy.


% Difficulties of meeting performance with minimal energy. (1)
% complexity---heterogeneous resources---and (2) dynamics---adjust to
% unforeseen changes in workload and environment.
There are two primary difficulties in determining how to allocate
heterogeneous resources.  The first is \emph{complexity}: resources
interact in complicated ways, leading to non-convex optimization
spaces.  The second is \emph{dynamics}: perfor\-mance requirements
must be met despite unpredictable disturbances; \eg{} phases in input
or changes in operating environment.  Prior work addresses each of
these difficulties individually.

% Prior approaches addressed each of these difficulties individually.
% ML---can handle complexity.  ML advantages: can handle
% non-convexity, avoid local optima, get to true optimal solution. ML
% disadvantages: advanced techniques are expensive and no notion of
% dynamics.  Control---handles dynamics.  Control advantages: formally
% analyzable guarantees despite dynamics.  Control disadvantages:
% relies on good models---no local optima, bounded error.
Many machine learning approaches model the complex performance/power
tradeoff spaces inherent to heterogeneous computing
\cite{reddiHPCA2013,dubach2010,Bitirgen2008,Ipek,Koala,LEO,Flicker,Ponamarev,Paragon}.
Learning handles non-convexity, identifying local optima and moving to
globally optimal solutions. These techniques are computationally
expensive and lack support for dynamics; \ie{} when the environment
changes the expensive model building process must be restarted.
Control theoretic solutions provide formal guarantees that they will
converge to the desired performance despite system dynamics
\cite{Hellerstein2004a,Chen2011,POET,ControlWare,Agilos,grace2,JouleGuard}.
Control provides formal guarantees of convergence to the desired
performance, but these guarantees require \emph{ground truth} models
and control will not converge if the models do not accurately capture
system complexity---including local optima and non-linearities.


% Want to combine learning and control to address both difficulties
% simultaneously.  Need an interface that allows learned models to be
% used by control system.  Challenges: (1) overhead and (2) formal
% guarantees.  
We want the benefits of both learning and control to ensure
performance requirements are met with minimal energy in complex and
dynamic environments.  Thus, we present
\SYSTEM{}:\footnote{\textbf{C}ontrol \textbf{A}nd \textbf{L}earning
  for \textbf{O}ptimal \textbf{R}esource \textbf{E}nergy
  \textbf{E}fficiency} a \emph{parameter-free} framework for combining
machine learning and control theory to build resource allocators.
\SYSTEM{} is parameter-free because it automatically tunes all
internal parameters and models to customize itself to the application
under management.  In contrast, many approaches require user-specified
parameters, like learning rate \cite{dubach2010,Tokic2010} or poles
and zeros of characteristic equations \cite{ControlWare,POET}.

\SYSTEM{} creates a general control system implementation that factors
out the parameters that must be learned.  While traditional control
designs assume all combinations of resource configurations have been
measured accurately (requiring 100s or 1000s of samples
\cite{sysid,FSE2015}), \SYSTEM{} tunes its internal models and
parameters while sampling only a small subset of possible
configurations.  Among these, the controller's \emph{pole} is a key
parameter.  \SYSTEM{} automatically tunes the pole, providing formal
guarantees the controller converges to the desired performance---even
when behavior is estimated by a noisy learner rather than directly
measured.  \SYSTEM{} implements modular abstractions and self-tuning
mechanisms such that the expensive learning can be offloaded to
another system and the controller itself runs in
constant---$O(1)$---time.  Thus, learning's high cost can be amortized
as \SYSTEM{} uses \emph{transfer learning} to apply knowledge from
multiple devices and applications \cite{pan2010survey}.


\SYSTEM{}'s generality allows a wide range of learning techniques to
be paired with its controller.  So, \SYSTEM{} not only provides an
advantage over existing individual learning and control techniques, it
allows us to explore different combinations of learning and control to
find the best.

% Implement CALOREE.  Test against state of the art learning and
% self-tuning control systems.  We find that:
We evaluate \SYSTEM{} by implementing the learner on an x86 server and
the controller on heterogeneous ARM big.LITTLE devices.  We compare
\SYSTEM{} to existing, state-of-the-art learning (including polynomial
regression \cite{Koala,dubach2010}, the Netflix algorithm
\cite{netflix,Paragon}, and a hierarchical Bayesian model \cite{LEO})
and control (including proportional-integral-derivative
\cite{Hellerstein2004a} and adaptive, or self-tuning
\cite{HandbookControl}) techniques.  \PUNT{Additionally, we compare to
  a naive combination of learning and control that does not account
  for the inaccuracies of learned models.}  We set performance
goals---as latency requirements---for a set of benchmark applications
and then measure both the percentage of time the requirements are
violated and the energy for each application.  We test both
\emph{single-app}, where an application runs alone, and
\emph{multi-app} environments, where other applications enter the
system and compete for resources.  \SYSTEM{} achieves the:
\begin{itemize}[leftmargin=1em]
\item \textit{Most reliable performance:}
  \begin{itemize}[leftmargin=1em]
  \item In the \emph{single-app} case, the best prior technique misses
    12\% of deadlines on average, while \SYSTEM{} misses only 5\% on
    average---reducing deadline misses by more than 58\% compared to
    prior approaches.
  \item In the \emph{multi-app} case, the best prior approach averages
    30\% deadline misses, but \SYSTEM{} misses just 5.6\% of
    deadlines---a huge improvement over prior work.
  \end{itemize}
\item \textit{Best energy savings:} We compare to an \emph{oracle}
  with a perfect model of the application, system, and future events.
  \begin{itemize}[leftmargin=1em]
  \item In the \emph{single-app} case, the best prior approach
    averages 12\% more energy consumption than the oracle, but
    \SYSTEM{} consumes only 5\% more.
  \item In the \emph{multi-app} case, the best prior approach averages
    18\% more energy than the oracle, while \SYSTEM{} consumes just
    8\% more.
  \end{itemize}
\end{itemize}

% Key contributions.
% Contributions, but I decided against bulleted llist for thsi paper
In summary, control approaches are well suited to dynamic environments
and learning techniques accurately model complex, heterogeneous
processors.  \emph{To the best of our knowledge, \SYSTEM{} is the
  first work to combine the two to ensure application performance
  without prior knowledge of the controlled application.}  Our formal
analysis of \SYSTEM{}'s convergence despite noisy inputs shows how to
incorporate learned variance into control theoretic guarantees.  We
demonstrate the practical benefits of these contributions by
implementing \SYSTEM{} for mobile/embedded processors to show it
provides more reliable performance and lower energy than individual,
state-of-the-art learning and control solutions.


