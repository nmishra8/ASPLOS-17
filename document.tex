%\documentclass[10pt,twocolumn]{article}
%\documentclass[pageno]{jpaper}
%\documentclass{sig-alternate}
\documentclass[sigplan,10pt,review,anonymous]{acmart}\settopmatter{printfolios=true}
\usepackage{mathptmx} % This is Times font

%\newcommand{\asplossubmissionnumber}{209}
%\newcommand{\iscasubmissionnumber}{457}
\setcopyright{none}             %% For review submission

%\usepackage{times}
\usepackage{fullpage}
\usepackage[utf8]{inputenc}
\usepackage[english]{babel}
\usepackage{amstext,amssymb,amsmath,amsthm}
\usepackage{verbatim}
\usepackage{subfigure}
\usepackage{color}
\usepackage{paralist}
\usepackage{multirow}
\usepackage{listings}
\usepackage{array}
\usepackage{graphicx,color}
\usepackage{moreverb}
\usepackage{xspace}
\usepackage{algorithmic}
\usepackage{algorithm}
\usepackage{times}
\usepackage{relsize}
\newcolumntype{x}[1]{>{\centering\arraybackslash}p{#1}}
\newlength\SUBSIZE
\newcommand{\ONECOLFIGMUL}{0.81}
\newcommand{\TALLFIGMUL}{0.75}
\newcommand{\ie}{\textit{i.e.},~}
\newcommand{\eg}{\textit{e.g.},~}
\usepackage{caption}
%\usepackage{subcaption}

\newcounter{cnt}
% \def\full{0}
\newtheorem{lem}[cnt]{Lemma}
\newtheorem{thm}[cnt]{Theorem}
\newtheorem{problem}[cnt]{Problem}
\newtheorem{defn}[cnt]{Definition}
\newtheorem{fact}[cnt]{Fact}
\newtheorem{example}[cnt]{Example}
\newtheorem{corollary}[cnt]{Corollary}
\newtheorem{assumption}[cnt]{Assumption}
\newtheorem{claim}[cnt]{Claim}
\newcommand{\ip}[2]{\langle #1,#2\rangle}
\renewcommand{\algorithmicrequire}{\textbf{Input:}}
\renewcommand{\algorithmicensure}{\textbf{Output:}}
\newcommand{\nptheta}{\hat\theta}
\newcommand{\symdiff}{\Delta}
\newcommand{\noise}{{\sf Noise}}
\newcommand{\privtheta}{\theta_{priv}}
\renewcommand{\paragraph}[1]{\vspace{3pt}\noindent\textbf{#1}}
\newcommand{\scrX}{\ensuremath{\mathcal{X}}}
\newcommand{\scrZ}{\ensuremath{\mathcal{Z}}}
\newcommand{\rA}{\ensuremath{\rightarrow}}
\newcommand{\rrA}{\ensuremath{\longrightarrow}}
\newcommand{\qB}{\ensuremath{\mathbf{q}}}
\newcommand{\XB}{\ensuremath{\mathbf{X}}}
\newcommand{\zeroB}{\ensuremath{\mathbf{0}}}
\newcommand{\sm}{\mbox{\textendash}}
\newcommand{\ltwo}[1]{\left\|#1\right\|_2}
\newcommand{\lone}[1]{\left\|#1\right\|_1}
\newcommand{\eps}{\epsilon}
\newcommand{\A}{\mathcal{A}}
\newcommand{\D}{\mathcal{D}}
\newcommand{\T}{\mathcal{T}}
\newcommand{\R}{\mathbb{R}}
\newcommand{\K}{\mathcal{K}}
\newcommand{\mineig}{\eta}
\newcommand{\I}{\mathbb{I}}
\newcommand{\E}{\mathbb{E}}
\newcommand{\F}{\mathcal{C}}
\newcommand{\hatw}{\hat{w}}
\newcommand{\hatv}{{\hat{v}}}
\newcommand{\hatV}{{\hat{V}}}
\newcommand{\hatW}{{\hat{W}}}
\newcommand{\kl}{{\sf KL}}
\newcommand{\dagw}{w^\dagger}
\newcommand{\tildew}{\tilde{w}}
\newcommand{\tildeF}{\tilde{F}}
\newcommand{\tildef}{\tilde{f}}
\newcommand{\re}{\mathbb{R}}
\newcommand{\B}{\mathbb{B}}
\newcommand{\bP}{\mathbb{P}}
\renewcommand{\P}{\mathcal{P}}
\newcommand{\grad}{\bigtriangledown}
\newcommand{\mypar}[1]{\smallskip
\noindent{\bf\em {#1}.}}
\newcommand{\etal}{\emph{et al.}}
\newcommand{\negl}{\text{negl}}
\newcommand{\mynote}[3]{\marginpar{\parbox{0.7in}{\tiny {\color{#2} {\sc #1}: {\sf #3}}}}}
\newcommand{\atnote}[1]{\mynote{at}{blue}{#1}}
\newcommand{\nmnote}[1]{\mynote{prmohan}{red}{#1}}
\newcommand{\ignore}[1]{}
\newcommand{\rpm}{\sbox0{$1$}\sbox2{$\scriptstyle\pm$}  \raise\dimexpr(\ht0-\ht2)/2\relax\box2 }
\newcommand{\TODO}[1]{{\bf TODO: #1}}
\newcommand{\NOTE}[1]{{\bf NOTE: #1}}

\newcommand{\name}{\textsc{GUPT}\xspace}
\newcommand{\nameplain}{GUPT\xspace}
\newcommand{\aname}{a \textsc{GUPT}\xspace}
\newcommand{\Aname}{A \textsc{GUPT}\xspace}
\newcommand{\C}{\mathcal{C}}
\newcommand{\hatd}{{\hat d}}
\newcommand{\hatf}{{\hat f}}
\newcommand{\tr}{{\sf tr}}
\newcommand{\Lap}{\mathsf{Lap}}
\newcommand{\M}{\mathcal{M}}
\newcommand{\besta}{a_{{\sf best}}}
\newcommand{\Vect}{\text{vec}}
\newcommand{\Vech}{\text{vech}}
\newcommand{\rank}{\text{rank}}
\newcommand{\diag}{\text{diag}}
\newcommand{\nnz}{\text{nnz}}
\newcommand{\y}{\mathbf{y}}
\newcommand{\w}{\mathbf{w}}
\newcommand{\x}{\mathbf{x}}
\newcommand{\z}{\mathbf{z}}
\DeclareMathOperator*{\argmin}{arg\,min}
\DeclareMathOperator*{\argmax}{arg\,max}

\usepackage{balance}
%\usepackage{times}
%\usepackage{fullpage}
\usepackage{enumitem}
%\usepackage[shortlabels]{enumitem}
%\usepackage{mathtools}
\usepackage[normalem]{ulem}
\usepackage{todonotes}
\usepackage{subfig}
\usepackage{mathtools}
\usepackage{algorithm2e}
\usepackage{algorithmic}
\usepackage{wrapfig}
%\usepackage{algorithmicx}
%\usepackage{algpseudocode}

%\newtheorem{theorem}{Theorem}
%\usepackage{amstext,amssymb,amsmath,amsthm}
%\DeclarePairedDelimiter{\ceil}{\lceil}{\rceil}
%\DeclarePairedDelimiter\floor{\lfloor}{\rfloor}
\let\bibhang\relax
\let\citename\relax
\let\bibfont\relax
\let\Citeauthor\relax
\let\citet\relax
\let\citep\relax
\let\citealt\relax
\let\citealp\relax
\let\Citet\relax
\let\Citep\relax
\let\Citealt\relax
\let\Citealp\relax
\let\citefullauthor\relax
\let\citetext\relax
\let\defcitealias\relax
\expandafter\let\csname ver@natbib.sty\endcsname\relax
\usepackage[natbib=true,backend=bibtex,firstinits=true,style=numeric-comp,sorting=nyt,defernumbers,maxnames=99,maxcitenames=99]{biblatex}

\setlist{noitemsep,topsep=0pt}

%\setlength{\itemsep}{0pt}
%\setlength{\textheight}{9.0in}
%\setlength{\columnsep}{0.25in}
%\setlength{\textwidth}{6.50in}
%\setlength{\topmargin}{0.0in}
%\setlength{\headheight}{0.0in}
%\setlength{\headsep}{0.0in}
\setlength{\leftmargini}{0em}%
\setlength{\leftmarginii}{1em}%

\setlength{\abovecaptionskip}{1pt plus 2pt minus 2pt}
\setlength{\floatsep}{3pt}
\setlength{\intextsep}{4pt}


%--------
\definecolor{mygreen}{rgb}{0,0.6,0}
\definecolor{mygray}{rgb}{0.5,0.5,0.5}
\definecolor{mymauve}{rgb}{0.58,0,0.82}

\definecolor{on}{RGB}{159, 169,133} 
\definecolor{off}{RGB}{255,165,0}%{240,232,205} 
\definecolor{leo}{RGB}{72,181,163} 
\definecolor{poet}{RGB}{249,140,182} 
\definecolor{calnp}{RGB}{117,137,191}
\definecolor{cal}{RGB}{51,255,51} %green


\definecolor{OPTIMAL}{RGB}{0,0,0}
\definecolor{RACE}{RGB}{165,15,21}

\definecolor{CONTROL}{RGB}{222,45,38}
\definecolor{PID-CONTROL}{RGB}{222,45,38}

\definecolor{ONLINE}{RGB}{251,106,74}

\definecolor{OFFLINE}{RGB}{255,165,0}

\definecolor{NUCLEAR}{RGB}{249,140,182}
\definecolor{NETFLIX}{RGB}{249,140,182}

\definecolor{HBM}{RGB}{159, 169,133}

\definecolor{ADAPT-CONTROL}{RGB}{51,255,51}
\definecolor{POET}{RGB}{51,255,51}

\definecolor{CALOREE-NP}{RGB}{72,181,163}

\definecolor{ONLINE-ADAPT}{RGB}{140,150,198}
\definecolor{NUCLEAR-ADAPT}{RGB}{136,86,167}
\definecolor{HBM-ADAPT}{RGB}{129,15,124}

\definecolor{CALOREE-ONLINE}{RGB}{140,150,198}
\definecolor{CALOREE-NETFLIX}{RGB}{136,86,167}
%\definecolor{CALOREE-HBM}{RGB}{129,15,124}
\definecolor{CALOREE-HBM}{RGB}{102,0,102}



\usepackage{listings}
\lstset{ %
  backgroundcolor=\color{white},   % choose the background color; you must add \usepackage{color} or \usepackage{xcolor}
  basicstyle=\scriptsize\ttfamily, % the size of the fonts that are used for the code
  breakatwhitespace=false,         % sets if automatic breaks should only happen at whitespace
  breaklines=true,                 % sets automatic line breaking
  captionpos=b,                    % sets the caption-position to bottom
  commentstyle=\color{mygreen},    % comment style
  deletekeywords={...},            % if you want to delete keywords from the given language
  escapeinside={\%*}{*},          % if you want to add LaTeX within your code
  extendedchars=true,              % lets you use non-ASCII characters; for 8-bits encodings only, does not work with UTF-8
  frame=leftline,                  % adds a frame around the code
  keepspaces=true,                 % keeps spaces in text, useful for keeping indentation of code (possibly needs columns=flexible)
  keywordstyle=\color{blue},       % keyword style
  morekeywords={*,...},            % if you want to add more keywords to the set
  numbers=left,                    % where to put the line-numbers; possible values are (none, left, right)
  numbersep=5pt,                   % how far the line-numbers are from the code
  numberstyle=\tiny\color{mygray}, % the style that is used for the line-numbers
  rulecolor=\color{black},         % if not set, the frame-color may be changed on line-breaks within not-black text (e.g. comments (green here))
  showspaces=false,                % show spaces everywhere adding particular underscores; it overrides 'showstringspaces'
  showstringspaces=false,          % underline spaces within strings only
  showtabs=false,                  % show tabs within strings adding particular underscores
  stepnumber=1,                    % the step between two line-numbers. If it's 1, each line will be numbered
  stringstyle=\color{black},     % string literal style
  tabsize=1,                       % sets default tabsize to 2 spaces
  title=\lstname                   % show the filename of files included with \lstinputlisting; also try caption instead of title
}

\usepackage{pgfplots, pgfplotstable}
% options for pgfplots
\pgfplotscreateplotcyclelist{colorbrewer-ylgnbu}{
{fill=on, draw=black},
{fill=off, draw=black},
{fill=leo, draw=black},
{fill=poet, draw=black},
{fill=calnp, draw=black},
{fill=cal, draw=black},
}
\pgfplotsset{compat=1.8,compat/show suggested version=false}
\usetikzlibrary{calc,trees,arrows,patterns,plotmarks,shapes,snakes,er,3d,automata,backgrounds,topaths,decorations.pathmorphing,decorations.markings}
%\pgfplotsset{compat=newest}
\pgfplotsset{
   /pgfplots/bar  cycle  list/.style={/pgfplots/cycle  list={%
        {black,fill=black!30!white,mark=none},%
        {black,fill=red!30!white,mark=none},%
        {black,fill=green!30!white,mark=none},%
        {black,fill=yellow!30!white,mark=none},%
        {black,fill=brown!30!white,mark=none},%
     }
   },
}
% begin of externalization
\usetikzlibrary{external}
\tikzexternalize[prefix=out/]
\tikzexternalize
\usetikzlibrary{patterns}
\usepgfplotslibrary{groupplots}
\pgfplotsset{
every axis label/.append style={font=\footnotesize},
tick label style={font=\footnotesize},
}

\makeatletter
\g@addto@macro\normalsize{%
  \setlength\abovedisplayskip{4pt plus 2pt minus 1pt}
  \setlength\belowdisplayskip{4pt plus 2pt minus 1pt}
  \setlength\abovedisplayshortskip{4pt plus 2pt minus 1pt}
  \setlength\belowdisplayshortskip{4pt plus 2pt minus 1pt}
}
\makeatother

\newif{\ifanonymous}
\anonymoustrue

\newcommand{\cutout}[1]{}
\newcommand{\smallcaption}[1]{\caption[#1]{{\protect\small \protect\bf #1}}}
\newcommand{\dids}{{\sc dids}}
\newcommand{\us}{\,$\mu$s}
\newcommand{\ms}{\,ms}
\newcommand{\KB}{\,KB}
\newcommand{\MB}{\,MB}
\newcommand{\GB}{\,GB}
\newcommand{\MHz}{\,MHz}
\newcommand{\GHz}{\,GHz}
\newcommand{\eg}{\emph{e.g.},}
\newcommand{\ie}{\emph{i.e.},}
\newcommand{\y}{\mathbf{y}}

\newcommand{\SYSTEM}{CALO\-REE}
%\newcommand{\TODO}[1]{\textbf{TODO: 1}}
\definecolor{gray}{gray}{0.75}
\newcommand{\TODO}[1]{\textcolor{gray}{\textbf{\ [TODO:\ #1]\ }}}
\newcommand{\secref}[1]{Section~\ref{sec:#1}}
\newcommand{\Secref}[1]{Section~\ref{#1}}
\newcommand{\PUNT}[1]{}
\newcommand{\figref}[2][{}]{{Figure~\ref{#2}#1}}
\newcommand{\figsref}[2]{Figures~\ref{#1} and~\ref{#2}}
\newcommand{\tblref}[1]{Table~\ref{tbl:#1}}
\newcommand{\algref}[1]{Algorithm~\ref{alg:#1}}
\renewcommand{\textfraction}{0.05}
%\newcommand{\figref}[1]{Figure~\ref{fig:#1}}
%\newcommand{\figsref}[2]{Figures~\ref{fig:#1} and~\ref{fig:#2}}
\newcommand{\figrref}[2]{Figures~\ref{#1}--\ref{#2}}
%\newcommand{\secref}[1]{Section~\ref{sec:#1}}
\newcommand{\secsref}[2]{Sections~\ref{sec:#1} and~\ref{sec:#2}}
\newcommand{\eqnref}[1]{Eqn.~\ref{eqn:#1}}
\newcommand{\eqnsref}[2]{Eqns.~\ref{eqn:#1} and~\ref{eqn:#2}}
\newcommand{\eqnrref}[2]{Eqns.~\ref{eqn:#1}--\ref{eqn:#2}}
%\newcommand{\insref}[1]{Instruction~\ref{ins:#1}}
%\newcommand{\tblref}[1]{Table~\ref{tbl:#1}}
\newcommand{\appref}[1]{Appendix~\ref{app:#1}}
%\newcommand{\y}{\mathbf{y}}
\newcommand{\w}{\mathbf{w}}
\newcommand{\x}{\mathbf{x}}
\newcommand{\z}{\mathbf{z}}
\newcommand{\R}{\mathbb{R}}
\newcommand{\I}{\mathbb{I}}
\DeclarePairedDelimiter\Floor\lfloor\rfloor
\DeclarePairedDelimiter\Ceil\lceil\rceil
%\newcommand{\minimize}{minimize}

\renewcommand{\paragraph}[1]{\vspace{3pt}\noindent\textbf{#1}}
\newcommand{\st}{s.t.}

\DeclareMathOperator*{\argmin}{arg\,min}
\DeclareMathOperator*{\argmax}{arg\,max}
\DeclareMathOperator*{\minimize}{minimize}
\bibliography{reference}

\begin{document}

\title[\SYSTEM{}]{\SYSTEM{}: Combining Control and Learning for Predictable
  Performance and Low Energy} \date{}

%\thispagestyle{firstpage}
%\pagestyle{plain}

\begin{abstract}

  
  Control systems are a proven method for allocating resources to meet
  streaming applications' performance requirements with minimal energy
  on mobile and embedded devices. Control designs, however, require
  \emph{a priori} models to estimate an application's performance and
  energy based on its resource usage -- a controller designed for one
  application and system must be redesigned for a new deployment.  We
  propose CALOREE to achieve control theoretic benefits -- \ie{}
  formal, bounded convergence to required performance in dynamic
  environments -- without \emph{a priori} models.  CALOREE combines a
  generalized control system (GCS) \PUNT{that runs on a local device}
  with a hierarchical Bayesian model (HBM)\PUNT{that runs on a remote
    server}. The HBM runs on a remote server and aggregates data from
  multiple applications and devices to produce a highly accurate model
  that it sends to the GCS to customize control for the local device
  and application. We extend standard control analysis to show that
  CALOREE provides probabilistic convergence guarantees despite having
  no prior model of the controlled application.

  We implement CALOREE using ARM big.LITTLE boards for streaming
  applications and an x86 server for the HBM.  We test in both single-
  and multi-application environments (where applications compete for
  resources).  Compared to state-of-the-art learning and control
  techniques, CALOREE consistently provides the most reliable
  performance: a worst case performance error of only 12\% compared to
  70-80\% for prior approaches.  Additionally, CALOREE provides the
  lowest energy: delivering average savings from 8-47\%.


  \PUNT{CALOREE has three main components: (1) a remote learner that
    aggregates data across devices and applications to model
    application performance and energy, (2) a lightweight control
    system that uses the learned models to customize control for a
    particular application, and (3) the interfaces that map non-convex
    learned models to the continuous linear models used by the
    controllers.

  We implement CALOREE's learning on an x86 server and stream
  processing on heterogeneous ARM big.LITTLE devices. We test in both
  \emph{stable} and \emph{unstable} environments (where available
  resources change over time) and we compare to state-of-the-art
  learning and control-based resource managers.  In all test
  scenarios, \SYSTEM{} provides the lowest error between the required
  and delivered performance. Depending on the scenario, \SYSTEM{}
  provides 12-25\% average energy reduction compared to
  state-of-the-art approaches.  Additionally, \SYSTEM{}'s worst case
  performance and energy is uniformly better than prior approaches, by
  factors as high as $2\times$.
}


\PUNT{
  Streaming sensor processing forms a foundational workload for
  embedded, mobile, and Internet-of-things. Stream processing on these
  platforms requires both reliable performance and low energy, and two
  challenges must be addressed to meet these conflicting requirements.
  The first is \emph{complexity}: hardware exposes heterogeneous
  resources which interact in complicated ways. The second is
  \emph{dynamics}: performance must keep up with the data stream
  despite unpredictable changes in operating environment.  Prior work
  shows that machine learning addresses the complexity challenge and
  control systems tackle dynamics, but streaming sensor processing on
  energy-limited devices requires that both challenges be met
  simultaneously.

  To address both complexity and dynamics, we propose \SYSTEM{}, a
  resource management system that integrates hierarchical Bayesian
  learning with a lightweight control system.  The learning framework
  runs on a remote server aggregating data from multiple applications
  and devices to estimate performance/energy models for different
  resource configurations.  The controller runs on energy-limited
  devices using the learned models to deliver required performance for
  streaming processing with minimal energy.  We test \SYSTEM{} by
  implementing its learning on an x86 server and stream processing on
  four heterogeneous ARM big.LITTLE devices. We test in both
  \emph{stable} and \emph{unstable} environments (where available
  resources change over time) and we compare to state-of-the-art
  learning and control approaches.  In all test scenarios, \SYSTEM{}
  provides the lowest error between the required and delivered
  performance. Depending on the scenario, \SYSTEM{} also provides
  12-25\% average energy reduction compared to state-of-the-art
  learning and control approaches.  Additionally, \SYSTEM{}'s worst
  case performance and energy is uniformly better than prior
  approaches.  Thus, \SYSTEM{}'s unique combination of learning and
  control is well-suited to supporting streaming sensor processing in
  embedded, mobile, and IoT platforms.
}


  \PUNT{ Mobile systems must deliver performance to interactive
    applications while simultaneously conserving resources to extend
    battery life.  There are two central challenges to meeting these
    conflicting goals: (1) the complicated optimization spaces arising
    from hardware heterogeneity and (2) dynamic changes in application
    behavior and resource availability.  Machine learning techniques
    handle complicated optimization spaces, but do not incorporate
    models of system dynamics; control theory provides formal
    guarantees of dynamic behavior, but struggles with non-linear
    system models.  In this paper, we propose \SYSTEM{}, a combination
    of learning and control techniques to meet performance
    requirements on heterogeneous devices in unpredictable
    environments.  \SYSTEM{} combines a hierarchical Bayesian model
    (HBM) with a lightweight control system (LCS).  The HBM runs
    remotely, learning customized performance/power models.  The LCS
    runs on the mobile system and tunes resource usage to meet
    performance goals.  The Performance Hash Table (PHT) is the
    interface between the two and allows the LCS to apply the learned
    models in constant time.  We test \SYSTEM{}'s ability to manage
    ARM big.LITTLE systems.  Compared to existing learning and control
    methods, \SYSTEM{} delivers more reliable performance -- only 2\%
    error compared to 4.5-5.4\% for learning and 4.7\% for control --
    and lower energy -- within 7\% of optimal on average as compared
    to 25-52\% for learning and 26\% for control. Furthermore, we
    demonstrate \SYSTEM{}'s ability to meet performance and energy
    goals in dynamic systems with phase changes and multiple
    applications running on the same system.  }



  \PUNT{ When multiple applications compete for resources, these
    numbers improve: 7\% error compared to 11-15\% for learning and
    9\% for control and improvements of 2-20\% and 3\%, respectively,
    in energy efficiency.}
\end{abstract}

\maketitle
\section{Introduction}
% Dennard Scaling making energy essential.  Architects address energy
% by making more complicated processors which expose resources to
% software management.  For a wide range of applications, need to meet
% performance goals with minimal energy.
Large classes of computing systems---from embedded to cloud---must
deliver reliable performance to users while minimizing energy to
increase battery life or decrease operating costs.  To address these
conflicting requirements, hardware architects have begun to expose
increasingly diverse, heterogeneous resources with an array of
different performance and energy tradeoffs.  It is then software's
responsibility to allocate these resources such that performance
requirements are met with minimal energy.


% Difficulties of meeting performance with minimal energy. (1)
% complexity---heterogeneous resources---and (2) dynamics---adjust to
% unforeseen changes in workload and environment.
There are two primary difficulties in determining how to allocate
heterogeneous resources.  The first is \emph{complexity}---these
resources interact in complicated ways, leading to non-convex
optimization spaces.  The second is \emph{dynamics}---perfor\-mance
requirements must be met despite unpredictable disturbances; \eg{}
phases in input or changes in operating environment.  Prior work
addresses each of these difficulties individually.

% Prior approaches addressed each of these difficulties individually.
% ML---can handle complexity.  ML advantages: can handle
% non-convexity, avoid local optima, get to true optimal solution. ML
% disadvantages: advanced techniques are expensive and no notion of
% dynamics.  Control---handles dynamics.  Control advantages: formally
% analyzable guarantees despite dynamics.  Control disadvantages:
% relies on good models---no local optima, bounded error.
Many machine learning approaches accurately model the complex
performance/power tradeoff spaces inherent to heterogeneous computing
systems
\cite{reddiHPCA2013,dubach2010,Bitirgen2008,Ipek,Koala,LEO,Flicker,Ponamarev}.
Such ML approaches handle non-convexity, identifying local optima to
find globally optimal solutions; however, these techniques are
computationally expensive and lack support for dynamics; \ie{} when
the environment changes the expensive model building process must be
restarted.  Control theoretic solutions perform efficient resource
allocation in the presence of system dynamics
\cite{Hellerstein2004a,Chen2011,PTRADE,POET,ControlWare,Agilos,grace2}.
Control provides formally analyzable guarantees that the system will
deliver the required performance, but these guarantees require
accurate models and do not support the non-convexity arising from
complicated interactions between heterogeneous resources.


% Want to combine learning and control to address both difficulties
% simultaneously.  Need an interface that allows learned models to be
% used by control system.  Challenges: (1) overhead and (2) formal
% guarantees.  
Our goal is to combine learning and control to ensure performance
requirements are met with minimal energy in complex and dynamic
environments.  The challenges to combining these techniques are (1)
mitigating learning's overhead and (2) preserving the controller's
formal guarantees in the presence of the learned models.

% Combine learning and control through CALOREE.  Describe it.
We address these challenges with \SYSTEM{}, \footnote{\textbf{C}ontrol
  \textbf{A}nd \textbf{L}earning for \textbf{O}ptimal
  \textbf{R}esource \textbf{E}nergy \textbf{E}fficiency} a framework
for combining machine learning and control theory to build resource
management frameworks that can meet application performance
requirements with minimal energy.  The two key components of this
interface are (1) a data structure (called the performance hash table)
that allows the controller to access the learned model in constant
($O(1)$) time and (2) a confidence interval and estimated standard
deviation that provide probabilistic guarantees that the combined
learning and control system will converge to the desired performance.
The \SYSTEM{} interface not only combines control and learning
techniques, but allows the learning and control software to run on
physically separate devices.  This physical separation further
mitigates the cost of expensive learning techniques by running them on
a remote server while the constant time control systems run on the
device to be managed.  In addition to amortizing the cost of learning,
moving it to a remote server allows us to take advantage of learning
techniques that work across devices and applications; \ie{} those that
can learn similarities between different applications and systems.

In fact, the \SYSTEM{} interface is general enough to allow a wide
range of learning techniques to be paired with control systems.  So,
this approach not only provides an advantage over existing individual
learning and control techniques, it allows us to explore different
combinations of learning and control to find the best combination.

% Implement CALOREE.  Test against state of the art learning and
% self-tuning control systems.  We find that:
To demonstrate \SYSTEM{}, we implement it with learning running on an
x86 server and the control systems working to manage heterogeneous ARM
big.LITTLE devices.  We compare \SYSTEM{} to existing,
state-of-the-art learning (including polynomial
regression \cite{}, the Netflix algorithm \cite{}, and a hierarchical
Bayesian model \cite{}) and control (including
proportional-integral-derivative \cite{} and adaptive, or self-tuning
\cite{}) techniques.  Additionally, we compare to a naive combination
of learning and control that does not account for the confidence
interval and standard deviation of the learned model.  We set
performance goals (in terms of latency requirements) for a set of
benchmark applications and then measure the percentage of time the
requirements are violated as well as the energy for each application.
We test both \emph{single-app} environments, where on application runs
alone, and \emph{multi-app} environments where other applications
unpredictably enter the system and compete for resources.  We find
that \SYSTEM{} achieves the:
\begin{itemize} 
\item \textit{Most reliable performance:} 
 \begin{itemize} 
 \item In the \emph{single-app} case, the best prior learning and
   control techniques miss about 12\% of deadlines on average, the
   naive combination of learning and control misses 45\% of deadlines
   on average, but \SYSTEM{} misses only 5\% on average, reducing
   deadline misses by 50\% compared to prior approaches.
 \item In the \emph{multi-app} case, the best prior approach averages
   30\% deadline misses, the naive combination of learning and control
   averages 31\%, but \SYSTEM{} misses just 5.6\% of deadlines, a huge
   reduction compared to prior approaches.
\end{itemize}
  \item \textit{Best energy savings:} We compare to an \emph{oracle}
    with a perfect model of the application, system, and future
    events.
    \begin{itemize}
    \item In the \emph{single-app} case, the best prior approach
      averages 12\% more energy consumption than the oracle, the naive
      combination of control and learning consumes 11\% more, and
      \SYSTEM{} consumes 5\% more.  
    \item In the \emph{multi-app} case, the best prior approach
      averages 18\% more energy than the oracle, the naive combination
      of control and learning consumes 11\% more, and \SYSTEM{}
      consumes 7\% more.
    \end{itemize}
\end{itemize}

% Key contributions.
% Contributions, but I decided against bulleted llist for thsi paper
In summary, control theoretic approaches are well suited to manage
resources in dynamic environments and machine learning techniques can
produce accurate models of complex processors.  \emph{To the best of
  our knowledge, \SYSTEM{} is the first work to propose combining the
  two at runtime to ensure application performance goals without prior
  knowledge of the controlled application.}  We demonstrate this
contribution by implementing a resource manager that can meet
performance requirements on mobile/embedded processors with minimal
energy.  Additional contributions include formal analysis for convergence guarantees of the control system with noisy inputs, thus showing how
to incorporate learned variance into the control theoretic guarantees
and the empirical evaluation showing the combined control and learning
system outperforms individual, state-of-the-art control or learning
solutions.



\section{Background and Motivation}
\label{sec:example}

\PUNT{
We first demonstrate how learning addresses complexity, while even
advanced, adaptive controllers cannot.  We then show how control
theory handles system dynamics.  We conclude this section by
demonstrating what can go wrong when a controller incorporates a
learned model without proper tuning.
}
Many machine learning approaches estimate the most energy efficient
set of resources to allocate to an application.  These include
\emph{offline} techniques that build models using a training set and
then apply those models to new applications
\cite{Yi2003,LeeBrooks2006,CPR,ChenJohn2011,reddiHPCA2013,Paragon,PUPiL}.
Other approaches use \emph{online} techniques that construct models
dynamically as an application runs
\cite{Li2006,Flicker,ParallelismDial,Ponamarev,LeeBrooks}.  Finally,
\emph{hybrid} techniques combine offline modeling with online model
updates \cite{Zhang2012,packandcap,Winter2010,dubach2010,Koala,Cinder,
  wu2012inferred,LEO}.  

Machine learning is well suited to building
models of complicated systems like heterogeneous ARM big.LITTLE
systems.  These processors have two different core types including:
four big high-performance cores and four LITTLE energy efficient
cores.  The big cores support 19 clock speeds, while the LITTLE cores
support 14.

Control theory is a collection of mechanisms for maintaining required
behavior in dynamic systems \cite{Hellerstein2004a}. A subset of these
mechanisms---known as \emph{adaptive control} or \emph{self-tuning
  regulators}---are highly resilient to external effects that alter
behavior.  Adaptive controllers are thus especially useful in
webservers with fluctuating request rates
\cite{Horvarth,LuEtAl-2006a,SunDaiPan-2008a} and multimedia
applications with dynamically varying inputs
\cite{TCST,Agilos,grace2}.  Prior control solutions, however, are
always highly application dependent---with application-specific models
encoded in the controller's design---making a controller for video
playback unsuitable for controlling GPS navigation.  Some prior work
has generalized adaptive control design by exposing key parameters to
users so that the controller can be customized for a user's needs
\cite{ControlWare,POET}.  
This provides greater flexibility to the users but the controller can still fail to converge to the desired performance if there is mis-characterization in the relationship between resources and performance.  \PUNT{This limitation
  is why existing adaptive control approaches are built for specific
  classes of applications---specializing for the class accounts for
  common non-convexities of that class.}

\subsection{\emph{Learning} Complexity}
\PUNT{
\begin{figure*}
\centering
  \subfloat[]
  {
    \includegraphics[width=.25\textwidth]{figures/STREAM-contour.png}
    \label{fig:STREAM_contour}
  }
  \subfloat[]
  {
    \begin{tikzpicture}
\begin{centering}

\definecolor{s1}{RGB}{228, 26, 28}
\definecolor{s2}{RGB}{55, 126, 184}
\definecolor{s3}{RGB}{77, 175, 74}
\definecolor{s4}{RGB}{152, 78, 163}
\definecolor{s5}{RGB}{255, 127, 0}

\begin{groupplot}[
    group style={
        group name=plots,
        group size=1 by 1,
        xlabels at=edge bottom,
        xticklabels at=edge bottom,
        vertical sep=5pt
    },
height=3.5cm,
width=0.45\columnwidth,
xmajorgrids,
ymajorgrids,
grid style={dashed},
xmax=20,
yticklabel pos=left,
enlargelimits=false,
tick align = outside,
tick style={white},
xticklabel shift={-5pt},
yticklabel shift={-5pt},
ylabel shift={-2pt},
ylabel style={align=center},
unbounded coords=jump,
]

\nextgroupplot[ylabel={\scriptsize Latency (Normalized)}, % Performance
xlabel={\footnotesize Iteration},
ymin=0.8,
ymax=1.2,
ytick={0.8,0.9,1.0,1.1,1.2},
yticklabels={0.8,,1.0,,1.2},
legend entries={{\scriptsize $\mathsf{Latency Requirement}$},{\scriptsize $\mathsf{Learning}$},{\scriptsize $\mathsf{Adaptive Control}$},},
legend style={fill=none,draw=none,at={(0.5,1.65)},anchor=north,legend columns=1,line width=3pt},
]

\addplot[thick, solid, black] coordinates {(0,1) (20,1)};
\addplot[thick, solid, color=s4, mark=o, mark size=1pt] table[x index=0,y index=1,col sep=tab] {img/complexity-example-leo-latency.txt};
\addplot[thick, solid, color=s5, mark=square, mark size=1pt] table[x index=0,y index=1,col sep=tab] {img/complexity-example-poet-latency.txt};
\end{groupplot}
\end{centering}
\end{tikzpicture}
    \label{fig:STREAM_timeline}
  }
  \subfloat[]
  {
    \includegraphics[width=.25\textwidth]{figures/BODYTRACK-contour.png}
    \label{fig:BODYTRACK_contour}
  }
  \subfloat[]
  {
    \input{img/BODYTRACK-example-resized.tex}
    \label{fig:BODYTRACK_timeline}    
  }
  \caption{\small \bf (a) Performance for \texttt{STREAM} as a
    function of configuration.  (b) Managing \texttt{STREAM}'s
    performance: \emph{Learning} navigates the complicated
    configuration space, but \emph{control}'s simple model leads to
    oscillation.  (c) Performance for \texttt{bodytrack} as a function
    of configuration. (d) Managing \texttt{bodytrack}'s performance
    when another application starts: \emph{Adaptive Control} detects
    the change and adjusts, but \emph{Learning} has no mechanism to
    handle these dynamics. }
  \label{fig:learning-models1}
\end{figure*}
}

\begin{figure}
\centering
  \subfloat[]
  {
    \includegraphics[width=.25\textwidth]{figures/STREAM-contour.png}
    \label{fig:STREAM_contour}
  }
  \subfloat[]
  {
    \begin{tikzpicture}
\begin{centering}

\definecolor{s1}{RGB}{228, 26, 28}
\definecolor{s2}{RGB}{55, 126, 184}
\definecolor{s3}{RGB}{77, 175, 74}
\definecolor{s4}{RGB}{152, 78, 163}
\definecolor{s5}{RGB}{255, 127, 0}

\begin{groupplot}[
    group style={
        group name=plots,
        group size=1 by 1,
        xlabels at=edge bottom,
        xticklabels at=edge bottom,
        vertical sep=5pt
    },
height=3.5cm,
width=0.45\columnwidth,
xmajorgrids,
ymajorgrids,
grid style={dashed},
xmax=20,
yticklabel pos=left,
enlargelimits=false,
tick align = outside,
tick style={white},
xticklabel shift={-5pt},
yticklabel shift={-5pt},
ylabel shift={-2pt},
ylabel style={align=center},
unbounded coords=jump,
]

\nextgroupplot[ylabel={\scriptsize Latency (Normalized)}, % Performance
xlabel={\footnotesize Iteration},
ymin=0.8,
ymax=1.2,
ytick={0.8,0.9,1.0,1.1,1.2},
yticklabels={0.8,,1.0,,1.2},
legend entries={{\scriptsize $\mathsf{Latency Requirement}$},{\scriptsize $\mathsf{Learning}$},{\scriptsize $\mathsf{Adaptive Control}$},},
legend style={fill=none,draw=none,at={(0.5,1.65)},anchor=north,legend columns=1,line width=3pt},
]

\addplot[thick, solid, black] coordinates {(0,1) (20,1)};
\addplot[thick, solid, color=s4, mark=o, mark size=1pt] table[x index=0,y index=1,col sep=tab] {img/complexity-example-leo-latency.txt};
\addplot[thick, solid, color=s5, mark=square, mark size=1pt] table[x index=0,y index=1,col sep=tab] {img/complexity-example-poet-latency.txt};
\end{groupplot}
\end{centering}
\end{tikzpicture}
    \label{fig:STREAM_timeline}
  }
  \caption{\small \bf (a) Performance for \texttt{STREAM} as a
    function of configuration.  (b) Managing \texttt{STREAM}'s
    performance: \emph{Learning} navigates the complicated
    configuration space, but \emph{control} leads to oscillation.}
  \label{fig:learning-models1}
\end{figure}

We demonstrate how well learning complex performance vs resource relationship for \texttt{STREAM} as illustrated in \figref{fig:STREAM_contour} on our ARM big.LITTLE
processor, can improve resource management for the application.  This memory-bound application has
complicated behavior: due to memory pressure, the LITTLE cores' memory
hierarchy cannot deliver the required performance.  The big cores'
more powerful memory system delivers much greater performance, but the
peak occurs with 3 big cores.  Furthermore, at low clockspeeds, these
3 big cores cannot saturate the memory bandwidth, while at high
clockspeeds the performance drops as the processor overheats and
triggers thermal management.  For \texttt{STREAM}, the peak speed
occurs with 3 big cores at 1.2 GHz, and it is not efficient to spend
any time on the LITTLE cores.  \texttt{STREAM}, however, does not have
distinct phases, so once a resource allocator finds the most energy
efficient configuration, it simply needs to maintain it.  \TODO{We
  should change the levels in the contour so there is a clear maximum
  at 3 cores and a middle clock speed.}

The \emph{learning}
approach estimates the application's performance and power for all
configurations and then uses the lowest power configuration that
delivers the required performance.  The \emph{adaptive controller}
begins with a generic model of power/performance tradeoffs.  As the
controller runs, it continually measures performance and adjusts the
allocated resources and its own parameters in response to
the running application.  While many controllers use linear models,
this adaptive controller dynamically adjusts to non-linearities with a
series of linear approximations; however, the adaptive controller is
sensitive to local maxima, which can cause oscillations and prevent
the controller from meeting the requirement. \figref{fig:STREAM_timeline} shows 20 iterations of \texttt{STREAM}.  The
x-axis shows iteration number and the y-axis shows performance
normalized to the requirement.  The learning approach achieves the
goal, but the adaptive controller oscillates wildly around it,
sometimes not achieving the goal and sometimes delivering performance
that is too high (and wastes energy).  The oscillations occur because
the controller's adaptive mechanisms cannot handle the non-convexity
in STREAM's behavior, a known limitation of adaptive control systems
\cite{ControlWare,POET,ICSE2014}.  Hence, the \emph{learner}'s ability
to handle complex behavior is crucial for reliable performance in this
example.

% This result may be somewhat counter-intuitive.  The problem is that
% the controller cannot handle \texttt{STREAM}'s complexity.  One way to
% address this problem would be to build a custom controller just for
% this application, but that controller would not be useful for other
% applications.  In contrast, the learner can find the local maxima in
% the configuration space, and as this application has no phase changes
% or other dynamics, the one configuration that the learner finds is
% suitable for the entire application.

\subsection{\emph{Controlling} Dynamics}

\PUNT{
\begin{figure*}
\centering
  \subfloat[]
  {
    \includegraphics[width=.25\textwidth]{figs/kmeans.png}
    \label{fig:kmeans_contour}
  }
  \subfloat[]
  {
    \input{img/kmeans-example-resized.tex}
    \label{fig:kmeans_timeline}    
  }
  \caption{\small \bf (a) Performance for \texttt{kmeans} as a function of
    configuration.  (b) Managing \texttt{kmeans}' performance when
    another application starts: \emph{Control} detects the change and
    adapts, but \emph{learning} has no mechanism to handle these
    dynamics.}
  \label{fig:learning-models2}
\end{figure*}
} 


\begin{figure}
\centering
  \subfloat[]
  {
    \includegraphics[width=.25\textwidth]{figures/BODYTRACK-contour.png}
    \label{fig:BODYTRACK_contour}
  }
  \subfloat[]
  {
    \input{img/BODYTRACK-example-resized.tex}
    \label{fig:BODYTRACK_timeline}    
  }
  \caption{\small \bf (a) Performance for \texttt{bodytrack} as a
    function of configuration. (b) Managing \texttt{bodytrack}'s
    performance when another application starts: \emph{Adaptive
      Control} detects the change and adjusts, but \emph{Learning} has
    no mechanism to handle these dynamics. }
  \label{fig:control}
\end{figure}


We now consider a dynamic environment.  We begin with
\texttt{bodytrack} as the only application running on the system.
Halfway through its execution, we launch a second
application---\texttt{STREAM}---on a single big core, dynamically
altering resource availability. \texttt{bodytrack}'s behavior is
shown in \figref{fig:BODYTRACK_contour}; it achieves its best performance on 4
big cores at the maximum clockspeed. Additionally the 4 LITTLE core 
can offer high performance at much lower energy consumption.  
The challenge of allocating resources for this application is determining how to split time between the LITTLE cores and big cores to conserve energy while still meeting the
performance requirements.  

\figref{fig:BODYTRACK_timeline} shows the results of this experiment.  The
vertical dashed line---at frame 99---represents when the second
application begins.  The figure clearly shows the benefits of the
adaptive control system in this dynamic scenario.  When the second
application starts, the controller detects the change in
\texttt{bodytrack}'s performance and then changes resource allocation
(increasing clockspeed and moving bodytrack from 4 to 3 big cores).
The controller is not aware of the second application, rather it
observes \texttt{bodytrack}'s performance drop at frame 99 and
immediately restores performance in the next frame. The learning
system however, does not have any inherent mechanism to measure the
change or adapt to the altered performance.  While we could
theoretically add feedback to the learner and re-estimate the
configuration space whenever the environment changes, doing so is
impractical due to high overhead\cite{pargon,LEO}.


\subsection{Challenges of Parameter-free Control}
\PUNT{
The learner illustrated above is parameter-free; \ie{} it has no
user-specifed parameters, but estimates all values from measurements.}
The controller as we mentioned in earlier section requires some user-specified parameters.
\SYSTEM{}'s goal is to provide a parameter-free approach that provides
the same formal guarantees as a controller with user-specified
parameters.

Perhaps the most essential parameter for a control system is the
\emph{pole} of its characteristic equation.  Control engineers tune
the pole to trade control response time for noise sensitivity.
Following standard control design, the pole is tuned based on a known
model.  This model may be an abstraction of the full system, but it is
considered to be ``ground truth'' \cite{Hellerstein2004a}. In a
computing system ground truth means that all possible configurations
the controller might select have been directly measured.  \SYSTEM{},
however, must tune the pole based on an estimated model, which may
have noise and/or errors.


\begin{figure}
\centering
\input{img/BODYTRACK-example2-resized.tex}
\caption{\small \bf Comparison of existing adaptive control and a
  naive combination of control and learning.}
\label{fig:not-simple}
\end{figure}

To demonstrate the importance of the pole in the presence of a learned
model, we again show control of \texttt{bodytrack}, this time using
the adaptive control system from the previous subsection with a model
produced by the learner from the first subsection.  We compare the
control system with a carefully tuned pole to the same system using
the default pole provided by the controller developers---recall that
the pole is typically a user-specified parameter.  

\figref{fig:not-simple} shows the results.  The system with the
carefully tuned pole converges because the pole accounts for the error
in the learned model. The pole parameterizes the inertia of the system as how fast should the system change in presence of dynamic environment. If the estimated model is too noisy, the controller should trust the model less and move less frequently even though that might results in slower convergence. On the other hand, the system with the default pole, however
oscillates around the performance target, resulting in a number of
missed deadlines.  In addition, the frames that exceed the desired
performance waste energy because they spend much more time on the big,
inefficient cores than necessary.

The next section describes \SYSTEM{}'s approach to combining learning
and control that abstracts the controller's key parameters, so they
can be learned while the system runs.  Rather than have a user
carefully tune the pole, \SYSTEM{} incorporates the learner's
confidence interval and estimated variance to compute a pole that
provides probabilistic convergence guarantees.  



\section{\SYSTEM{}: Learning Control}
\label{sec:framework}

\begin{figure}
  \includegraphics[width=\columnwidth]{figures/Overview.pdf}
  \caption{\SYSTEM{} overview.}
  \label{fig:overview}
\end{figure}


\figref{fig:overview} shows \SYSTEM{}'s approach of splitting resource
management into separate learning and control tasks and then composing
these individual solutions.  When a new application enters the system,
an adaptive control system allocates resources using a generic model
and records latency and power.  The recorded values are sent to a
learner, which predicts the application's latency and power in all
other resource configurations. The learner extracts those that are
predicted to be Pareto-optimal and packages them in a data structure:
the performance hash table (PHT).  The PHT and the estimated variance
are sent to the controller, which sets its pole and selects an energy
minimal resource configuration with formal guarantees of convergence
to the desired latency.  \SYSTEM{}'s only user-specified parameter
is the latency requirement.

\begin{figure}
  \includegraphics[width=\columnwidth]{figures/Timeline.pdf}
  \caption{Temporal relationship of learning and control.}
  \label{fig:timeline}
  \vskip -1.5em
\end{figure}

\figref{fig:timeline} illustrates the asynchronous interaction between
\SYSTEM{}'s learner and controller. The controller starts---using a
conservative, generic speedup model---when a new application launches.
The controller sends the learner the application's name and device
type (message 1, \figref{fig:timeline}).  The learner determines how
many samples are needed for an accurate prediction and sends this
number to the controller (message 2).  The controller takes these
samples and sends the latency and power of each measured
configuration to the learner (message 3).  The learner may require
time to make predictions; so, the controller does not wait, but
continues with the conservative model.  Once the learner predicts the
optimal configurations, it sends that data and the variance estimate
to the controller (message 4), which uses the learned model from then
on.

\figref{fig:timeline} shows several key points about the relationship
between learning and control.  First, the controller never waits for
the learner: it uses a conservative, less-efficient control
specification until the learner produces application-specific
predictions.  Second, the controller does not continuously communicate
with the learner---this interaction happens once at application
launch.  Third, if the learner crashed, the controller defaults to the
generic adaptive control system.  If the learner crashed after sending
its predictions, the controller does not need to know.  Finally, the
learner and controller have a clearly defined interface, so they can
be run in separate processes or physically separate devices.

We first describe adaptive control.  We then generalize this approach,
separating out parameters to be learned.  Next, we discuss the class
of learners that work with \SYSTEM{}.  Finally, we formally analyze
\SYSTEM{}'s guarantees.


\subsection{Traditional Control for Computing}
A multiple-input, multiple-output (MIMO) controller manages multiple
resources to meet multiple goals.  The inputs are measurements, \eg{}
latency.  The outputs are the resource settings to be used at a
particular time, \eg{} an allocation of big and LITTLE cores and a
clockspeed for each.

These difference equations describe a generic MIMO controller for
allocating $n$ resources to meet $m$ goals at time $t$:\footnote{We
  assume discrete time, and thus, use difference equations rather than
  differential equations that would be used for continuous systems.}
\begin{equation}
\begin{aligned}
&\x(t+1) &=& \mathbf{A} \cdot \x(t)& + \mathbf{B} \cdot \mathbf{u}(t)\\
&\y(t)   &=& \mathbf{C} \cdot \x(t)&,
\end{aligned}
\label{eqn:system:mimo}
\end{equation}
where $\x \in \R^{q}$ is the controller's \emph{state}, an abstraction
of the relationship between resources and goals; $q$ is the
controller's \emph{degree}, or complexity of its internal state.
$\mathbf{u}(t) \in \R^n$ represents the current resource
\emph{configuration}; \ie{} the $i$th vector element is the amount of
resource $i$ allocated at time $t$.  $\y(t) \in \R^{m}$ represents the
value of the goal dimensions at time $t$. The matrices $\mathbf{A} \in
\R^{q \times q}$ and $\mathbf{B} \in \R^{q \times n}$ relate the
resource configuration to the controller state.  The matrix
$\mathbf{C} \in \R^{m \times q}$ relates the controller state to the
expected behavior.  This control definition does not assume the states
or the resources are independent, but it does assume a linear
relationship.

For example, in our ARM big.LITTLE system there are four resources:
the number of big cores, the number of LITTLE cores, and the speeds
for each of the big and LITTLE cores.  There is also a single goal:
latency.  Thus, in this example, $n=4$ and $m=1$. The vector
$\mathbf{u}(t)$ has four elements representing the resource allocation
at time $t$. $q$ is the number of variables in the controller's state
which can vary between 1 to $n$.  The matrices $\mathbf{A}$,
$\mathbf{B}$, and $\mathbf{C}$ capture the linear relationship between
the control state $\x$, the resource usage $\mathbf{u}$, and the
measured behavior.  In this example, we know there is a non-linear
relationship between the resources.  We overcome this difficulty by
tuning the matrices at each time step---approximating the non-linear
system through a series of changing linear formulations.  This
approximation is a form of \emph{adaptive} or \emph{self-tuning}
control \cite{HandbookControl}.  Such adaptive controllers provide
formal guarantees that they will converge to the desired latency
even in the face of non-linearities, but they still assume convexity.

This controller has two major drawbacks.  First, it requires matrix
computation, so its overhead scales poorly in the number of resources
and in the number of goals \cite{Hellerstein2004a,METE}.  Second, the
adaptive mechanisms require users to specify both (1) starting values
of the matrices $\mathbf{A}$, $\mathbf{B}$, and $\mathbf{C}$ and (2)
the method for updating these matrices to account for any
non-convexity in the relationship between resources and latency
\cite{POET,METE,ControlWare,HandbookControl}.  Therefore, typically
100s to 1000s of samples are taken at design time to ensure that the
starting matrices are sufficient to ensure convergence
\cite{FSE2015,sysid,josep-isca2016}.

\subsection{\SYSTEM{} Control System}
To overcome the above issues, \SYSTEM{} abstracts the controller of
\eqnref{system:mimo} and factors out those parameters to be learned.
Specifically, \SYSTEM{} takes three steps to transform a standard
control system into one that works without prior knowledge of the
application to be controlled:
\begin{enumerate}[leftmargin=1em]
\item controlling \emph{speedup} (which is an abstraction of latency) rather than resources;
\item turning speedup into a minimal energy \emph{resource schedule};
\item and exploiting the \emph{problem structure} to solve this
  scheduling problem in constant time.
\end{enumerate}
These steps assume a separate learner has produced predictions of how
resource usage affects latency and power.  The result is that
\SYSTEM{}'s controller runs in constant time without requiring any
user-specified parameters.


% We first describe our formulation for controlling speedup and then
% converting that speedup into resource allocations.

\subsubsection{Controlling Speedup}
\SYSTEM{} converts \eqnref{system:mimo} into a single-input
(latency), single-output (speedup) controlling using $\mathbf{A} =
0$, $\mathbf{B} = b(t)$, $\mathbf{C} = 1$,$\mathbf{u}= speedup$, and
$y = perf$; where $b(t)$ is a time-varying parameter representing the
application's \emph{base speed}---the speed when all resources are
available---and $perf$ is the measured latency. Using these
substitutions, we eliminate $\x$ from \eqnref{system:mimo} to relate
speedup to latency:
\begin{equation}
  lat(t) = 1/(b(t) \cdot speedup(t-1)) \label{eqn:speedup}
\end{equation}
While $b(t)$ is application-specific.  \SYSTEM{} assumes base speed is
time-variant as applications will transition through phases and it
estimates this value online using the standard technique of Kalman
filter estimation \cite{welch2006kalman}. 


\SYSTEM{} must eliminate the error between the target latency and the goal: $ error(t) = goal - 1/lat(t)$.
Given \eqnref{speedup}, \SYSTEM{} uses the integral control law
\cite{Hellerstein2004a}:
\begin{eqnarray}
  % error(t) &=& goal - perf(t) \label{eqn:speedup-error} \\
  % speedup(t) &=& speedup(t-1) - \frac{error(t)}{b}
  speedup(t) &=& speedup(t-1) - \frac{1 - \rho(t)}{b(t)}.error(t)
  \label{eqn:speedup-control}
\end{eqnarray}
which states that the speedup at time $t$ is a function of the
previous speedup, the error at time $t$, the base speed $b(t)$, and
the controller's \emph{pole}, $\rho(t)$.  Standard control techniques
statically determine the pole and the base speed, but \SYSTEM{}
\emph{dynamically sets the pole and base speed to account for error in
  the learner's predictions---an essential modification for providing
  formal guarantees of the combined control and learning systems.}
For stable control, \SYSTEM{} ensures $0 \le \rho(t) < 1$. Small
values of $\rho(t)$ eliminate error quickly, but make the controller
more sensitive to the learner's inaccuracies.  Larger $\rho(t)$ makes
the system more robust at the cost of increased convergence time.
\secref{guarantees} describes how \SYSTEM{} sets the pole, but we
first address converting speedup into a resource allocation.

\subsubsection{Converting Speedup to Resource Schedules}
\SYSTEM{} must map \eqnref{speedup-control}'s speedup into a resource
allocation.  On our example big.LITTLE architecture an allocation
includes big and LITTLE cores as well as a speed for both.  The
primary challenge is that speedups in real systems are discrete
non-linear functions of resource usage, while \eqnref{speedup-control}
is a continuous linear function.  We bridge this divide by assigning
time to resource allocations such that the average speedup over a
control interval is that produced by \eqnref{speedup-control}.

The assignment of time to resource configurations is a
\emph{schedule}; \eg{} spending 10 ms on the LITTLE cores at 0.6 GHz
and then 15 ms on the big cores at 1 GHz. Typically many schedules can
deliver a particular speedup and \SYSTEM{} must find one with minimal
energy.  Given a time interval $T$, the $speedup(t)$ from
\eqnref{speedup-control}, and $C$ different resource configurations,
\SYSTEM{} solves:
\begin{eqnarray}
  \minimize_{\mathbf{\tau} \in \R^{C}} && \sum_{c=0}^{C-1} \tau_c \cdot p_c \label{eqn:power}  \\
  \st %&& \nonumber\\
  && \sum_{c=0}^{C-1} \tau_c \cdot s_c =  speedup(t)T \label{eqn:work} \\
  && \sum_{c=0}^{C-1} \tau_c =  T \label{eqn:deadline} \\
  && 0 \le \tau_c \le T, \qquad \forall c \in \{0,\ldots,C-1\} \label{eqn:time}
\end{eqnarray}
where $p_c$ and $s_c$ are configuration $c$'s estimated
\emph{powerup}---analogous to speedup---and speedup; $\tau_c$ is the
time to spend in configuration $c$.  \eqnref{power} is the objective:
minimizing energy (power times time).  \eqnref{work} states that the
average speedup must be maintained, while \eqnref{deadline} requires
the time to be fully utilized.  \eqnref{time} simply avoids negative
time.

\subsection{Exploiting Problem Structure for Fast Solutions}
%\SYSTEM{} solves \eqnrref{power}{time} on the local device (see
%\figref{fig:timeline}, so the solution must be efficient.  
By encoding the learner's predictions in the performance hash table
(PHT), \SYSTEM{} solves \eqnrref{power}{time} in constant time.

Kim et al. analyze the problem of minimizing energy while meeting a
latency constraint and observe that there must be an optimal
solution with the following properties \cite{kim-cpsna}:
\begin{itemize}[leftmargin=1em]
\item At most two of $\tau_c$ are non-zero, meaning that at most two
  configurations will be used in any time interval.
\item If you chart the configurations in the power and speedup
  tradeoff space (\eg{} the top half of \figref{fig:pht}) the two
  configurations with non-zero $\tau_c$ lie on the lower convex hull
  of the points in that space.
\item The two configurations with non-zero $\tau_c$ are adjacent on
  the convex hull: one above the constraint and one below.
\end{itemize}
%\SYSTEM{} uses these two facts to construct a constant time algorithm
%for finding the optimal solution online.

\begin{figure}
\includegraphics[width=\columnwidth]{figures/performance-hash-table.pdf}
\caption{Data structure to efficiently convert required speedup into a
  resource configuration.}
  \label{fig:pht}
\end{figure}

The PHT (shown in \figref{fig:pht}) provides constant time access to
the lower convex hull.  It consists of two arrays: the first being
pointers into the second: a configuration array, which stores resource
configurations the learner estimates to be on the lower convex hull
sorted by speedup.  Recall speedups are computed relative to the base
speed, which uses all resources.  The largest estimated speedup is
therefore 1.  The first array of pointers has a \emph{resolution}
indicating how many decimal points of precision it captures and it is
indexed by speedup.  The example in \figref{fig:pht} has a resolution
of $0.1$.  Each pointer in the first array points to the configuration
in the second array that has the largest speedup less than or equal to
the index.

\SYSTEM{} computes $speedup(t)$ and uses the PHT to convert speedup
into two configurations: $hi$ and $lo$.  To find the $hi$
configuration, \SYSTEM{} clamps the desired speedup to the largest
index lower than $speedup(t)$, indexes into the configuration array,
and then walks forward until it finds the first configuration with
speedup higher than $speedup(t)$.  To find $lo$, it clamps the desired
speedup to the smallest index higher than $speedup(t)$, indexes into
the configuration array, and then walks backwards until it finds the
configuration with the largest speedup less than $speedup(t)$.

For example, consider the PHT in \figref{fig:pht} and a $speedup(t) =
.65$.  To find $hi$, \SYSTEM{} indexes at .6 and walks up to find
$c=2$ with $s_c=.8$, setting $hi = 2$.  To find $lo$, \SYSTEM{}
indexes the table at .7 and walks backward to find $c=1$ with
$s_c=.2$, setting $lo = 1$.

\SYSTEM{} sets $\tau_{hi}$ and $\tau_{lo}$ by solving:
\begin{eqnarray}
  T &=& \tau_{hi} + \tau_{lo}    \label{eqn:s1} \\
  speedup(t) &=& \frac{s_{hi} \cdot \tau_{hi} + s_{lo} \cdot \tau_{lo}}{T} \label{eqn:s2}
\end{eqnarray}
where the controller provides $speedup(t)$ and the learner predicts
$s_c$.  By solving \eqnsref{s1}{s2}, \SYSTEM{} has turned the
controller's speedup into a resource schedule using predictions stored
in the PHT.

\subsection{\SYSTEM{} Learning Algorithms}
The previous subsection describes a general control system, which can
be customized with a number of different learning methods.  The
requirements on the learner are that it must produce 1) predictions of
each resource configuration's speedup and powerup and 2) estimate of
its own variance $\sigma^2$.  This section describes the general class
of learning mechanisms that meet these requirements.

We refer to application-specific predictors as \emph{online} because
they work for the current application and do not incorporate knowledge
of other applications.  We refer to general predictors as
\emph{offline} as they use prior observations of other applications to
predict the behavior of a new application. A third class of
\emph{transfer learning} combines information from the previously seen
applications and current application to model the future behavior of
the current application \cite{pan2010survey}. Transfer learning
produces highly accurate models since it augments online data with
offline information from other applications.  \SYSTEM{} uses transfer
learners because \SYSTEM{}'s separation of learning and control makes
it easy to incorporate data from other applications---the learner in
\figref{fig:timeline} can simply aggregate data from multiple
controllers. We describe two examples of appropriate transfer learning
algorithms.

\paragraph{Netflix Algorithm:}
The Netflix problem is a famous challenge posted by Netflix to predict
users' movie preferences. The challenge was won by realizing that if 2
users both like some movies, they might have similar taste in other
movies \cite{netflix}. This approach allows learners to borrow large
amounts of data from other applications to answer questions about a
new application.  One formulation of this problem is to assume the
matrix of resource-vs-speedup is low-rank and solve the problem
using mathematical optimization techniques.  The Netflix approach has
been used to predict application response to heterogeneous resources
in data centers \cite{Paragon,quasar}.

\paragraph{ Bayesian Predictors:} A hierarchical Bayesian model (HBM)
provides a statistically sound framework for learning across
applications and devices
\cite{gelman2013bayesian,morris1983parametric}.  In the HBM, each
application has its own model, allowing specificity, but these models
are conditionally dependent on some underlying probability
distribution with a hidden mean and co-variance.  In practice, an HBM
predicts behavior for a new application using a small number of
observations and combining those with the large number of observations
of other applications.  Rather than over-generalizing, the HBM uses
only similar applications to predict new application behavior.  The
HBM's accuracy increases as more applications are observed because
increasingly diverse behaviors are represented in the pool of prior
knowledge \cite{LEO}.  Of course, the computational complexity of
learning also increases with increasing applications.


\subsection{Formal Analysis}
\label{sec:guarantees}
\paragraph{Control System Complexity}

\SYSTEM{}'s control system (see Algorithm \ref{alg:gcs}) runs on the
local device along with the application under control, so its overhead
must be minimal.  In fact, each controller invocation is $O(1)$ .  The
only parts that are not obviously constant time are the PHT lookups.
Provided the PHT resolution is sufficiently high to avoid collisions,
then each PHT lookup requires constant time.
\begin{algorithm}[t]
\caption{\SYSTEM{}'s runtime control algorithm.}
\label{alg:gcs}
\begin{algorithmic}
%\REQUIRE Initialize the controller with a generic prediction of speedup and powerup. Send power and performance samples to learner and receive a PHT.
\WHILE{$True$}
    \STATE    Measure application latency 
    \STATE    Compute required speedup (Equation \eqref{eqn:speedup})
    \STATE    Lookup $s_{hi}$ and $s_{lo}$ with PHT
    \STATE    Compute $\tau_{hi}$ and $\tau_{lo}$ (Equations \ref{eqn:s1} \& \ref{eqn:s2})
    \STATE    Configure to system to $hi$ $\&$ sleep $\tau_{hi}$.
    \STATE    Configure to $lo$ $\&$ sleep $\tau_{lo}$.
\ENDWHILE
\vskip -1.5em
\end{algorithmic}
\end{algorithm}

\paragraph{Control Theoretic Formal Guarantees}

The controller's pole $\rho(t)$ is critical to providing control
theoretic guarantees in the presence of learned---rather than directly
measured---data.  \SYSTEM{} requires any learner estimate not only
speedup and powerup, but also the variance $\sigma$.  \SYSTEM{} uses
this information to derive a lower bound for the pole which guarantees
probabilistic convergence to the desired latency. Specifically, we
prove that with probability 99.7\% \SYSTEM{} converges to the desired
latency if the pole is
$$\Floor{1- \Floor{max(\hat{s})/(min(\hat{s}) - 3\sigma)}_0}_0 \leq \rho(t)
< 1,$$ where $\Floor{x}_0 = \max(x,0)$ and $\hat{s}$ is the estimated
speedup. See appendix A for the proof. Users who need higher
confidence can set the scalar multiplier on $\sigma$ higher; \eg{}
using $6$ provides a 99.99966\% probability of convergence.

Thus we provide a lower-bound on the value of $\rho(t)$ required for
confidence that \SYSTEM{} converges to the desired latency.  This
pole value only considers latency, and not energy efficiency.  In
practice, we find it better to use a higher pole based on the
\emph{uncertainty} between the controller's observed energy efficiency
and that predicted by the learner.  We follow prior work
\cite{Tokic2010} in quantifying uncertainty as $\beta(t)$, and setting
the pole based on this uncertainty:
\begin{equation}
  \begin{array}{rcl}
    \beta(t) &=&  \text{exp}{\left(- \left( \left|   \frac{\bar{s}(t)}{\bar{p}(t)}  -\frac{ \hat{s}(t)}{\hat{p}_(t)} \right| \right) /5\right)} \\
    \rho(t) &=& \frac{1-\beta(t)}{1+\beta(t)} 
  \end{array}
  \label{eqn:uncer}
\end{equation}
where $\bar{s}$ and $\bar{p}$ are the measured values of speedup and
powerup and $\hat{s}$ and $\hat{p}$ are the estimated values from the
learner.  This measure of uncertainty captures both power and
latency.  We find that it is generally higher than the pole value
given by our lower bound, so in practice \SYSTEM{} sets the pole
dynamically to be the higher of the two values and \SYSTEM{} makes
spot updates to the estimated speedup and power based on its
observations.

\section{Experimental evaluations}


\begin{figure*}[t]
  \input{img/accuracy-performance.tex}
   \vskip -1em
  \caption{Accuracy performance}
  \label{fig:accuracy}
\end{figure*}


\begin{figure*}[t]
  \input{img/accuracy-power.tex}
   \vskip -1em
  \caption{Accuracy power}
  \label{fig:accuracy}
\end{figure*}

\begin{figure*}[t]
  \input{img/single-app-performance.tex}
   \vskip -1em
  \caption{Single Applications MAPE}
  \label{fig:single-perf}
\end{figure*}

\begin{figure*}[t]
  \input{img/single-app-energy.tex}
   \vskip -1em
  \caption{Single Applications energy}
  \label{fig:single-energy}
\end{figure*}


\begin{figure*}[t]
  \input{img/multi-app-performance.tex}
   \vskip -1em
  \caption{Multi Applications MAPE}
  \label{fig:multi-perf}
\end{figure*}

\begin{figure*}[t]
  \input{img/multi-app-energy.tex}
   \vskip -1em
  \caption{Multi Applications Energy}
  \label{fig:multi-energy}
\end{figure*}

%\PUNT{
\begin{figure*}[t]
  \begin{tikzpicture}
\begin{centering}

\definecolor{s1}{RGB}{228, 26, 28}
\definecolor{s2}{RGB}{55, 126, 184}
\definecolor{s3}{RGB}{77, 175, 74}
\definecolor{s4}{RGB}{152, 78, 163}
\definecolor{s5}{RGB}{255, 127, 0}

\begin{groupplot}[
    group style={
        group name=plots,
        group size=1 by 2,
        xlabels at=edge bottom,
        xticklabels at=edge bottom,
        vertical sep=5pt
    },
height=3.5cm,
width=0.95\columnwidth,
xmajorgrids,
ymajorgrids,
grid style={dashed},
xmin=0,
xmax=100,
yticklabel pos=left,
enlargelimits=false,
tick align = outside,
tick style={white},
xticklabel shift={-5pt},
yticklabel shift={-5pt},
ylabel shift={-2pt},
ylabel style={align=center},
unbounded coords=jump,
]

\nextgroupplot[ylabel={\footnotesize Accuracy\\(Performance)}, % Performance
xtick={0,20,40,60,80,100},
ytick={0.0,0.3,0.6,0.9,1.0},
yticklabels={,0.3,0.6,0.9,1.0},
yticklabel style={font=\footnotesize},
ymin=0,
ymax=1,
legend entries={{$\mathsf{LEO}$},{$\mathsf{Online}$}},
legend style={draw=none,at={(0.5,1.4)},anchor=north,legend columns=4,line width=5pt},
]
%\addplot[thick, solid, color=s3] table[x index=0,y index=1,col sep=tab] {img/old/x264-phases-clover-dvfs.txt};
\addplot[thick, solid, color=s3] table[x index=0,y index=1,col sep=tab] {img/sample_accuracy.txt};
\addplot[thick, solid, color=s4] table[x index=0,y index=3,col sep=tab] {img/sample_accuracy.txt};
%\addplot[thick, solid, black] coordinates {(0, 1) (4500, 1)};
%\addplot[thick, dashed, black] coordinates {(1500,0) (1500, 2)};
%\addplot[thick, dashed, black] coordinates {(3000,0) (3000, 2)};


\nextgroupplot[ylabel={\footnotesize Accuracy\\ (Power)}, % Power
ytick={0.0,0.3,0.6,0.9,1.0},
yticklabels={,0.3,0.6,0.9,1.0},
yticklabel style={font=\footnotesize},
ymin=0,
ymax=1,
xlabel={\footnotesize \% Samples for training},
xlabel near ticks,
xtick={0,20,40,60,80,100},
xticklabels={0,20,40,60,80,100},
xticklabel style={font=\footnotesize},
]
\addplot[thick, solid, color=s3] table[x index=0,y index=2,col sep=tab] {img/sample_accuracy.txt};
\addplot[thick, solid, color=s4] table[x index=0,y index=4,col sep=tab] {img/sample_accuracy.txt};
%\addplot[thick, dashed, black] coordinates {(1500,0) (1500, 250)};
%\addplot[thick, dashed, black] coordinates {(3000,0) (3000, 250)};

\end{groupplot}
\end{centering}

\end{tikzpicture}

   \vskip -1em
  \caption{Multi Applications Energy}
  \label{fig:multi-energy}
\end{figure*}
%}

\section{Related Work}

We discuss related work in managing resources to meet performance
goals and reduce energy.  

\subsection{Machine Learning}
There are a huge array of different learning techniques that are
applicable to different problems.  As discussed in in \secref{framework:HBM}
we break learning for resource management into three categories:
offline, online, and hybrid approaches.  

\subsubsection{Offline Learning}
The offline approaches build models before deployment and then use
those fixed models to allocate resources
\cite{Yi2003,LeeBrooks2006,CPR,ChenJohn2011,petabricksStatic}.  In
these approaches, the model-building phase is generally very expensive,
requiring both a large number of samples and substantial computation
to turn those samples into a model that accurately captures the
relationship between the observed features and the behavior to be
estimated.  Applying the model online, however, tends to be low
overhead.  The main drawback is that the models are not updated as the
system runs, so there is no chance to correct mistakes or adapt to
specific workloads.

A good example of an offline approach applies learning to render web
pages on mobile systems with low energy \cite{reddiHPCA2013}.  This
system is similar in spirit to \SYSTEM{}.  It builds an offline model
mapping web page features into estimations of performance for
different core types.  When a new page is downloaded, the system
quickly estimates the resource need to render the web page and uses the
lowest energy resources that will still maintain user satisfaction.
The mapping of web pages to resource use is very complicated and this
approach deals with that complication.  It does not, however, address
system dynamics; \eg{} when other apps are running concurrently with the
web browser.

\subsubsection{Online Learning}
Online techniques use observations of the current application to tune
system resource usage for that application
\cite{Li2006,Flicker,ParallelismDial,Ponamarev,petabricksDynamic,LeeBrooks}.
For example, Flicker is a configurable architecture and optimization
framework that uses only online models to maximize performance under a
power limitation \cite{Flicker}.  Another example, ParallelismDial,
uses online adaptation to tailor parallelism to application workload
\cite{ParallelismDial}.



\subsubsection{Hybrid Approaches}
Some approaches combine offline predictive models with online
adaptation
\cite{Zhang2012,packandcap,Winter2010,dubach2010,Koala,Cinder,
  wu2012inferred}.  For example, Dubach et al.  propose such a combo
for optimizing the microarchitecture of a single core
\cite{dubach2010}.  Such predictive models have also been employed at
the operating systems level to manage system energy consumption \cite{Koala,Cinder}.
\cite{wu2012inferred}.


Still other approaches combine offline modeling with online model
updates \cite{JouleGuard,Bitirgen2008,Ipek}.  Bitirgen et
al use an artificial neural network to allocate resources to multiple
applications in a multicore \cite{Bitirgen2008}.  The neural network
is trained offline and then adapted online using measured feedback.
This approach optimizes performance but does not consider power or
energy minimization.  LEO, the system we extend in this paper, also
uses a combination of offline and online approaches.  LEO collects
data about a number of applications offline and combines that with a
small number of observations made online for the current application
\cite{LEO}.

\subsection{Control}
Almost all control solutions can be thought of as a combination of
offline model building with online adaptation.  Usually the model
building involves substantial empirical measurement and statistical
regression to build a model that is then used to synthesize a control
system
\cite{Wu2004,TCST,Chen2011,PTRADE,POET,ControlWare,Agilos,Rajkumar,Sojka,Raghavendra2008}.
The combination of offline
learning and control works well over a narrow range of applications, as the offline models capture the
general behavior of the entire class of application and require
negligible online overhead.  This focused approach is extremely
effective for multimedia applications
\cite{grace2,flinn99,flinn2004,xtune,TCST} and web-servers
\cite{Horvarth,LuEtAl-2006a,SunDaiPan-2008a} because the workloads can
be characterized ahead of time so that the models produce sound
control.

Indeed, the need for good models is the central tension in developing
control for computing systems.  It is always possible to build a
controller for a specific application and system by extensively
modeling that pair.  More general controllers which work with a range
of applications have addressed this issue with models in several ways.
Some provide control libraries that encapsulate control
functionality and require users to input a model
\cite{ControlWare,Sojka,Rajkumar,POET}.  Others
automatically synthesize both a model and a controller for either
hardware \cite{josep-isca2016} or software \cite{ICSE2014,FSE2015}.
JouleGuard combines learning for energy efficiency with control for
managing application parameters \cite{JouleGuard}.  In JouleGuard, a
learner adapts the controller's coefficients to model certainty, but
JouleGuard's learner does not produce a new model for the controller.
Because JouleGuard's learner runs on the same device as the controlled
application, it must be computationally efficient and thus it cannot
identify correlations across applications or even different resource
configurations.  \SYSTEM{} is unique in that a remote server generates
an application-specific model automatically.  By offloading the
learning task, we are able to (1) combine data from many applications
and systems and (2) apply computationally expensive, but highly accurate
learning techniques.

Perhaps the most similar approach to \SYSTEM{} is Carat \cite{carat}.
Carat aggregates data across many mobile devices and sends a report to
human users about how to configure their device to increase battery
life.  While both Carat and \SYSTEM{} learn across devices, they have
very different goals.  Carat's goal is to return very high-level
information to human users; \eg{} you should update a driver to extend
battery life.  \SYSTEM{} returns lower-level models to another
automated system that will apply those models to save energy.


\section{Conclusion}
Much recent work builds systems to support learning, \SYSTEM{} uses
learning to build better systems.  \SYSTEM{} is a resource manager
that meets application latency requirements with minimal energy, even
without prior knowledge of the application.  \SYSTEM{} is the first
work that provides formal guarantees that it will converge to the
required latency despite not having prior knowledge.  \SYSTEM{}
achieves this breakthrough by using learning to model complex resource
interaction and control theory to manage system dynamics.  \SYSTEM{}
proposes foundational techniques that allow control to be applied
using noisy learned models---instead of ground truth models---while
maintaining formal guarantees.  We demonstrate \SYSTEM{}'s
effectiveness with a case study using embedded applications on a
heterogeneous processor.  Compared to prior learning and control
approaches, \SYSTEM{} is the only approach that provides reliable
latency for all applications with near minimal energy.

\appendix
\section{Probabilistic Convergence Guarantees}
\begin{theorem}
  Let $\mathbf{s_c}$ and $\hat{\mathbf{s_c}}$ denote the true and
  estimated speedups of various configurations in set $C$ as
  $\mathbf{c} \in \mathbb{R}^{|C|}$. Let $\sigma$ denote the
  estimation error for speedups such that, $\hat{s_i} \sim N(s_i,
  \sigma^2)$ $\forall$ $i$. We show that with probability greater than
  99.7\%, the pole $\rho(t)$ can be chosen to lie in the range,
  $\Floor{1- \Floor{max(\hat{s})/(min(\hat{s}) - 3\sigma)}_0}_0, 1)$,
  where $\Floor{x}_0 = \max(x,0)$.
\end{theorem}

\begin{proof}
  Let $\Delta$ denote the multiplicative error over speedups, such
  that $ \widehat{s_c}\Delta = s_c $. To guarantee convergence the
  value of pole, $\rho(t)$ can vary in the range
  $\Floor{1-\frac{2}{\Delta})}_0, 1)$ \cite{ICSE2014}. The lower
  $\rho(t)$, the faster the convergence. Equations \ref{eqn:s1} \&
  \ref{eqn:s2} show that any $s(t)$ is a linear combination of two
  speedups:
\begin{align}
  s(t) = \hat{s}_{hi} \cdot \tau_{hi} + \hat{s}_{lo} \cdot (T -
  \tau_{hi})
\end{align}
\begin{align}
  \widehat{s}(t) = s_{hi} \cdot \tau_{hi} + s_{lo} \cdot (T -
  \tau_{hi})
\end{align}

We can upper bound and lower bound each of these terms,
\begin{align}
s(t) \leq T \hat{s}_{hi} \;\; \text{and} \;\; \hat{s}(t) \geq T s_{lo}
\end{align}

The speedup estimates are close to the actual speedups since $\hat{s}
\sim N(s, \sigma^2)$, therefore with probability greater than 99.7\%
and the speedups can be given by, $s_{lo} \geq \hat{s}_{lo} - 3
\sigma$. Hence, $\hat{s}(t) \geq T (\hat{s}_{lo} -3 \sigma)$. Since,
over all configurations, $\Delta \leq
\Floor{max(\hat{s})/(min(\hat{s}) - 3\sigma)}_0$, we can choose the
pole from the range, $(\Floor{1- \Floor{max(\hat{s})/(min(\hat{s}) -
    3\sigma)}_0}_0, 1)$.


\end{proof}


\newpage
\clearpage
%\bibliographystyle{abbrv}
%\bibliography{reference}
\printbibliography
\end{document}
