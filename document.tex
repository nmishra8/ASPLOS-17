%\documentclass[10pt,twocolumn]{article}
\documentclass[pageno]{jpaper}

\newcommand{\asplossubmissionnumber}{209}
%\usepackage{times}
\usepackage{fullpage}
\usepackage[utf8]{inputenc}
\usepackage[english]{babel}
\usepackage{amstext,amssymb,amsmath,amsthm}
\usepackage{verbatim}
\usepackage{subfigure}
\usepackage{color}
\usepackage{paralist}
\usepackage{multirow}
\usepackage{listings}
\usepackage{array}
\usepackage{graphicx,color}
\usepackage{moreverb}
\usepackage{xspace}
\usepackage{algorithmic}
\usepackage{algorithm}
\usepackage{times}
\usepackage{relsize}
\newcolumntype{x}[1]{>{\centering\arraybackslash}p{#1}}
\newlength\SUBSIZE
\newcommand{\ONECOLFIGMUL}{0.81}
\newcommand{\TALLFIGMUL}{0.75}
\newcommand{\ie}{\textit{i.e.},~}
\newcommand{\eg}{\textit{e.g.},~}
\usepackage{caption}
%\usepackage{subcaption}

\newcounter{cnt}
% \def\full{0}
\newtheorem{lem}[cnt]{Lemma}
\newtheorem{thm}[cnt]{Theorem}
\newtheorem{problem}[cnt]{Problem}
\newtheorem{defn}[cnt]{Definition}
\newtheorem{fact}[cnt]{Fact}
\newtheorem{example}[cnt]{Example}
\newtheorem{corollary}[cnt]{Corollary}
\newtheorem{assumption}[cnt]{Assumption}
\newtheorem{claim}[cnt]{Claim}
\newcommand{\ip}[2]{\langle #1,#2\rangle}
\renewcommand{\algorithmicrequire}{\textbf{Input:}}
\renewcommand{\algorithmicensure}{\textbf{Output:}}
\newcommand{\nptheta}{\hat\theta}
\newcommand{\symdiff}{\Delta}
\newcommand{\noise}{{\sf Noise}}
\newcommand{\privtheta}{\theta_{priv}}
\renewcommand{\paragraph}[1]{\vspace{3pt}\noindent\textbf{#1}}
\newcommand{\scrX}{\ensuremath{\mathcal{X}}}
\newcommand{\scrZ}{\ensuremath{\mathcal{Z}}}
\newcommand{\rA}{\ensuremath{\rightarrow}}
\newcommand{\rrA}{\ensuremath{\longrightarrow}}
\newcommand{\qB}{\ensuremath{\mathbf{q}}}
\newcommand{\XB}{\ensuremath{\mathbf{X}}}
\newcommand{\zeroB}{\ensuremath{\mathbf{0}}}
\newcommand{\sm}{\mbox{\textendash}}
\newcommand{\ltwo}[1]{\left\|#1\right\|_2}
\newcommand{\lone}[1]{\left\|#1\right\|_1}
\newcommand{\eps}{\epsilon}
\newcommand{\A}{\mathcal{A}}
\newcommand{\D}{\mathcal{D}}
\newcommand{\T}{\mathcal{T}}
\newcommand{\R}{\mathbb{R}}
\newcommand{\K}{\mathcal{K}}
\newcommand{\mineig}{\eta}
\newcommand{\I}{\mathbb{I}}
\newcommand{\E}{\mathbb{E}}
\newcommand{\F}{\mathcal{C}}
\newcommand{\hatw}{\hat{w}}
\newcommand{\hatv}{{\hat{v}}}
\newcommand{\hatV}{{\hat{V}}}
\newcommand{\hatW}{{\hat{W}}}
\newcommand{\kl}{{\sf KL}}
\newcommand{\dagw}{w^\dagger}
\newcommand{\tildew}{\tilde{w}}
\newcommand{\tildeF}{\tilde{F}}
\newcommand{\tildef}{\tilde{f}}
\newcommand{\re}{\mathbb{R}}
\newcommand{\B}{\mathbb{B}}
\newcommand{\bP}{\mathbb{P}}
\renewcommand{\P}{\mathcal{P}}
\newcommand{\grad}{\bigtriangledown}
\newcommand{\mypar}[1]{\smallskip
\noindent{\bf\em {#1}.}}
\newcommand{\etal}{\emph{et al.}}
\newcommand{\negl}{\text{negl}}
\newcommand{\mynote}[3]{\marginpar{\parbox{0.7in}{\tiny {\color{#2} {\sc #1}: {\sf #3}}}}}
\newcommand{\atnote}[1]{\mynote{at}{blue}{#1}}
\newcommand{\nmnote}[1]{\mynote{prmohan}{red}{#1}}
\newcommand{\ignore}[1]{}
\newcommand{\rpm}{\sbox0{$1$}\sbox2{$\scriptstyle\pm$}  \raise\dimexpr(\ht0-\ht2)/2\relax\box2 }
\newcommand{\TODO}[1]{{\bf TODO: #1}}
\newcommand{\NOTE}[1]{{\bf NOTE: #1}}

\newcommand{\name}{\textsc{GUPT}\xspace}
\newcommand{\nameplain}{GUPT\xspace}
\newcommand{\aname}{a \textsc{GUPT}\xspace}
\newcommand{\Aname}{A \textsc{GUPT}\xspace}
\newcommand{\C}{\mathcal{C}}
\newcommand{\hatd}{{\hat d}}
\newcommand{\hatf}{{\hat f}}
\newcommand{\tr}{{\sf tr}}
\newcommand{\Lap}{\mathsf{Lap}}
\newcommand{\M}{\mathcal{M}}
\newcommand{\besta}{a_{{\sf best}}}
\newcommand{\Vect}{\text{vec}}
\newcommand{\Vech}{\text{vech}}
\newcommand{\rank}{\text{rank}}
\newcommand{\diag}{\text{diag}}
\newcommand{\nnz}{\text{nnz}}
\newcommand{\y}{\mathbf{y}}
\newcommand{\w}{\mathbf{w}}
\newcommand{\x}{\mathbf{x}}
\newcommand{\z}{\mathbf{z}}
\DeclareMathOperator*{\argmin}{arg\,min}
\DeclareMathOperator*{\argmax}{arg\,max}

\usepackage{balance}
%\usepackage{times}
%\usepackage{fullpage}
%\usepackage{enumitem}
%\usepackage[shortlabels]{enumitem}
%\usepackage{mathtools}
\usepackage[normalem]{ulem}
\usepackage{todonotes}
\usepackage{amstext,amssymb,amsmath,amsthm}
%\DeclarePairedDelimiter{\ceil}{\lceil}{\rceil}
%\DeclarePairedDelimiter\floor{\lfloor}{\rfloor}
\usepackage[natbib=true,backend=bibtex,firstinits=true,style=numeric-comp,sorting=nyt,defernumbers,maxnames=99,maxcitenames=99]{biblatex}


\setlist{noitemsep,topsep=0pt}

\setlength{\itemsep}{0pt}
\setlength{\textheight}{9.0in}
\setlength{\columnsep}{0.25in}
\setlength{\textwidth}{6.50in}
\setlength{\topmargin}{0.0in}
\setlength{\headheight}{0.0in}
\setlength{\headsep}{0.0in}
%
\setlength{\abovecaptionskip}{1pt plus 2pt minus 2pt}
\setlength{\textfloatsep}{3pt}



%--------
\definecolor{mygreen}{rgb}{0,0.6,0}
\definecolor{mygray}{rgb}{0.5,0.5,0.5}
\definecolor{mymauve}{rgb}{0.58,0,0.82}
\usepackage{listings}
\lstset{ %
  backgroundcolor=\color{white},   % choose the background color; you must add \usepackage{color} or \usepackage{xcolor}
  basicstyle=\scriptsize\ttfamily, % the size of the fonts that are used for the code
  breakatwhitespace=false,         % sets if automatic breaks should only happen at whitespace
  breaklines=true,                 % sets automatic line breaking
  captionpos=b,                    % sets the caption-position to bottom
  commentstyle=\color{mygreen},    % comment style
  deletekeywords={...},            % if you want to delete keywords from the given language
  escapeinside={\%*}{*)},          % if you want to add LaTeX within your code
  extendedchars=true,              % lets you use non-ASCII characters; for 8-bits encodings only, does not work with UTF-8
  frame=leftline,                  % adds a frame around the code
  keepspaces=true,                 % keeps spaces in text, useful for keeping indentation of code (possibly needs columns=flexible)
  keywordstyle=\color{blue},       % keyword style
  morekeywords={*,...},            % if you want to add more keywords to the set
  numbers=left,                    % where to put the line-numbers; possible values are (none, left, right)
  numbersep=5pt,                   % how far the line-numbers are from the code
  numberstyle=\tiny\color{mygray}, % the style that is used for the line-numbers
  rulecolor=\color{black},         % if not set, the frame-color may be changed on line-breaks within not-black text (e.g. comments (green here))
  showspaces=false,                % show spaces everywhere adding particular underscores; it overrides 'showstringspaces'
  showstringspaces=false,          % underline spaces within strings only
  showtabs=false,                  % show tabs within strings adding particular underscores
  stepnumber=1,                    % the step between two line-numbers. If it's 1, each line will be numbered
  stringstyle=\color{black},     % string literal style
  tabsize=1,                       % sets default tabsize to 2 spaces
  title=\lstname                   % show the filename of files included with \lstinputlisting; also try caption instead of title
}

\usepackage{pgfplots}
% options for pgfplots
\pgfplotsset{compat=1.8,compat/show suggested version=false}
\usetikzlibrary{calc,trees,arrows,patterns,plotmarks,shapes,snakes,er,3d,automata,backgrounds,topaths,decorations.pathmorphing,decorations.markings}
%\pgfplotsset{compat=newest}
\pgfplotsset{
   /pgfplots/bar  cycle  list/.style={/pgfplots/cycle  list={%
        {black,fill=black!30!white,mark=none},%
        {black,fill=red!30!white,mark=none},%
        {black,fill=green!30!white,mark=none},%
        {black,fill=yellow!30!white,mark=none},%
        {black,fill=brown!30!white,mark=none},%
     }
   },
}
% begin of externalization
\usetikzlibrary{external}
\tikzexternalize[prefix=out/]
\tikzexternalize
\usetikzlibrary{patterns}
\usepgfplotslibrary{groupplots}
\pgfplotsset{
every axis label/.append style={font=\footnotesize},
tick label style={font=\footnotesize},
}
\makeatletter
\g@addto@macro\normalsize{%
  \setlength\abovedisplayskip{4pt plus 2pt minus 1pt}
  \setlength\belowdisplayskip{4pt plus 2pt minus 1pt}
  \setlength\abovedisplayshortskip{4pt plus 2pt minus 1pt}
  \setlength\belowdisplayshortskip{4pt plus 2pt minus 1pt}
}
\makeatother

\newif{\ifanonymous}
\anonymoustrue

\newcommand{\cutout}[1]{}
\newcommand{\smallcaption}[1]{\caption[#1]{{\protect\small \protect\bf #1}}}
\newcommand{\dids}{{\sc dids}}
\newcommand{\us}{\,$\mu$s}
\newcommand{\ms}{\,ms}
\newcommand{\KB}{\,KB}
\newcommand{\MB}{\,MB}
\newcommand{\GB}{\,GB}
\newcommand{\MHz}{\,MHz}
\newcommand{\GHz}{\,GHz}
\newcommand{\eg}{\emph{e.g.},}
\newcommand{\ie}{\emph{i.e.},}
\newcommand{\y}{\mathbf{y}}

\newcommand{\SYSTEM}{CALOREE}
%\newcommand{\TODO}[1]{\textbf{TODO: 1}}
\definecolor{gray}{gray}{0.75}
\newcommand{\TODO}[1]{\textcolor{gray}{\textbf{\ [TODO:\ #1]\ }}}
\newcommand{\secref}[1]{Section~\ref{sec:#1}}
\newcommand{\Secref}[1]{Section~\ref{#1}} 
\newcommand{\PUNT}[1]{}
\newcommand{\figref}[2][{}]{{\figurename~\ref{#2}#1}} 
\newcommand{\figsref}[2]{Figures~\ref{#1} and~\ref{#2}}
\newcommand{\tblref}[1]{Table~\ref{tbl:#1}}
\newcommand{\algref}[1]{Algorithm~\ref{alg:#1}}
\renewcommand{\textfraction}{0.05} 
%\newcommand{\figref}[1]{Figure~\ref{fig:#1}}
%\newcommand{\figsref}[2]{Figures~\ref{fig:#1} and~\ref{fig:#2}}
\newcommand{\figrref}[2]{Figures~\ref{#1}--\ref{#2}}
%\newcommand{\secref}[1]{Section~\ref{sec:#1}}
\newcommand{\secsref}[2]{Sections~\ref{sec:#1} and~\ref{sec:#2}}
\newcommand{\eqnref}[1]{Eqn.~\ref{eqn:#1}}
\newcommand{\eqnsref}[2]{Eqns.~\ref{eqn:#1} and~\ref{eqn:#2}}
\newcommand{\eqnrref}[2]{Eqns.~\ref{eqn:#1}--\ref{eqn:#2}}
%\newcommand{\insref}[1]{Instruction~\ref{ins:#1}}
%\newcommand{\tblref}[1]{Table~\ref{tbl:#1}}
\newcommand{\appref}[1]{Appendix~\ref{app:#1}}


\newcommand{\minimize}{minimize}
\newcommand{\st}{s.t.}

\DeclareMathOperator*{\argmin}{arg\,min}
\DeclareMathOperator*{\argmax}{arg\,max}

\bibliography{reference}

\begin{document}

\title{Big Data for LITTLE Cores \\ {\large Combining Learning and Control for
  Mobile Energy Efficiency}} \author{Anonymous Submission} \date{}

\maketitle
\thispagestyle{empty}


\begin{abstract}

  
  Control systems are a proven method for allocating resources to meet
  streaming applications' performance requirements with minimal energy
  on mobile and embedded devices. Control designs, however, require
  \emph{a priori} models to estimate an application's performance and
  energy based on its resource usage -- a controller designed for one
  application and system must be redesigned for a new deployment.  We
  propose CALOREE to achieve control theoretic benefits -- \ie{}
  formal, bounded convergence to required performance in dynamic
  environments -- without \emph{a priori} models.  CALOREE combines a
  generalized control system (GCS) \PUNT{that runs on a local device}
  with a hierarchical Bayesian model (HBM)\PUNT{that runs on a remote
    server}. The HBM runs on a remote server and aggregates data from
  multiple applications and devices to produce a highly accurate model
  that it sends to the GCS to customize control for the local device
  and application. We extend standard control analysis to show that
  CALOREE provides probabilistic convergence guarantees despite having
  no prior model of the controlled application.

  We implement CALOREE using ARM big.LITTLE boards for streaming
  applications and an x86 server for the HBM.  We test in both single-
  and multi-application environments (where applications compete for
  resources).  Compared to state-of-the-art learning and control
  techniques, CALOREE consistently provides the most reliable
  performance: a worst case performance error of only 12\% compared to
  70-80\% for prior approaches.  Additionally, CALOREE provides the
  lowest energy: delivering average savings from 8-47\%.


  \PUNT{CALOREE has three main components: (1) a remote learner that
    aggregates data across devices and applications to model
    application performance and energy, (2) a lightweight control
    system that uses the learned models to customize control for a
    particular application, and (3) the interfaces that map non-convex
    learned models to the continuous linear models used by the
    controllers.

  We implement CALOREE's learning on an x86 server and stream
  processing on heterogeneous ARM big.LITTLE devices. We test in both
  \emph{stable} and \emph{unstable} environments (where available
  resources change over time) and we compare to state-of-the-art
  learning and control-based resource managers.  In all test
  scenarios, \SYSTEM{} provides the lowest error between the required
  and delivered performance. Depending on the scenario, \SYSTEM{}
  provides 12-25\% average energy reduction compared to
  state-of-the-art approaches.  Additionally, \SYSTEM{}'s worst case
  performance and energy is uniformly better than prior approaches, by
  factors as high as $2\times$.
}


\PUNT{
  Streaming sensor processing forms a foundational workload for
  embedded, mobile, and Internet-of-things. Stream processing on these
  platforms requires both reliable performance and low energy, and two
  challenges must be addressed to meet these conflicting requirements.
  The first is \emph{complexity}: hardware exposes heterogeneous
  resources which interact in complicated ways. The second is
  \emph{dynamics}: performance must keep up with the data stream
  despite unpredictable changes in operating environment.  Prior work
  shows that machine learning addresses the complexity challenge and
  control systems tackle dynamics, but streaming sensor processing on
  energy-limited devices requires that both challenges be met
  simultaneously.

  To address both complexity and dynamics, we propose \SYSTEM{}, a
  resource management system that integrates hierarchical Bayesian
  learning with a lightweight control system.  The learning framework
  runs on a remote server aggregating data from multiple applications
  and devices to estimate performance/energy models for different
  resource configurations.  The controller runs on energy-limited
  devices using the learned models to deliver required performance for
  streaming processing with minimal energy.  We test \SYSTEM{} by
  implementing its learning on an x86 server and stream processing on
  four heterogeneous ARM big.LITTLE devices. We test in both
  \emph{stable} and \emph{unstable} environments (where available
  resources change over time) and we compare to state-of-the-art
  learning and control approaches.  In all test scenarios, \SYSTEM{}
  provides the lowest error between the required and delivered
  performance. Depending on the scenario, \SYSTEM{} also provides
  12-25\% average energy reduction compared to state-of-the-art
  learning and control approaches.  Additionally, \SYSTEM{}'s worst
  case performance and energy is uniformly better than prior
  approaches.  Thus, \SYSTEM{}'s unique combination of learning and
  control is well-suited to supporting streaming sensor processing in
  embedded, mobile, and IoT platforms.
}


  \PUNT{ Mobile systems must deliver performance to interactive
    applications while simultaneously conserving resources to extend
    battery life.  There are two central challenges to meeting these
    conflicting goals: (1) the complicated optimization spaces arising
    from hardware heterogeneity and (2) dynamic changes in application
    behavior and resource availability.  Machine learning techniques
    handle complicated optimization spaces, but do not incorporate
    models of system dynamics; control theory provides formal
    guarantees of dynamic behavior, but struggles with non-linear
    system models.  In this paper, we propose \SYSTEM{}, a combination
    of learning and control techniques to meet performance
    requirements on heterogeneous devices in unpredictable
    environments.  \SYSTEM{} combines a hierarchical Bayesian model
    (HBM) with a lightweight control system (LCS).  The HBM runs
    remotely, learning customized performance/power models.  The LCS
    runs on the mobile system and tunes resource usage to meet
    performance goals.  The Performance Hash Table (PHT) is the
    interface between the two and allows the LCS to apply the learned
    models in constant time.  We test \SYSTEM{}'s ability to manage
    ARM big.LITTLE systems.  Compared to existing learning and control
    methods, \SYSTEM{} delivers more reliable performance -- only 2\%
    error compared to 4.5-5.4\% for learning and 4.7\% for control --
    and lower energy -- within 7\% of optimal on average as compared
    to 25-52\% for learning and 26\% for control. Furthermore, we
    demonstrate \SYSTEM{}'s ability to meet performance and energy
    goals in dynamic systems with phase changes and multiple
    applications running on the same system.  }



  \PUNT{ When multiple applications compete for resources, these
    numbers improve: 7\% error compared to 11-15\% for learning and
    9\% for control and improvements of 2-20\% and 3\%, respectively,
    in energy efficiency.}
\end{abstract}

\section{Introduction}
% Dennard Scaling making energy essential.  Architects address energy
% by making more complicated processors which expose resources to
% software management.  For a wide range of applications, need to meet
% performance goals with minimal energy.
Large classes of computing systems---from embedded to cloud---must
deliver reliable performance to users while minimizing energy to
increase battery life or decrease operating costs.  To address these
conflicting requirements, hardware architects have begun to expose
increasingly diverse, heterogeneous resources with an array of
different performance and energy tradeoffs.  It is then software's
responsibility to allocate these resources such that performance
requirements are met with minimal energy.


% Difficulties of meeting performance with minimal energy. (1)
% complexity---heterogeneous resources---and (2) dynamics---adjust to
% unforeseen changes in workload and environment.
There are two primary difficulties in determining how to allocate
heterogeneous resources.  The first is \emph{complexity}---these
resources interact in complicated ways, leading to non-convex
optimization spaces.  The second is \emph{dynamics}---perfor\-mance
requirements must be met despite unpredictable disturbances; \eg{}
phases in input or changes in operating environment.  Prior work
addresses each of these difficulties individually.

% Prior approaches addressed each of these difficulties individually.
% ML---can handle complexity.  ML advantages: can handle
% non-convexity, avoid local optima, get to true optimal solution. ML
% disadvantages: advanced techniques are expensive and no notion of
% dynamics.  Control---handles dynamics.  Control advantages: formally
% analyzable guarantees despite dynamics.  Control disadvantages:
% relies on good models---no local optima, bounded error.
Many machine learning approaches accurately model the complex
performance/power tradeoff spaces inherent to heterogeneous computing
systems
\cite{reddiHPCA2013,dubach2010,Bitirgen2008,Ipek,Koala,LEO,Flicker,Ponamarev}.
Such ML approaches handle non-convexity, identifying local optima to
find globally optimal solutions; however, these techniques are
computationally expensive and lack support for dynamics; \ie{} when
the environment changes the expensive model building process must be
restarted.  Control theoretic solutions perform efficient resource
allocation in the presence of system dynamics
\cite{Hellerstein2004a,Chen2011,PTRADE,POET,ControlWare,Agilos,grace2}.
Control provides formally analyzable guarantees that the system will
deliver the required performance, but these guarantees require
accurate models and do not support the non-convexity arising from
complicated interactions between heterogeneous resources.


% Want to combine learning and control to address both difficulties
% simultaneously.  Need an interface that allows learned models to be
% used by control system.  Challenges: (1) overhead and (2) formal
% guarantees.  
Our goal is to combine learning and control to ensure performance
requirements are met with minimal energy in complex and dynamic
environments.  The challenges to combining these techniques are (1)
mitigating learning's overhead and (2) preserving the controller's
formal guarantees in the presence of the learned models.

% Combine learning and control through CALOREE.  Describe it.
We address these challenges with \SYSTEM{}, \footnote{\textbf{C}ontrol
  \textbf{A}nd \textbf{L}earning for \textbf{O}ptimal
  \textbf{R}esource \textbf{E}nergy \textbf{E}fficiency} a framework
for combining machine learning and control theory to build resource
management frameworks that can meet application performance
requirements with minimal energy.  The two key components of this
interface are (1) a data structure (called the performance hash table)
that allows the controller to access the learned model in constant
($O(1)$) time and (2) a confidence interval and estimated standard
deviation that provide probabilistic guarantees that the combined
learning and control system will converge to the desired performance.
The \SYSTEM{} interface not only combines control and learning
techniques, but allows the learning and control software to run on
physically separate devices.  This physical separation further
mitigates the cost of expensive learning techniques by running them on
a remote server while the constant time control systems run on the
device to be managed.  In addition to amortizing the cost of learning,
moving it to a remote server allows us to take advantage of learning
techniques that work across devices and applications; \ie{} those that
can learn similarities between different applications and systems.

In fact, the \SYSTEM{} interface is general enough to allow a wide
range of learning techniques to be paired with control systems.  So,
this approach not only provides an advantage over existing individual
learning and control techniques, it allows us to explore different
combinations of learning and control to find the best combination.

% Implement CALOREE.  Test against state of the art learning and
% self-tuning control systems.  We find that:
To demonstrate \SYSTEM{}, we implement it with learning running on an
x86 server and the control systems working to manage heterogeneous ARM
big.LITTLE devices.  We compare \SYSTEM{} to existing,
state-of-the-art learning (including polynomial
regression \cite{}, the Netflix algorithm \cite{}, and a hierarchical
Bayesian model \cite{}) and control (including
proportional-integral-derivative \cite{} and adaptive, or self-tuning
\cite{}) techniques.  Additionally, we compare to a naive combination
of learning and control that does not account for the confidence
interval and standard deviation of the learned model.  We set
performance goals (in terms of latency requirements) for a set of
benchmark applications and then measure the percentage of time the
requirements are violated as well as the energy for each application.
We test both \emph{single-app} environments, where on application runs
alone, and \emph{multi-app} environments where other applications
unpredictably enter the system and compete for resources.  We find
that \SYSTEM{} achieves the:
\begin{itemize} 
\item \textit{Most reliable performance:} 
 \begin{itemize} 
 \item In the \emph{single-app} case, the best prior learning and
   control techniques miss about 12\% of deadlines on average, the
   naive combination of learning and control misses 45\% of deadlines
   on average, but \SYSTEM{} misses only 5\% on average, reducing
   deadline misses by 50\% compared to prior approaches.
 \item In the \emph{multi-app} case, the best prior approach averages
   30\% deadline misses, the naive combination of learning and control
   averages 31\%, but \SYSTEM{} misses just 5.6\% of deadlines, a huge
   reduction compared to prior approaches.
\end{itemize}
  \item \textit{Best energy savings:} We compare to an \emph{oracle}
    with a perfect model of the application, system, and future
    events.
    \begin{itemize}
    \item In the \emph{single-app} case, the best prior approach
      averages 12\% more energy consumption than the oracle, the naive
      combination of control and learning consumes 11\% more, and
      \SYSTEM{} consumes 5\% more.  
    \item In the \emph{multi-app} case, the best prior approach
      averages 18\% more energy than the oracle, the naive combination
      of control and learning consumes 11\% more, and \SYSTEM{}
      consumes 7\% more.
    \end{itemize}
\end{itemize}

% Key contributions.
% Contributions, but I decided against bulleted llist for thsi paper
In summary, control theoretic approaches are well suited to manage
resources in dynamic environments and machine learning techniques can
produce accurate models of complex processors.  \emph{To the best of
  our knowledge, \SYSTEM{} is the first work to propose combining the
  two at runtime to ensure application performance goals without prior
  knowledge of the controlled application.}  We demonstrate this
contribution by implementing a resource manager that can meet
performance requirements on mobile/embedded processors with minimal
energy.  Additional contributions include formal analysis for convergence guarantees of the control system with noisy inputs, thus showing how
to incorporate learned variance into the control theoretic guarantees
and the empirical evaluation showing the combined control and learning
system outperforms individual, state-of-the-art control or learning
solutions.



\section{Background and Motivation}
\label{sec:example}

\PUNT{
We first demonstrate how learning addresses complexity, while even
advanced, adaptive controllers cannot.  We then show how control
theory handles system dynamics.  We conclude this section by
demonstrating what can go wrong when a controller incorporates a
learned model without proper tuning.
}
Many machine learning approaches estimate the most energy efficient
set of resources to allocate to an application.  These include
\emph{offline} techniques that build models using a training set and
then apply those models to new applications
\cite{Yi2003,LeeBrooks2006,CPR,ChenJohn2011,reddiHPCA2013,Paragon,PUPiL}.
Other approaches use \emph{online} techniques that construct models
dynamically as an application runs
\cite{Li2006,Flicker,ParallelismDial,Ponamarev,LeeBrooks}.  Finally,
\emph{hybrid} techniques combine offline modeling with online model
updates \cite{Zhang2012,packandcap,Winter2010,dubach2010,Koala,Cinder,
  wu2012inferred,LEO}.  

Machine learning is well suited to building
models of complicated systems like heterogeneous ARM big.LITTLE
systems.  These processors have two different core types including:
four big high-performance cores and four LITTLE energy efficient
cores.  The big cores support 19 clock speeds, while the LITTLE cores
support 14.

Control theory is a collection of mechanisms for maintaining required
behavior in dynamic systems \cite{Hellerstein2004a}. A subset of these
mechanisms---known as \emph{adaptive control} or \emph{self-tuning
  regulators}---are highly resilient to external effects that alter
behavior.  Adaptive controllers are thus especially useful in
webservers with fluctuating request rates
\cite{Horvarth,LuEtAl-2006a,SunDaiPan-2008a} and multimedia
applications with dynamically varying inputs
\cite{TCST,Agilos,grace2}.  Prior control solutions, however, are
always highly application dependent---with application-specific models
encoded in the controller's design---making a controller for video
playback unsuitable for controlling GPS navigation.  Some prior work
has generalized adaptive control design by exposing key parameters to
users so that the controller can be customized for a user's needs
\cite{ControlWare,POET}.  
This provides greater flexibility to the users but the controller can still fail to converge to the desired performance if there is mis-characterization in the relationship between resources and performance.  \PUNT{This limitation
  is why existing adaptive control approaches are built for specific
  classes of applications---specializing for the class accounts for
  common non-convexities of that class.}

\subsection{\emph{Learning} Complexity}
\PUNT{
\begin{figure*}
\centering
  \subfloat[]
  {
    \includegraphics[width=.25\textwidth]{figures/STREAM-contour.png}
    \label{fig:STREAM_contour}
  }
  \subfloat[]
  {
    \begin{tikzpicture}
\begin{centering}

\definecolor{s1}{RGB}{228, 26, 28}
\definecolor{s2}{RGB}{55, 126, 184}
\definecolor{s3}{RGB}{77, 175, 74}
\definecolor{s4}{RGB}{152, 78, 163}
\definecolor{s5}{RGB}{255, 127, 0}

\begin{groupplot}[
    group style={
        group name=plots,
        group size=1 by 1,
        xlabels at=edge bottom,
        xticklabels at=edge bottom,
        vertical sep=5pt
    },
height=3.5cm,
width=0.45\columnwidth,
xmajorgrids,
ymajorgrids,
grid style={dashed},
xmax=20,
yticklabel pos=left,
enlargelimits=false,
tick align = outside,
tick style={white},
xticklabel shift={-5pt},
yticklabel shift={-5pt},
ylabel shift={-2pt},
ylabel style={align=center},
unbounded coords=jump,
]

\nextgroupplot[ylabel={\scriptsize Latency (Normalized)}, % Performance
xlabel={\footnotesize Iteration},
ymin=0.8,
ymax=1.2,
ytick={0.8,0.9,1.0,1.1,1.2},
yticklabels={0.8,,1.0,,1.2},
legend entries={{\scriptsize $\mathsf{Latency Requirement}$},{\scriptsize $\mathsf{Learning}$},{\scriptsize $\mathsf{Adaptive Control}$},},
legend style={fill=none,draw=none,at={(0.5,1.65)},anchor=north,legend columns=1,line width=3pt},
]

\addplot[thick, solid, black] coordinates {(0,1) (20,1)};
\addplot[thick, solid, color=s4, mark=o, mark size=1pt] table[x index=0,y index=1,col sep=tab] {img/complexity-example-leo-latency.txt};
\addplot[thick, solid, color=s5, mark=square, mark size=1pt] table[x index=0,y index=1,col sep=tab] {img/complexity-example-poet-latency.txt};
\end{groupplot}
\end{centering}
\end{tikzpicture}
    \label{fig:STREAM_timeline}
  }
  \subfloat[]
  {
    \includegraphics[width=.25\textwidth]{figures/BODYTRACK-contour.png}
    \label{fig:BODYTRACK_contour}
  }
  \subfloat[]
  {
    \input{img/BODYTRACK-example-resized.tex}
    \label{fig:BODYTRACK_timeline}    
  }
  \caption{\small \bf (a) Performance for \texttt{STREAM} as a
    function of configuration.  (b) Managing \texttt{STREAM}'s
    performance: \emph{Learning} navigates the complicated
    configuration space, but \emph{control}'s simple model leads to
    oscillation.  (c) Performance for \texttt{bodytrack} as a function
    of configuration. (d) Managing \texttt{bodytrack}'s performance
    when another application starts: \emph{Adaptive Control} detects
    the change and adjusts, but \emph{Learning} has no mechanism to
    handle these dynamics. }
  \label{fig:learning-models1}
\end{figure*}
}

\begin{figure}
\centering
  \subfloat[]
  {
    \includegraphics[width=.25\textwidth]{figures/STREAM-contour.png}
    \label{fig:STREAM_contour}
  }
  \subfloat[]
  {
    \begin{tikzpicture}
\begin{centering}

\definecolor{s1}{RGB}{228, 26, 28}
\definecolor{s2}{RGB}{55, 126, 184}
\definecolor{s3}{RGB}{77, 175, 74}
\definecolor{s4}{RGB}{152, 78, 163}
\definecolor{s5}{RGB}{255, 127, 0}

\begin{groupplot}[
    group style={
        group name=plots,
        group size=1 by 1,
        xlabels at=edge bottom,
        xticklabels at=edge bottom,
        vertical sep=5pt
    },
height=3.5cm,
width=0.45\columnwidth,
xmajorgrids,
ymajorgrids,
grid style={dashed},
xmax=20,
yticklabel pos=left,
enlargelimits=false,
tick align = outside,
tick style={white},
xticklabel shift={-5pt},
yticklabel shift={-5pt},
ylabel shift={-2pt},
ylabel style={align=center},
unbounded coords=jump,
]

\nextgroupplot[ylabel={\scriptsize Latency (Normalized)}, % Performance
xlabel={\footnotesize Iteration},
ymin=0.8,
ymax=1.2,
ytick={0.8,0.9,1.0,1.1,1.2},
yticklabels={0.8,,1.0,,1.2},
legend entries={{\scriptsize $\mathsf{Latency Requirement}$},{\scriptsize $\mathsf{Learning}$},{\scriptsize $\mathsf{Adaptive Control}$},},
legend style={fill=none,draw=none,at={(0.5,1.65)},anchor=north,legend columns=1,line width=3pt},
]

\addplot[thick, solid, black] coordinates {(0,1) (20,1)};
\addplot[thick, solid, color=s4, mark=o, mark size=1pt] table[x index=0,y index=1,col sep=tab] {img/complexity-example-leo-latency.txt};
\addplot[thick, solid, color=s5, mark=square, mark size=1pt] table[x index=0,y index=1,col sep=tab] {img/complexity-example-poet-latency.txt};
\end{groupplot}
\end{centering}
\end{tikzpicture}
    \label{fig:STREAM_timeline}
  }
  \caption{\small \bf (a) Performance for \texttt{STREAM} as a
    function of configuration.  (b) Managing \texttt{STREAM}'s
    performance: \emph{Learning} navigates the complicated
    configuration space, but \emph{control} leads to oscillation.}
  \label{fig:learning-models1}
\end{figure}

We demonstrate how well learning complex performance vs resource relationship for \texttt{STREAM} as illustrated in \figref{fig:STREAM_contour} on our ARM big.LITTLE
processor, can improve resource management for the application.  This memory-bound application has
complicated behavior: due to memory pressure, the LITTLE cores' memory
hierarchy cannot deliver the required performance.  The big cores'
more powerful memory system delivers much greater performance, but the
peak occurs with 3 big cores.  Furthermore, at low clockspeeds, these
3 big cores cannot saturate the memory bandwidth, while at high
clockspeeds the performance drops as the processor overheats and
triggers thermal management.  For \texttt{STREAM}, the peak speed
occurs with 3 big cores at 1.2 GHz, and it is not efficient to spend
any time on the LITTLE cores.  \texttt{STREAM}, however, does not have
distinct phases, so once a resource allocator finds the most energy
efficient configuration, it simply needs to maintain it.  \TODO{We
  should change the levels in the contour so there is a clear maximum
  at 3 cores and a middle clock speed.}

The \emph{learning}
approach estimates the application's performance and power for all
configurations and then uses the lowest power configuration that
delivers the required performance.  The \emph{adaptive controller}
begins with a generic model of power/performance tradeoffs.  As the
controller runs, it continually measures performance and adjusts the
allocated resources and its own parameters in response to
the running application.  While many controllers use linear models,
this adaptive controller dynamically adjusts to non-linearities with a
series of linear approximations; however, the adaptive controller is
sensitive to local maxima, which can cause oscillations and prevent
the controller from meeting the requirement. \figref{fig:STREAM_timeline} shows 20 iterations of \texttt{STREAM}.  The
x-axis shows iteration number and the y-axis shows performance
normalized to the requirement.  The learning approach achieves the
goal, but the adaptive controller oscillates wildly around it,
sometimes not achieving the goal and sometimes delivering performance
that is too high (and wastes energy).  The oscillations occur because
the controller's adaptive mechanisms cannot handle the non-convexity
in STREAM's behavior, a known limitation of adaptive control systems
\cite{ControlWare,POET,ICSE2014}.  Hence, the \emph{learner}'s ability
to handle complex behavior is crucial for reliable performance in this
example.

% This result may be somewhat counter-intuitive.  The problem is that
% the controller cannot handle \texttt{STREAM}'s complexity.  One way to
% address this problem would be to build a custom controller just for
% this application, but that controller would not be useful for other
% applications.  In contrast, the learner can find the local maxima in
% the configuration space, and as this application has no phase changes
% or other dynamics, the one configuration that the learner finds is
% suitable for the entire application.

\subsection{\emph{Controlling} Dynamics}

\PUNT{
\begin{figure*}
\centering
  \subfloat[]
  {
    \includegraphics[width=.25\textwidth]{figs/kmeans.png}
    \label{fig:kmeans_contour}
  }
  \subfloat[]
  {
    \input{img/kmeans-example-resized.tex}
    \label{fig:kmeans_timeline}    
  }
  \caption{\small \bf (a) Performance for \texttt{kmeans} as a function of
    configuration.  (b) Managing \texttt{kmeans}' performance when
    another application starts: \emph{Control} detects the change and
    adapts, but \emph{learning} has no mechanism to handle these
    dynamics.}
  \label{fig:learning-models2}
\end{figure*}
} 


\begin{figure}
\centering
  \subfloat[]
  {
    \includegraphics[width=.25\textwidth]{figures/BODYTRACK-contour.png}
    \label{fig:BODYTRACK_contour}
  }
  \subfloat[]
  {
    \input{img/BODYTRACK-example-resized.tex}
    \label{fig:BODYTRACK_timeline}    
  }
  \caption{\small \bf (a) Performance for \texttt{bodytrack} as a
    function of configuration. (b) Managing \texttt{bodytrack}'s
    performance when another application starts: \emph{Adaptive
      Control} detects the change and adjusts, but \emph{Learning} has
    no mechanism to handle these dynamics. }
  \label{fig:control}
\end{figure}


We now consider a dynamic environment.  We begin with
\texttt{bodytrack} as the only application running on the system.
Halfway through its execution, we launch a second
application---\texttt{STREAM}---on a single big core, dynamically
altering resource availability. \texttt{bodytrack}'s behavior is
shown in \figref{fig:BODYTRACK_contour}; it achieves its best performance on 4
big cores at the maximum clockspeed. Additionally the 4 LITTLE core 
can offer high performance at much lower energy consumption.  
The challenge of allocating resources for this application is determining how to split time between the LITTLE cores and big cores to conserve energy while still meeting the
performance requirements.  

\figref{fig:BODYTRACK_timeline} shows the results of this experiment.  The
vertical dashed line---at frame 99---represents when the second
application begins.  The figure clearly shows the benefits of the
adaptive control system in this dynamic scenario.  When the second
application starts, the controller detects the change in
\texttt{bodytrack}'s performance and then changes resource allocation
(increasing clockspeed and moving bodytrack from 4 to 3 big cores).
The controller is not aware of the second application, rather it
observes \texttt{bodytrack}'s performance drop at frame 99 and
immediately restores performance in the next frame. The learning
system however, does not have any inherent mechanism to measure the
change or adapt to the altered performance.  While we could
theoretically add feedback to the learner and re-estimate the
configuration space whenever the environment changes, doing so is
impractical due to high overhead\cite{pargon,LEO}.


\subsection{Challenges of Parameter-free Control}
\PUNT{
The learner illustrated above is parameter-free; \ie{} it has no
user-specifed parameters, but estimates all values from measurements.}
The controller as we mentioned in earlier section requires some user-specified parameters.
\SYSTEM{}'s goal is to provide a parameter-free approach that provides
the same formal guarantees as a controller with user-specified
parameters.

Perhaps the most essential parameter for a control system is the
\emph{pole} of its characteristic equation.  Control engineers tune
the pole to trade control response time for noise sensitivity.
Following standard control design, the pole is tuned based on a known
model.  This model may be an abstraction of the full system, but it is
considered to be ``ground truth'' \cite{Hellerstein2004a}. In a
computing system ground truth means that all possible configurations
the controller might select have been directly measured.  \SYSTEM{},
however, must tune the pole based on an estimated model, which may
have noise and/or errors.


\begin{figure}
\centering
\input{img/BODYTRACK-example2-resized.tex}
\caption{\small \bf Comparison of existing adaptive control and a
  naive combination of control and learning.}
\label{fig:not-simple}
\end{figure}

To demonstrate the importance of the pole in the presence of a learned
model, we again show control of \texttt{bodytrack}, this time using
the adaptive control system from the previous subsection with a model
produced by the learner from the first subsection.  We compare the
control system with a carefully tuned pole to the same system using
the default pole provided by the controller developers---recall that
the pole is typically a user-specified parameter.  

\figref{fig:not-simple} shows the results.  The system with the
carefully tuned pole converges because the pole accounts for the error
in the learned model. The pole parameterizes the inertia of the system as how fast should the system change in presence of dynamic environment. If the estimated model is too noisy, the controller should trust the model less and move less frequently even though that might results in slower convergence. On the other hand, the system with the default pole, however
oscillates around the performance target, resulting in a number of
missed deadlines.  In addition, the frames that exceed the desired
performance waste energy because they spend much more time on the big,
inefficient cores than necessary.

The next section describes \SYSTEM{}'s approach to combining learning
and control that abstracts the controller's key parameters, so they
can be learned while the system runs.  Rather than have a user
carefully tune the pole, \SYSTEM{} incorporates the learner's
confidence interval and estimated variance to compute a pole that
provides probabilistic convergence guarantees.  



\section{Combining Learning and Control}
\label{sec:framework}

%Overview. Learning background.  Control background.
\SYSTEM{} combines learning with control to tackle both complexity and
dynamics.  \figref{fig:overview} shows a detailed overview of
\SYSTEM{}.  A mobile system runs an application and some small number
of measurements are taken and sent to a server. The server uses a
hierarchical Bayesian model to combine these measurements with ones
taken on other devices and with measurements of other applications.
Using this large volume of data, the HBM produces an
application-specific model of performance and power for all resource
configurations.  The model is sent to a lightweight control system
(LCS) that manages the device, adjusting resource usage to meet
application performance requirements with minimal energy.  The learned
models are stored a performance hash table (PHT), which is the
interface between the remote HBM and the LCS that manages the device.
The expensive process of constructing the PHT is done by the server.
Once built, the PHT allows the LCS to apply the learned models in
constant ($O(1)$) time.  This section provides a brief overview of the
relevant learning and control techniques used in \SYSTEM{}, and then
describes the interface that combines them.

\subsection{The Hierarchical Bayesian Model}
\label{sec:framework:HBM}

%\PUNT{

Machine learning models are predictive in nature.  Given some
observations of a system, they create a model to predict future
behavior in unobserved settings. \SYSTEM{} uses a hierarchical
Bayesian model (HBM), based on LEO \cite{LEO}, to turn observations of
applications' performance and power given some resource allocation
into predictions of the performance and power of other, unobserved resource allocations.  The HBM provides a statistically
sound framework for learning across applications and devices.

The HBM is non-parametric in terms of resource configurations. Instead
of modeling a configuration's performance/power as a function of 
the exact number of cores or clockspeed, it captures correlations between
configurations.  Hence, the HBM is well suited to learning complicated
configuration spaces, like those in
\figsref{fig:lavamd_contour}{fig:kmeans_contour}.  The non-smoothness
and application-specificity of these configuration spaces means that
parametric models -- based on clockspeed/cores -- will not produce accurate predictions.  In contrast, the HBM requires fewer assumptions about the
relationship between configurations and performance/power and is thus
more robust for this modeling problem \cite{LEO}.

\figrref{fig:online}{fig:HBM} compare \SYSTEM{} to \emph{online} and
\emph{offline} learning models.  The online model only uses
observations of the current application and it must be parametric in
the configuration space, since it has no other information. The online
model is thus highly dependent on the model parameterization; \eg{} if
the model is specified to be linear in frequency, but an application
is memory-bound, the online model will over-allocate resources.  The
offline model only uses information from previously observed
applications and lacks knowledge of the current application.  This
general model will capture trends -- \eg{} when most applications
should transition from LITTLE to big cores -- but it will miss key
inflection points for applications that deviate from the general
trend.

\begin{figure}

  \subfloat[]
  {
    %\includegraphics[width=.33\textwidth]{figures/Online.pdf}
    \includegraphics[width=.33\columnwidth]{figures/Online.pdf}

    \label{fig:online}
  }
  \subfloat[]
  {
    %\includegraphics[width=.33\textwidth]{figures/Offline.pdf}
    \includegraphics[width=.33\columnwidth]{figures/Offline.pdf}

    \label{fig:offline}
  }
  \subfloat[]
  {
    %\includegraphics[width=.33\textwidth]{figures/HBM.pdf}
    \includegraphics[width=.33\columnwidth]{figures/HBM.pdf}

    \label{fig:HBM}
  }
  \caption{ Comparison of online, offline, and hierarchical Bayesian
    models.  Arrows represent dependences, circles are random
    variables, white circles are hidden and must be learned, solid
    circles are fully observed data, and shaded circles are partially
    observed.}
\label{fig:learning-models}
\end{figure}


The HBM is a combination of the \emph{online} and \emph{offline})
approaches, incorporating both (1) observations of the current
application and (2) observations of other applications to learn
correlations between different configurations.  The HBM's correlation
matrix captures these relations and scales with the number of
configurations, making the model non-parametric.  In practice, such a
non-parametric model is much more flexible; \eg{} it can learn a
linear relationship between frequency and performance for a
compute-bound application and that there is no relationship for a
memory-bound one.  Unlike the pure online approach, in the HBM all
applications' models are conditionally dependent on a hidden mean and co-variance
matrix.  Rather than over-generalizing (like the offline model) or
over-specifying (like the online model) the HBM implicitly uses a pool
of similar applications to produce new models. Additionally, the HBM's
accuracy increases as more applications are observed because more
behaviors are represented in the pool of prior knowledge.  Of
course, the HBM's computational cost -- which is linear in the number
of applications -- also increases with increasing applications, but
this is why we offload the learning to a remote server.

\subsection{The Lightweight Control System}

Control theory provides a discipline for tuning system parameters to
ensure operating goals are met in a dynamic environment.
We use a controller to adjust system resource usage to see that the
performance goals (corresponding to a quality-of-service or real-time
constraint) are met over time.  The difficulty is that classical
control formulations integrate the application-dependent relationship
between performance and the controlled resource directly in the
control formulation. This makes the control models too specific to a
particular class of applications.  Thus, we face the problem of
implementing a general control system that is applicable to a number
of applications, and where the models relating resources to
performance are not known ahead of time, but are provided by the HBM
at runtime.

\SYSTEM{} addresses this problem using the classic computer science
approach of adding a layer of indirection, as illustrated in
\figref{ig:framework:lcs}.  Instead of directly controlling resources
using an application-dependent model, \SYSTEM{} controls
\emph{speedup} and a separate module uses the learned models to
optimize energy while respecting this speedup constraint.  Similar
models have been used to build generalized controllers where users are
responsible for supplying the models \cite{ControlWare,POET}.  Our
goal is to eliminate this user burden and have a remote HBM supply the
model\PUNT{, which are not just more accurate but become more
  intelligent with time as it acquires more data}.

\begin{figure}
\includegraphics[width=\columnwidth]{figures/LCS.pdf}
\caption{Light-weight control system (LCS) }
  \label{fig:framework:lcs}
\end{figure}


\subsubsection{Controlling Speedup}
We write a simple difference model relating speedup to performance:
\begin{equation}
  perf(t) = m \cdot speedup(t-1) + \delta \label{eqn:speedup}
\end{equation}
where $m$ is the \emph{max speed} of the application, here defined as
the speed when all resources are available.  While $m$ is application
specific, it is easy to measure online, by simply allocating all
resources. Such a configuration should not violate any performance
constraints (although it is unlikely to be energy efficient) so it is
safe to take this measurement without risk of violating performance
constraints.

With this model, the control law is simply:
\begin{eqnarray}
  error(t) &=& goal - perf(t) \label{eqn:speedup-error} \\
  speedup(t) &=& speedup(t-1) - \frac{error(t)}{m}
  \label{eqn:speedup-control}
\end{eqnarray}
which states that the speedup to apply at time $t$ is a function of
the previous speedup, the error at time $t$ and the max speed $m$.
This is a very simple \emph{deadbeat} controller that provides all the
standard control theoretic formal guarantees \cite{Hellerstein2004a}.
Using the above definition of max speed, most speedups will be less
than one.  In addition to making max speed easier to measure, this
definition bounds the HBM's output, making for more robust learning.

\subsubsection{Optimizing Speedup}
\SYSTEM{} must turn the speedup produced by \eqnref{speedup-control}
into a resource allocation.  The primary challenge here is that
the HBM produces a non-linear
function mapping the discrete system resource allocations into speedup and powerup,
while \eqnref{speedup-control} is a continuous linear function.
\SYSTEM{} bridges this divide by assigning time to resource
allocations such that the average speedup over a control interval is
that produced by \eqnref{speedup-control}.

We call an assignment of time to resources a \emph{schedule}. We call
a combination of settings for each resource a \emph{configuration}.
For example, using two big cores at 1.5 \GHz and putting all the
little cores to sleep is one configuration.  There are typically many
schedules that meet a required speedup.  To extend battery life,
\SYSTEM{} finds a minimal energy schedule. Given a time interval $T$,
a workload $W$ to complete in that interval, and a set of $C$
configurations, we formalize this problem as:
\begin{eqnarray}
  \argmin_{\tau} && \sum_{c=0}^{C-1} \tau_c \cdot p_c \label{eqn:power} \\
  \text{such that} %&& \nonumber\\
  && \sum_{c=0}^{C-1} \tau_c \cdot s_c \cdot b =  W \label{eqn:work} \\
  && \sum_{c=0}^{C-1} \tau_c =  T \label{eqn:deadline} \\
  && 0 \le \tau_c \le \tau, \qquad \forall c \in \{0,\ldots,C-1\} \label{eqn:time}
\end{eqnarray}
where $p_c$ and $s_c$ are the estimated powerup and speedup of
configuration $c$ and $\tau_c$ is the amount of time to spend in
configuration $c$.  \eqnref{power} simply states that the objective is
to minimize energy (power times time).  \eqnref{work} states that the
work must be done, while \eqnref{deadline} requires the work to be
done on time.  \eqnref{time} simply avoids negative time.


\subsection{The Performance Hash Table}
While most linear programming problems would be inefficient to solve
repeatedly on a mobile device, the one in \eqnrref{power}{time} has a
constant time, $O(1)$, solution.  Kim et al. analyzed solutions to the
problem of minimizing energy while meeting a performance constraint
\cite{kim-cpsna}.  They observed that there must be an optimal
solution with the following properties:
\begin{itemize}
\item At most two of $\tau_c$ are non-zero, meaning that at most two
  resource configurations will be used in any time interval.  This
  property both drastically limits the search space and puts a limit
  on the overhead of switching configurations, since it is never
  profitable to switch more than twice in an interval.
\item If one plots the configurations in the power and performance
  tradeoff space (so that performance is on the x-axis and power is on
  the y-axis) the two configurations with non-zero $\tau_c$ lie on the
  lower convex hull of that space.
\end{itemize}
We use these two facts to construct a constant time algorithm for
finding the optimal solution to \eqnrref{power}{time} online.  The
intuition behind our solution is illustrated in the upper half of
\figref{fig:pht}.  This figure shows a hypothetical example of the power
and performance tradeoffs the HBM might learn for some application and
device.  Each point represents a configuration and each configuration
is charted with normalized performance on the x-axis and normalized
power on the y-axis.  For any feasible performance requirement, there
is a minimal energy schedule that uses no more than two of the
configurations on the lower convex hull, as any points that lie above
the lower convex hull require more power for equivalent performance
\cite{kim-cpsna}.  Therefore, the HBM estimates the power and
performance of all configurations, then finds the lower convex hull,
and sends that to the LCS.  This lower convex hull is the interface
between the HBM and the LCS, and the key enabler of \SYSTEM{}'s
combination of learning and control.

% \begin{figure}
% \includegraphics[width=\columnwidth]{figures/TradeoffExample.png}
% \caption{Example of plotting configurations in the power versus
%   performance space.}
%   \label{fig:convexhull}
% \end{figure}



\begin{figure}
  \includegraphics[width=\columnwidth]{figures/performance-hash-table.pdf}
  \caption{The Performance Hash Table and its relationship to the
    power/performance tradeoff space.  The HBM encodes the
    configuration as the lower convex hull of points in the
    performance/power tradeoff space and stores those in a table
    indexed by speedup.}
  \label{fig:pht}
\end{figure}

The HBM stores the lower convex hull in a \emph{performance hash
  table} (PHT).  The PHT and its relationship to the lower convex hull
is illustrated in \figref{fig:pht}.  It consists of two arrays, the
first is an array of pointers into the second array, which stores the
configurations on the lower convex hull sorted by speedup.  Recall
that speedups are computed relative to the maximum speed.  We
therefore know the largest speedup is 1, so we need only concern
ourselves with speedups less than 1.  The first table of pointers has
a \emph{resolution} indicating how many decimal points of precision it
captures.  The example in \figref{fig:pht} has a resolution of $0.1$.
Each pointer in the bottom table points to the configuration in the
second array that has the largest speedup less than or equal to the
index.

To use the table, the optimizer receives a speedup $s(t)$ from the
controller.  It needs to convert this into two configurations referred
to as $hi$ and $lo$.  To find the $hi$ configuration, the optimizer
clamps the desired speedup to the largest index lower than $s(t)$ and
then walks forward until it finds the first configuration with a
speedup higher than $s(t)$.  To find the $lo$ configuration, the
optimizer clamps the desired speedup to the smallest index higher than
$s(t)$ and then walks backwards until it finds the configuration with
the largest speedup less than $s(t)$.

\PUNT{
For example, consider the PHT in \figref{fig:pht} and an optimizer trying
to meet a speedup $s(t) = .65$.  To find $hi$, the optimizer indexes
at .6 and walks up to find $c=2$ with $s_c=.8$, setting $hi = 2$.  To
find $lo$, the optimizer indexes the table at .7 and walks backward to
find $c=1$ with $s_c=.2$, setting $lo = 1$.
}

Finally, the optimizer sets $\tau_{hi}$ and $\tau_{lo}$ by solving the
following set of equations:
\begin{eqnarray}
  \tau &=& \tau_{hi} + \tau_{lo}    \label{eqn:s1} \\
  s(t) &=& s_{hi} \cdot \tau_{hi} + s_{lo} \cdot \tau_{lo} \label{eqn:s2} 
\end{eqnarray}
In these equations, $s(t)$ is the speedup requested by the optimizer
and $s_c$ are speedups estimated by the learner.

By solving \eqnsref{s1}{s2}, the optimizer has turned the controller's
requested speedup into a resource allocation schedule using the models
provided by the HBM.  Provided that the resolution is large enough to
get a good spread of configurations to indices, the optimizer will
index the configuration one entry (on average) from where it needs to
be.  Thus, the entire optimization process runs in constant time --
assuming that the learner is responsible for building the PHT once
before passing it on to the optimizer.  This efficiency comes at a
cost of memory usage, as many of the entries in the speedup index
table will point to redundant locations in the configuration array.
This tradeoff is reasonable in practice as the code that runs on the
mobile device must be fast or we risk wasting energy while trying to
save energy.  In practice, we recommend a table of size 100 which
provides a sufficient resolution and is not too wasteful of space.

\PUNT{
\subsection{Putting It All Together}
We briefly summarize our proposal for combining learning and control.
Our approach consists of a number of independent mobile devices each
running a lightweight controller.  Each device makes a small number of
local observations of an application it runs and sends those to the
server.  The server integrates those observations in its HBM to
produce customized models for each device.  These models are sent back
to the individual devices where they are used to meet performance
requirements with minimal energy by turning a speedup signal into an
optimal schedule of configurations.

\TODO{The control theoretic guarantees are still valid provided that
  the error in the learned speedup is not too great.  Is it worth
  analyzing those formal guarantees?}

}

%control slowdown

%map slowdown into configurations

% solve optimization problem

% slowdown table




\PUNT{
\section{Combining Learning and Control}
\label{sec:framework}

%Overview. Learning background.  Control background.
\SYSTEM{} combines learning with control to tackle both complexity and
dynamics.  \figref{fig:overview} shows a detailed overview of
\SYSTEM{}.  A mobile system runs an application and some small number
of measurements are taken and sent to a server. The server uses a
hierarchical Bayesian model to combine these measurements with ones
taken on other devices and with measurements of other applications.
Using this large volume of data, the HBM produces an
application-specific model of performance and power for all resource
configurations.  The model is sent to a lightweight control system
(LCS) that manages the device's resources to meet application
performance requirements with minimal energy.  The learned models are
stored in a performance hash table (PHT), a data structure which
serves as the interface between the remote HBM and the LCS.  The
expensive process of constructing the PHT is done by the server, and
once built, the PHT allows the LCS to apply the learned models in
constant ($O(1)$) time.  This section provides a brief overview of the
relevant learning and control techniques used in \SYSTEM{}, and then
describes the interface that combines them.

\subsection{The Hierarchical Bayesian Model}
\label{sec:framework:HBM}

%\PUNT{

Machine learning models are predictive in nature.  Given some
observations of a system, they create a model to predict future
behavior in unobserved settings. \SYSTEM{} uses a hierarchical
Bayesian model (HBM), based on LEO \cite{LEO}, to turn observations of
applications' performance and power given some resource allocation
into predictions about how other, unobserved resource allocations will
alter that performance and power.  The HBM provides a statistically
sound framework for learning across applications and devices.

The HBM is non-parametric in terms of resource configurations. Instead
of modeling performance of a configuration as a function of exact
number of cores or clockspeed, it captures correlations between
configurations.  Hence, the HBM is well suited to learning complicated
configuration spaces, like those in
\figsref{fig:lavamd_contour}{fig:kmeans_contour}.  The non-smoothness
and application-specificity of these configuration spaces means that
parametric models -- based on clockspeed/cores -- will not perform
well.  In contrast, the HBM requires fewer assumptions about the
relationship between configurations and performance/power and is thus
more robust for this modeling problem \cite{LEO}.

\figrref{fig:online}{fig:HBM} compare \SYSTEM{} to \emph{online} and
\emph{offline} learning models.  The online model only uses
observations of the current application and it must be parametric in
the configuration space, since it has no other information. The online
model is thus highly dependent on the model parameterization; \eg{} if
the model is specified to be linear in frequency, but an application
is memory-bound, the online model will over-allocate resources.  The
offline model only uses information from previously observed
applications and lacks knowledge of the current application.  This
general model will capture trends -- \eg{} when most applications
should transition from LITTLE to big cores -- but it will miss key
inflection points for applications that deviate from the general
trend.

\begin{figure}

  \subfloat[]
  {
    %\includegraphics[width=.33\textwidth]{figures/Online.pdf}
    \includegraphics[width=.33\columnwidth]{figures/Online.pdf}

    \label{fig:online}
  }
  \subfloat[]
  {
    %\includegraphics[width=.33\textwidth]{figures/Offline.pdf}
    \includegraphics[width=.33\columnwidth]{figures/Offline.pdf}

    \label{fig:offline}
  }
  \subfloat[]
  {
    %\includegraphics[width=.33\textwidth]{figures/HBM.pdf}
    \includegraphics[width=.33\columnwidth]{figures/HBM.pdf}

    \label{fig:HBM}
  }
  \caption{ Comparison of online, offline, and hierarchical Bayesian
    models.  Arrows represent dependences, circles are random
    variables, white circles are hidden and must be learned, solid
    circles are fully observed data, and shaded circles are partially
    observed.}
\label{fig:learning-models}
\end{figure}


The HBM is a combination of the \emph{online} and \emph{offline}
approaches. The HBM incorporates both (1) observations of the current
application and (2) the previously observed applications to learn
correlations between different configurations.  The HBM's correlation
matrix captures these relations and scales with the number of
configurations, making the model non-parametric.  In practice, such a
non-parametric model is much more flexible; \eg{} it can learn a
linear relationship between frequency and performance for a
compute-bound application and that there is no relationship for a
memory-bound one.  Unlike the pure online approach, in the HBM all
models are conditionally dependent on a hidden mean and co-variance
matrix.  Rather than over-generalizing (like the offline model) or
over-specifying (like the online model) the HBM implicitly uses a pool
of similar applications to produce new models. Additionally, the HBM's
accuracy increases as more applications are observed because more
types of behavior are represented in the pool of prior knowledge.  Of
course, the HBM's computational cost -- which is linear in the number
of applications -- also increases with increasing applications, but
this is why we offload the learning to a remote server.

\subsection{The Lightweight Control System}

Control theory provides a discipline for tuning system parameters to
ensure that a system meets the desired goals in a dynamic environment.
We use a controller to adjust system resource usage to see that the
performance goals (corresponding to a quality-of-service or real-time
constraint) are met over time.  The difficulty is that classical
control formulations integrate the application-dependent relationship
between performance and the controlled resource directly in the
control formulation. This makes the control models too specific to a
particular class of applications.  Thus, we face the problem of
implementing a general control system that is applicable to a number
of applications, and where the models relating resources to
performance are not known ahead of time, but are provided by the HBM
at runtime.

\SYSTEM{} addresses this problem using the classic computer science
approach of adding a layer of indirection, as illustrated in
\figref{fig:framework:lcs}.  Instead of directly controlling resources
using an application-dependent model, \SYSTEM{} controls
\emph{speedup} and a separate module uses the learned models to
optimize energy while respecting this speedup constraint.  Similar
models have been used to build generalized controllers where users supply the models \cite{ControlWare,POET}.  Our
goal is to eliminate this user burden and have a remote HBM supply the
model\PUNT{, which are not just more accurate but become more
  intelligent with time as it acquires more data}.

\begin{figure}
\includegraphics[width=\columnwidth]{figures/LCS.pdf}
\caption{Light-weight control system (LCS) }
  \label{fig:framework:lcs}
\end{figure}


\subsubsection{Controlling Speedup}
We write a simple difference model relating speedup to performance:
\begin{equation}
  perf(t) = m \cdot speedup(t-1) + \delta \label{eqn:speedup}
\end{equation}
where $m$ is the \emph{max speed} of the application, here defined as
the speed when all resources are available.  While $m$ is application
specific, it is easy to measure online, by simply allocating all
resources. Such a configuration should not violate any performance
constraints (although it is unlikely to be energy efficient) so it is
safe to take this measurement without risk of violating performance
constraints.

With this model, the control law is simply:
\begin{eqnarray}
  error(t) &=& goal - perf(t) \label{eqn:speedup-error} \\
  speedup(t) &=& speedup(t-1) - \frac{error(t)}{m}
  \label{eqn:speedup-control}
\end{eqnarray}
which states that the speedup to apply at time $t$ is a function of
the previous speedup, the error at time $t$ and the max speed $m$.
This is a very simple \emph{deadbeat} controller that provides all the
standard control theoretic formal guarantees \cite{Hellerstein2004a}.
Using the above definition of max speed, most speedups will be less
than one.  In addition to making max speed easier to measure, this
definition bounds the HBM's output, making for more robust learning.

\subsubsection{Optimizing Speedup}
\SYSTEM{} must turn the speedup produced by \eqnref{speedup-control}
into a resource allocation.  The primary challenge here is that since
the system resources are discrete, the HBM produces a non-linear
function mapping the resource allocations into speedup and powerup,
while \eqnref{speedup-control} is a continuous linear function.
\SYSTEM{} bridges this divide by assigning time to resource
allocations such that the average speedup over a control interval is
that produced by \eqnref{speedup-control}.

We call an assignment of time to resources a \emph{schedule}. We call
a combination of settings for each resource a \emph{configuration}.
For example, using two big cores at 1.5 \GHz and putting all the
little cores to sleep is one configuration.  There are typically many
schedules that meet a required speedup.  To extend battery life,
\SYSTEM{} finds a minimal energy schedule. Given a time interval $T$,
a workload $W$ to complete in that interval, and a set of $C$
configurations, we formalize this problem as:
\begin{eqnarray}
  \minimize_{\tau \in \mathbb{R}^C} && \sum_{c=0}^{C-1} \tau_c \cdot p_c \label{eqn:power} \\
  \text{such that} %&& \nonumber\\
  && \sum_{c=0}^{C-1} \tau_c \cdot s_c \cdot b =  W \label{eqn:work} \\
  && \sum_{c=0}^{C-1} \tau_c =  T \label{eqn:deadline} \\
  && 0 \le \tau_c \le T, \qquad \forall c \in \{0,\ldots,C-1\} \label{eqn:time}
\end{eqnarray}
where $p_c$ and $s_c$ are the estimated powerup and speedup of
configuration $c$ and $\tau_c$ is the amount of time to spend in
configuration $c$.  \eqnref{power} simply states that the objective is
to minimize energy (power times time).  \eqnref{work} states that the
work must be done, while \eqnref{deadline} requires the work to be
done on time.  \eqnref{time} simply avoids negative time.


\subsection{The Performance Hash Table}
While most linear programming problems would be inefficient to solve
repeatedly on a mobile device, the one in \eqnrref{power}{time} has a
constant time ($O(1)$ solution.  Kim et al. analyzed solutions to the
problem of minimizing energy while meeting a performance constraint
\cite{kim-cpsna}.  They observed that there must be an optimal
solution with the following properties:
\begin{itemize}
\item At most two of $\tau_c$ are non-zero, meaning that at most two
  resource configurations will be used in any time interval.  This
  property both drastically limits the search space and puts a limit
  on the overhead of switching configurations, since it is never
  profitable to switch more than twice in an interval.
\item If one plots the configurations in the power and performance
  tradeoff space (so that performance is on the x-axis and power is on
  the y-axis) the two configurations with non-zero $\tau_c$ lie on the
  lower convex hull of the power performance tradeoff space.
\end{itemize}
We use these two facts to construct a constant time algorithm for
finding the optimal solution to \eqnrref{power}{time} online.  The
intuition behind our solution is illustrated in the upper half of
\figref{fig:pht}.  This figure shows a hypothetical example of the power
and performance tradeoffs the HBM might learn for some application and
device.  Each point represents a configuration and each configuration
is charted with normalized performance on the x-axis and normalized
power on the y-axis.  For any feasible performance requirement, there
is a minimal energy schedule that uses no more than two of the
configurations on the lower convex hull as any points that lie above
the lower convex hull require more power for equivalent performance
\cite{kim-cpsna}.  Therefore, the HBM estimates the power and
performance of all configurations, then finds the lower convex hull,
and sends that to the LCS.  This lower convex hull is the interface
between the HBM and the LCS, and the key enabler of \SYSTEM{}'s
combination of learning and control.

% \begin{figure}
% \includegraphics[width=\columnwidth]{figures/TradeoffExample.png}
% \caption{Example of plotting configurations in the power versus
%   performance space.}
%   \label{fig:convexhull}
% \end{figure}



\begin{figure}
  \includegraphics[width=\columnwidth]{figures/performance-hash-table.pdf}
  \caption{The Performance Hash Table and its relationship to the
    power/performance tradeoff space.  The HBM encodes the
    configuration as the lower convex hull of points in the
    performance/power tradeoff space and stores those in a table
    indexed by speedup.}
  \label{fig:pht}
\end{figure}

The HBM stores the lower convex hull in a \emph{performance hash
  table} (PHT).  The PHT and its relationship to the lower convex hull
is illustrated in \figref{fig:pht}.  It consists of two arrays, the
first is an array of pointers into the second array, which stores the
configurations on the lower convex hull sorted by speedup.  Recall
that speedups are computed relative to the maximum speed.  We
therefore know the largest speedup is 1, so we need only concern
ourselves with speedups less than 1.  The first table of pointers has
a \emph{resolution} indicating how many decimal points of precision it
captures.  The example in \figref{fig:pht} has a resolution of $0.1$.
Each pointer in the bottom table points to the configuration in the
second array that has the largest speedup less than or equal to the
index.

To use the table, the optimizer receives a speedup $s(t)$ from the
controller.  It needs to convert this into two configurations referred
to as $hi$ and $lo$.  To find the $hi$ configuration, the optimizer
clamps the desired speedup to the largest index lower than $s(t)$ and
then walks forward until it finds the first configuration with a
speedup higher than $s(t)$.  To find the $lo$ configuration, the
optimizer clamps the desired speedup to the smallest index higher than
$s(t)$ and then walks backwards until it finds the configuration with
the largest speedup less than $s(t)$.

\PUNT{
For example, consider the PHT in \figref{fig:pht} and an optimizer trying
to meet a speedup $s(t) = .65$.  To find $hi$, the optimizer indexes
at .6 and walks up to find $c=2$ with $s_c=.8$, setting $hi = 2$.  To
find $lo$, the optimizer indexes the table at .7 and walks backward to
find $c=1$ with $s_c=.2$, setting $lo = 1$.
}

Finally, the optimizer sets $\tau_{hi}$ and $\tau_{lo}$ by solving the
following set of equations:
\begin{eqnarray}
  \tau &=& \tau_{hi} + \tau_{lo}    \label{eqn:s1} \\
  s(t) &=& s_{hi} \cdot \tau_{hi} + s_{lo} \cdot \tau_{lo} \label{eqn:s2} 
\end{eqnarray}
In these equations, $s(t)$ is the speedup requested by the optimizer
and $s_c$ are speedups estimated by the learner.

By solving \eqnsref{s1}{s2}, the optimizer has turned the controller's
requested speedup into a resource allocation schedule using the models
provided by the HBM.  Provided that the resolution is large enough to
get a good spread of configurations to indices, the optimizer will
index the configuration one entry (on average) from where it needs to
be.  Thus, the entire optimization process runs in constant time --
assuming that the learner is responsible for building the PHT once
before passing it on to the optimizer.  This efficiency comes at a
cost of memory usage, as many of the entries in the speedup index
table will point to redundant locations in the configuration array.
This tradeoff is reasonable in practice as the code that runs on the
mobile device must be fast or we risk wasting energy while trying to
save energy.  In practice, we recommend a table of size 100 which
provides a sufficient resolution and is not too wasteful of space.

\PUNT{
\subsection{Putting It All Together}
We briefly summarize our proposal for combining learning and control.
Our approach consists of a number of independent mobile devices each
running a lightweight controller.  Each device makes a small number of
local observations of an application it runs and sends those to the
server.  The server integrates those observations in its HBM to
produce customized models for each device.  These models are sent back
to the individual devices where they are used to meet performance
requirements with minimal energy by turning a speedup signal into an
optimal schedule of configurations.

\TODO{The control theoretic guarantees are still valid provided that
  the error in the learned speedup is not too great.  Is it worth
  analyzing those formal guarantees?}

}

%control slowdown

%map slowdown into configurations

% solve optimization problem

% slowdown table

}
\section{Experimental evaluations}


\begin{figure*}[t]
  \input{img/accuracy-performance.tex}
   \vskip -1em
  \caption{Accuracy performance}
  \label{fig:accuracy}
\end{figure*}


\begin{figure*}[t]
  \input{img/accuracy-power.tex}
   \vskip -1em
  \caption{Accuracy power}
  \label{fig:accuracy}
\end{figure*}

\begin{figure*}[t]
  \input{img/single-app-performance.tex}
   \vskip -1em
  \caption{Single Applications MAPE}
  \label{fig:single-perf}
\end{figure*}

\begin{figure*}[t]
  \input{img/single-app-energy.tex}
   \vskip -1em
  \caption{Single Applications energy}
  \label{fig:single-energy}
\end{figure*}


\begin{figure*}[t]
  \input{img/multi-app-performance.tex}
   \vskip -1em
  \caption{Multi Applications MAPE}
  \label{fig:multi-perf}
\end{figure*}

\begin{figure*}[t]
  \input{img/multi-app-energy.tex}
   \vskip -1em
  \caption{Multi Applications Energy}
  \label{fig:multi-energy}
\end{figure*}

%\PUNT{
\begin{figure*}[t]
  \begin{tikzpicture}
\begin{centering}

\definecolor{s1}{RGB}{228, 26, 28}
\definecolor{s2}{RGB}{55, 126, 184}
\definecolor{s3}{RGB}{77, 175, 74}
\definecolor{s4}{RGB}{152, 78, 163}
\definecolor{s5}{RGB}{255, 127, 0}

\begin{groupplot}[
    group style={
        group name=plots,
        group size=1 by 2,
        xlabels at=edge bottom,
        xticklabels at=edge bottom,
        vertical sep=5pt
    },
height=3.5cm,
width=0.95\columnwidth,
xmajorgrids,
ymajorgrids,
grid style={dashed},
xmin=0,
xmax=100,
yticklabel pos=left,
enlargelimits=false,
tick align = outside,
tick style={white},
xticklabel shift={-5pt},
yticklabel shift={-5pt},
ylabel shift={-2pt},
ylabel style={align=center},
unbounded coords=jump,
]

\nextgroupplot[ylabel={\footnotesize Accuracy\\(Performance)}, % Performance
xtick={0,20,40,60,80,100},
ytick={0.0,0.3,0.6,0.9,1.0},
yticklabels={,0.3,0.6,0.9,1.0},
yticklabel style={font=\footnotesize},
ymin=0,
ymax=1,
legend entries={{$\mathsf{LEO}$},{$\mathsf{Online}$}},
legend style={draw=none,at={(0.5,1.4)},anchor=north,legend columns=4,line width=5pt},
]
%\addplot[thick, solid, color=s3] table[x index=0,y index=1,col sep=tab] {img/old/x264-phases-clover-dvfs.txt};
\addplot[thick, solid, color=s3] table[x index=0,y index=1,col sep=tab] {img/sample_accuracy.txt};
\addplot[thick, solid, color=s4] table[x index=0,y index=3,col sep=tab] {img/sample_accuracy.txt};
%\addplot[thick, solid, black] coordinates {(0, 1) (4500, 1)};
%\addplot[thick, dashed, black] coordinates {(1500,0) (1500, 2)};
%\addplot[thick, dashed, black] coordinates {(3000,0) (3000, 2)};


\nextgroupplot[ylabel={\footnotesize Accuracy\\ (Power)}, % Power
ytick={0.0,0.3,0.6,0.9,1.0},
yticklabels={,0.3,0.6,0.9,1.0},
yticklabel style={font=\footnotesize},
ymin=0,
ymax=1,
xlabel={\footnotesize \% Samples for training},
xlabel near ticks,
xtick={0,20,40,60,80,100},
xticklabels={0,20,40,60,80,100},
xticklabel style={font=\footnotesize},
]
\addplot[thick, solid, color=s3] table[x index=0,y index=2,col sep=tab] {img/sample_accuracy.txt};
\addplot[thick, solid, color=s4] table[x index=0,y index=4,col sep=tab] {img/sample_accuracy.txt};
%\addplot[thick, dashed, black] coordinates {(1500,0) (1500, 250)};
%\addplot[thick, dashed, black] coordinates {(3000,0) (3000, 250)};

\end{groupplot}
\end{centering}

\end{tikzpicture}

   \vskip -1em
  \caption{Multi Applications Energy}
  \label{fig:multi-energy}
\end{figure*}
%}

\section{Related Work}

We discuss related work in managing resources to meet performance
goals and reduce energy.  

\subsection{Machine Learning}
There are a huge array of different learning techniques that are
applicable to different problems.  As discussed in in \secref{framework:HBM}
we break learning for resource management into three categories:
offline, online, and hybrid approaches.  

\subsubsection{Offline Learning}
The offline approaches build models before deployment and then use
those fixed models to allocate resources
\cite{Yi2003,LeeBrooks2006,CPR,ChenJohn2011,petabricksStatic}.  In
these approaches, the model-building phase is generally very expensive,
requiring both a large number of samples and substantial computation
to turn those samples into a model that accurately captures the
relationship between the observed features and the behavior to be
estimated.  Applying the model online, however, tends to be low
overhead.  The main drawback is that the models are not updated as the
system runs, so there is no chance to correct mistakes or adapt to
specific workloads.

A good example of an offline approach applies learning to render web
pages on mobile systems with low energy \cite{reddiHPCA2013}.  This
system is similar in spirit to \SYSTEM{}.  It builds an offline model
mapping web page features into estimations of performance for
different core types.  When a new page is downloaded, the system
quickly estimates the resource need to render the web page and uses the
lowest energy resources that will still maintain user satisfaction.
The mapping of web pages to resource use is very complicated and this
approach deals with that complication.  It does not, however, address
system dynamics; \eg{} when other apps are running concurrently with the
web browser.

\subsubsection{Online Learning}
Online techniques use observations of the current application to tune
system resource usage for that application
\cite{Li2006,Flicker,ParallelismDial,Ponamarev,petabricksDynamic,LeeBrooks}.
For example, Flicker is a configurable architecture and optimization
framework that uses only online models to maximize performance under a
power limitation \cite{Flicker}.  Another example, ParallelismDial,
uses online adaptation to tailor parallelism to application workload
\cite{ParallelismDial}.



\subsubsection{Hybrid Approaches}
Some approaches combine offline predictive models with online
adaptation
\cite{Zhang2012,packandcap,Winter2010,dubach2010,Koala,Cinder,
  wu2012inferred}.  For example, Dubach et al.  propose such a combo
for optimizing the microarchitecture of a single core
\cite{dubach2010}.  Such predictive models have also been employed at
the operating systems level to manage system energy consumption \cite{Koala,Cinder}.
\cite{wu2012inferred}.


Still other approaches combine offline modeling with online model
updates \cite{JouleGuard,Bitirgen2008,Ipek}.  Bitirgen et
al use an artificial neural network to allocate resources to multiple
applications in a multicore \cite{Bitirgen2008}.  The neural network
is trained offline and then adapted online using measured feedback.
This approach optimizes performance but does not consider power or
energy minimization.  LEO, the system we extend in this paper, also
uses a combination of offline and online approaches.  LEO collects
data about a number of applications offline and combines that with a
small number of observations made online for the current application
\cite{LEO}.

\subsection{Control}
Almost all control solutions can be thought of as a combination of
offline model building with online adaptation.  Usually the model
building involves substantial empirical measurement and statistical
regression to build a model that is then used to synthesize a control
system
\cite{Wu2004,TCST,Chen2011,PTRADE,POET,ControlWare,Agilos,Rajkumar,Sojka,Raghavendra2008}.
The combination of offline
learning and control works well over a narrow range of applications, as the offline models capture the
general behavior of the entire class of application and require
negligible online overhead.  This focused approach is extremely
effective for multimedia applications
\cite{grace2,flinn99,flinn2004,xtune,TCST} and web-servers
\cite{Horvarth,LuEtAl-2006a,SunDaiPan-2008a} because the workloads can
be characterized ahead of time so that the models produce sound
control.

Indeed, the need for good models is the central tension in developing
control for computing systems.  It is always possible to build a
controller for a specific application and system by extensively
modeling that pair.  More general controllers which work with a range
of applications have addressed this issue with models in several ways.
Some provide control libraries that encapsulate control
functionality and require users to input a model
\cite{ControlWare,Sojka,Rajkumar,POET}.  Others
automatically synthesize both a model and a controller for either
hardware \cite{josep-isca2016} or software \cite{ICSE2014,FSE2015}.
JouleGuard combines learning for energy efficiency with control for
managing application parameters \cite{JouleGuard}.  In JouleGuard, a
learner adapts the controller's coefficients to model certainty, but
JouleGuard's learner does not produce a new model for the controller.
Because JouleGuard's learner runs on the same device as the controlled
application, it must be computationally efficient and thus it cannot
identify correlations across applications or even different resource
configurations.  \SYSTEM{} is unique in that a remote server generates
an application-specific model automatically.  By offloading the
learning task, we are able to (1) combine data from many applications
and systems and (2) apply computationally expensive, but highly accurate
learning techniques.

Perhaps the most similar approach to \SYSTEM{} is Carat \cite{carat}.
Carat aggregates data across many mobile devices and sends a report to
human users about how to configure their device to increase battery
life.  While both Carat and \SYSTEM{} learn across devices, they have
very different goals.  Carat's goal is to return very high-level
information to human users; \eg{} you should update a driver to extend
battery life.  \SYSTEM{} returns lower-level models to another
automated system that will apply those models to save energy.


\section{Conclusion}
Much recent work builds systems to support learning, \SYSTEM{} uses
learning to build better systems.  \SYSTEM{} is a resource manager
that meets application latency requirements with minimal energy, even
without prior knowledge of the application.  \SYSTEM{} is the first
work that provides formal guarantees that it will converge to the
required latency despite not having prior knowledge.  \SYSTEM{}
achieves this breakthrough by using learning to model complex resource
interaction and control theory to manage system dynamics.  \SYSTEM{}
proposes foundational techniques that allow control to be applied
using noisy learned models---instead of ground truth models---while
maintaining formal guarantees.  We demonstrate \SYSTEM{}'s
effectiveness with a case study using embedded applications on a
heterogeneous processor.  Compared to prior learning and control
approaches, \SYSTEM{} is the only approach that provides reliable
latency for all applications with near minimal energy.


\newpage
\clearpage
%\bibliographystyle{abbrv} 
%\bibliography{reference}
\printbibliography
\end{document}
